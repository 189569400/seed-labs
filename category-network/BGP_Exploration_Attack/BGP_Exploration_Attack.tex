%%%%%%%%%%%%%%%%%%%%%%%%%%%%%%%%%%%%%%%%%%%%%%%%%%%%%%%%%%%%%%%%%%%%%%
%%  Copyright by Wenliang Du.                                       %%
%%  This work is licensed under the Creative Commons                %%
%%  Attribution-NonCommercial-ShareAlike 4.0 International License. %%
%%  To view a copy of this license, visit                           %%
%%  http://creativecommons.org/licenses/by-nc-sa/4.0/.              %%
%%%%%%%%%%%%%%%%%%%%%%%%%%%%%%%%%%%%%%%%%%%%%%%%%%%%%%%%%%%%%%%%%%%%%%

\newcommand{\commonfolder}{../../common-files}

\documentclass[11pt]{article}

\usepackage[most]{tcolorbox}
\usepackage{times}
\usepackage{epsf}
\usepackage{epsfig}
\usepackage{amsmath, alltt, amssymb, xspace}
\usepackage{wrapfig}
\usepackage{fancyhdr}
\usepackage{url}
\usepackage{verbatim}
\usepackage{fancyvrb}
\usepackage{adjustbox}
\usepackage{listings}
\usepackage{color}
\usepackage{subfigure}
\usepackage{cite}
\usepackage{sidecap}
\usepackage{pifont}
\usepackage{mdframed}
\usepackage{textcomp}
\usepackage{enumitem}


% Horizontal alignment
\topmargin      -0.50in  % distance to headers
\oddsidemargin  0.0in
\evensidemargin 0.0in
\textwidth      6.5in
\textheight     8.9in 

\newcommand{\todo}[1]{
\vspace{0.1in}
\fbox{\parbox{6in}{TODO: #1}}
\vspace{0.1in}
}


\newcommand{\unix}{{\tt Unix}\xspace}
\newcommand{\linux}{{\tt Linux}\xspace}
\newcommand{\minix}{{\tt Minix}\xspace}
\newcommand{\ubuntu}{{\tt Ubuntu}\xspace}
\newcommand{\setuid}{{\tt Set-UID}\xspace}
\newcommand{\openssl} {\texttt{openssl}}


\pagestyle{fancy}
\lhead{\bfseries SEED Labs}
\chead{}
\rhead{\small \thepage}
\lfoot{}
\cfoot{}
\rfoot{}


\definecolor{dkgreen}{rgb}{0,0.6,0}
\definecolor{gray}{rgb}{0.5,0.5,0.5}
\definecolor{mauve}{rgb}{0.58,0,0.82}
\definecolor{lightgray}{gray}{0.90}


\lstset{%
  frame=none,
  language=,
  backgroundcolor=\color{lightgray},
  aboveskip=3mm,
  belowskip=3mm,
  showstringspaces=false,
%  columns=flexible,
  basicstyle={\small\ttfamily},
  numbers=none,
  numberstyle=\tiny\color{gray},
  keywordstyle=\color{blue},
  commentstyle=\color{dkgreen},
  stringstyle=\color{mauve},
  breaklines=true,
  breakatwhitespace=true,
  tabsize=3,
  columns=fullflexible,
  keepspaces=true,
  escapeinside={(*@}{@*)}
}

\newcommand{\newnote}[1]{
\vspace{0.1in}
\noindent
\fbox{\parbox{1.0\textwidth}{\textbf{Note:} #1}}
%\vspace{0.1in}
}


%% Submission
\newcommand{\seedsubmission}{You need to submit a detailed lab report, with screenshots,
to describe what you have done and what you have observed.
You also need to provide explanation
to the observations that are interesting or surprising.
Please also list the important code snippets followed by
explanation. Simply attaching code without any explanation will not
receive credits.}

%% Book
\newcommand{\seedbook}{\textit{Computer \& Internet Security: A Hands-on Approach}, 2nd
Edition, by Wenliang Du. See details at \url{https://www.handsonsecurity.net}.}

%% Videos
\newcommand{\seedisvideo}{\textit{Internet Security: A Hands-on Approach},
by Wenliang Du. See details at \url{https://www.handsonsecurity.net/video.html}.}

\newcommand{\seedcsvideo}{\textit{Computer Security: A Hands-on Approach},
by Wenliang Du. See details at \url{https://www.handsonsecurity.net/video.html}.}

%% Lab Environment
\newcommand{\seedenvironment}{This lab has been tested on our pre-built
Ubuntu 16.04 VM, which can be downloaded from the SEED website. }

\newcommand{\seedenvironmentA}{This lab has been tested on our pre-built
Ubuntu 16.04 VM, which can be downloaded from the SEED website. }

\newcommand{\seedenvironmentB}{This lab has been tested on our pre-built
Ubuntu 20.04 VM, which can be downloaded from the SEED website. }

\newcommand{\seedenvironmentAB}{This lab has been tested on our pre-built
Ubuntu 16.04 and 20.04 VMs, which can be downloaded from the SEED website. }

\newcommand{\nodependency}{Since we use containers to set up the lab environment, 
this lab does not depend too much on our SEED VM. You can do this lab
using other VMs or physical machines. }







\newcommand{\seedlabcopyright}[1]{
\vspace{0.1in}
\fbox{\parbox{6in}{\small Copyright \copyright\ {#1}\ \ by Wenliang Du.\\
      This work is licensed under a Creative Commons
      Attribution-NonCommercial-ShareAlike 4.0 International License.
      If you remix, transform, or build upon the material, 
      this copyright notice must be left intact, or reproduced in a way that is reasonable to
      the medium in which the work is being re-published.}}
\vspace{0.1in}
}






\lhead{\bfseries SEED Labs -- BGP Exploration and Attack}
\newcommand{\bgpFigs}{./Figs}

\begin{document}


\begin{center}
{\LARGE BGP Exploration and Attack Lab}

  \vspace{0.1in}
  {\LARGE (Work in Progress)}
\end{center}

\seedlabcopyright{2021}

\tableofcontents




% *******************************************
% SECTION
% ******************************************* 
\section{Overview}

Border Gateway Protocol (BGP) is the standard exterior gateway protocol
designed to exchange routing and reachability information among autonomous
systems (AS) on the Internet. It is the ``glue'' of the Internet,
and is an essential piece of the Internet infrastructure. It is 
also a primary attack target, because if attackers can 
compromise BGP, they can disconnect the Internet and redirect traffics. 

Because of the complexity of BGP, it is hard to do everything in a single lab. 
Therefore, we have developed a series of labs related to BGP. This lab
is the first in the series, and it is the basis for all the other BGP labs.  
The goal of this lab is to help students understand how
BGP ``glues'' the Internet together, and how the Internet is actually
connected. In this lab, we guide students to build a small-scale Internet.
We call this Internet the Internet Simulator (or simply
Simulator in short). This simulator will be the basis for 
all other BGP labs, as well as for some non-BGP labs that 
depends on the Internet. 
This lab covers the following topics:
\begin{itemize}[noitemsep]
\item How the BGP protocol works
\item BGP configuration
\item Routing 
\item Internet Exchange Point (IXP)
\item BGP attack
\end{itemize}


\paragraph{Videos.}
Detailed coverage of the BGP protocol can be found in 
Section 10 of the SEED Lecture at Udemy, \seedisvideo 
The lecture was recorded before this lab was developed; 
it focuses mostly on the theory part, i.e., explaining how the BGP protocol works. 
This lab provides the practical part.  


\paragraph{Lab environment.} 
\seedenvironmentB
\nodependency


\paragraph{Acknowledgment.} 
This lab was developed with the help of Honghao Zeng, a graduate student
in the Department of Electrical Engineering and Computer Science at Syracuse University.
The SEED project is funded by the US National Science Foundation. 


% *******************************************
% SECTION
% *******************************************
\section{The Internet Emulator} 

This lab will be performed inside the SEED Internet Emulator. We will
provide a pre-built emulator in two different forms: (1) Python code,
and (2) container files. The container files are generated from
the Python code. However, students need to install the SEED Emulator source 
code from the GitHub to run the Python code. The container files
can be directly used without the Emulator source code. 

Instructors who would like to customize the emulator can modify the Python
code, generate their own container files, and then provide these
container files to students, replacing the ones included in the 
setup file. 



% *******************************************
% SECTION
% *******************************************
\section{Understanding the Configuration in BIRD} 

The routing software used in this lab is called BIRD, which is 
an acronym standing for ``BIRD Internet Routing Daemon''~\cite{bird}.
It is open source under the GNU General Public License. It 
supports a number of standard routing protocols, including 
the Border Gateway Protocol (BGP),
the Routing Information Protocol (RIP),
and the Open Shortest Path First protocol (OSPF). We will
mainly use BGP and OSPF in this lab. 


BIRD takes the configuration
file \texttt{bird.conf} from the \texttt{/etc/bird/} folder.
In this lab, students need to understand the entries in the 
configuration file. Students also need to make changes
to the file for some tasks. In this section, we provide 
a tutorial to help students understand the 
configuration file. It should be noted that the actual configuration
for BIRD is quite complicated, and most configuration entries 
are protocol specific. Full details of the configuration 
can be found in the BIRD manual~\cite{birdmanual}.



% -------------------------------------------
% SUBSECTION
% -------------------------------------------
\subsection{Routing Tables} 


BIRD has has several routing tables in memory. It uses
the term ``protocol'' to specify how 
to connect a routing table with another entity. Exactly
what entity is connected to depends on the type of the protocol.

If the protocol is a routing protocol, 
such as BGP, OSPF, or RIP, the entity will be an external router.
Namely, the routes received from the external router
are imported to the routing table, while 
the routes stored in the routing table are exported 
to the external router.

Some protocols used in BIRD are not not real protocols; 
BIRD simply uses the keyword \texttt{protocol} to specify where
a BIRD routing table gets its routes from, and where the routes
in the table need to be sent to.

% -------------------------------------------
% SUBSECTION
% -------------------------------------------
\subsection{Several Special Protocols} 

BIRD has a few special types of ``protocol'', which are not
real protocols, however, they do generate route information,
except that the generation is not through a protocol. 
The followings
are the three important ``fake'' protocols:

\begin{lstlisting}[caption={The \texttt{bird.conf} file for \texttt{AS150}}]
protocol device {
}

protocol direct {
    interface "eth0";
}

protocol kernel {
    import none;
    export all;
}
\end{lstlisting}

To define a routing protocol in BIRD, we use the \texttt{protocol} keyword.
BIRD supports many different routing protocols, like BGP and OSPF. A
\texttt{protocol} can import or export routes to the BIRD's routing table.
For this lab, we will use either \texttt{all} or \texttt{none}. In future
BGP labs, we will introduce filters here, so you can set up
rules to decide what routes can be imported or exported.

\begin{itemize}

\item The \texttt{device protocol} is not a real routing protocol. It doesn't
generate any route, nor does it accept any route. It is only used to get
information about the network interfaces from the kernel. Without the
\texttt{device protocol}, BIRD will know nothing about the network
interfaces; it will not even be able to run BGP, as it does not know how
to reach BGP peer's IP address. Therefore, this block is mandatory.

\item The \texttt{direct protocol} is for generating routes for the directly
connected networks from a list of interfaces. In the \texttt{direct protocol}
block, the \textit{interface} keyword is used to generate routes from an
interface. In this case, \texttt{AS150}'s bgp router uses \texttt{eth0} to
connect to \texttt{AS150}'s internal network (\texttt{10.150.0.0/24}).
Therefore, this \texttt{direct protocol} block will generate a routing
entry for \texttt{10.150.0.0/24}. BGP will announce this network
prefix to the peers.

\item The \texttt{kernel} protocol is not a real routing protocol like BGP or
OSPF. Instead of communicating with other routers in the network, it connects
BIRD's routing table to kernel's routing table. It is the kernel's routing table
that is used in the actual routing, not BIRD's routing table.
This protocol is used to set the kernel routing table using the routes
received from BGP peers. Here
\texttt{import none} means BIRD's routing table will not import anything from
the kernel's routing table; \texttt{export all} means BIRD's routing table
will export all the routes to the kernel's routing table. This is how BGP
routers set the routing table using the data collected from the BGP
protocol.
\end{itemize}


% -------------------------------------------
% SUBSECTION
% -------------------------------------------
\subsection{The Static Protocol} 

A special type of protocol is called \texttt{static}, which is 
useful in our BGP attack task. Just like the other special
types of protocols, this protocol is not a real protocol either.
It does not import routes from a peer, instead, it provides 
predefined routes that need to be imported into 
the routing table. Each route can have a route-specific
filter. This is especially useful for configuring route attributes. 

When specifying a route, we can either specify a router, or 
specifying a special target name, indicating the actions 
that need to be performed on the matching packets. 
A target name is \texttt{blackhole},
indicating that packet to the specified destination should be 
dropped. See the following example.

\begin{lstlisting}
protocol static {
    ipv4 {
        table t_bgp;  # Connect to a non-default routing table
    };
    route 10.152.0.0/24 via 10.3.0.100 { ... filter ... };
    route 10.153.0.0/24 blackhole { ... filter ... };
}
\end{lstlisting}


% -------------------------------------------
% SUBSECTION
% -------------------------------------------
\subsection{The BGP Protocol} 

The most important protocol relevant to this lab is the BGP protocol.
This is a real protocol. For each BGP peering, an instance of 
such a protocol needs to be specified in the BIRD configuration 
file. The protocol connects a routing table with another 
BGP router, i.e., it imports/exports rules from/to 
the peer specified in the protocol.
See the following example. 

\begin{lstlisting}
protocol bgp xyz {
    local    10.104.0.3   as 3;
    neighbor 10.104.0.104 as 104;     (*@\pointleft{BGP peer}@*)
    ipv4 {
        table t_bgp;
        import all;        (*@\pointleft{all the routes from peer will be imported}@*)
        export filter {    (*@\pointleft{export selected routes to peer}@*)
           if proto = "static" then reject;
           accept;
        };
    };
}
\end{lstlisting}

The \texttt{local} option in the BGP protocol sets the local IP address and ASN of the BGP session.
The syntax is \texttt{"local [ip\_address] as <asn>"}. The \texttt{[ip\_address]} part is optional,
but it makes the configuration looks clearer and can prevent selecting the wrong IP address for the
BGP session when there are multiple IP addresses on the router. This
IP address should be the one on the IX's network.

The \texttt{neighbor} option in the BGP protocol sets the neighbor IP address and ASN of the BGP
session. This is the actual peering part. In this example, we set up a BGP session
between \texttt{AS3} and \texttt{AS104}, so they can exchange route information using the BGP protocol.
Similarly, the IP address here should be the one on the IX's network.


\paragraph{Channel and table.}
Each protocol is connected to a routing table through a channel. 
BGP supports both IPv4 and IPv6 channels. 
Each channel has two filters, export and import filters, 
which can accept, reject and modify the routes. 
The export filter applies to the routes going from the routing table to the protocol, 
while the import filter applies to the routes coming into the routing table. 

In the example above, the routing table in the IPv4 channel is 
specified (\texttt{t\_bgp}). If no table is specified, the 
default table used is \texttt{master4} (for IPv6, it is \texttt{master6}). The 
\texttt{master4} table is BIRD's master routing table for IPv4 routes. It should 
be noted that this table is not the operating system's routing table, i.e.,
it is not the one used for the actual routing. The one used for the actual
routing is called the kernel routing table. The \texttt{kernel} protocol
described earlier specifies how the routes in  BIRD's master routing
table are exported to the kernel routing table. 

\paragraph{Filter.}
When importing/exporting routes to/from the routing table, 
filter rules can be applied. BIRD contains a simple programming language,
so filter rules are programs.
When a route is being passed between protocols and routing tables, 
the corresponding filter will be are interpreted by BIRD.
The filter will get the route, so it can inspect the route,
the attributes, and make changes to the route if needed. 
At the end, the filter decides
whether to pass the route through (using \texttt{accept}) 
or reject it (using \texttt{reject}). 
Here are some filter examples:


\begin{lstlisting}

\end{lstlisting}
 


% -------------------------------------------
% SUBSECTION
% -------------------------------------------
\subsection{The Pipe Protocol} 

For IPv4, only the route stored in BIRD's master table \texttt{master4} will be eventually
exported to the kernel's routing table (via the \texttt{kernel} protocol). 
If we specify a customized table
inside a protocol, the routes will not be stored in \texttt{master4},
and will therefore not affect the routing. We need to tell BIRD
to export the routes in those tables to the master table. This is done 
through another special type of protocol called \texttt{pipe}. 

The Pipe protocol links two routing tables, a primary table (specified using
the \texttt{table} keyword) and a secondary table (specified using 
the \texttt{peer} keyword). Filters can be specified for the import
and export directions. In the following example, all the entries
in the \texttt{t\_bgp} table are exported to the \texttt{master4} table,
while none of the routes in the \texttt{master4} are imported. 

\begin{lstlisting}
protocol pipe {
    table t_bgp;
    peer table master4;
    import none;
    export all;
}
\end{lstlisting}

In the following example, all the route entries from the \texttt{t\_direct}
table are ported to the \texttt{t\_bgp} table, and the local preference
attribute of the routes are set to \texttt{40}.

\begin{lstlisting}
protocol pipe {
    table t_direct;
    peer table t_bgp;
    import none;
    export filter { bgp_local_pref = 40; accept; };
}
\end{lstlisting}
 


% *******************************************
% SECTION
% *******************************************
\section{Task 1: Stub Autonomous System} 


% -------------------------------------------
% SUBSECTION
% -------------------------------------------
\subsection{Task 1.a: Exploration} 

Pick a BGP router from the emulator. 
Design the following activities.

\begin{itemize}
  \item Start with a stub AS (with two upstream ASes)
  \item Study the BGP configuration file, and understand its entries
  \item Inspect the BGP route table 
  \item Disable/Enable BGP sessions
  \item Simple filter  
  \item Path selection (use the filter to change the other attributes,
    see how it affects the path selection). Find two different routes for 
    a destination, changing the path attributes to affect how a route 
    is selected. 
\end{itemize}


\paragraph{Advanced features.} 
These are advanced features. Not sure whether they should be included or not.
We temporarily list them here.

\begin{itemize}
  \item Community: label for the route
\end{itemize}


\begin{lstlisting}
# birdc configure 
\end{lstlisting}
 


% -------------------------------------------
% SUBSECTION
% -------------------------------------------
\subsection{Task 1.b: Configuring AS-180} 

This AS is already included in the emulator. It connects to the
IX105 Internet exchange (Houston), but it does not peer with anybody. 
In this task, students need to do the following:


\begin{itemize}[noitemsep]
  \item Peer AS-180 with AS-171, so they can directly reach each other
  \item Peer with AS-2 and AS-3 transit autonomous systems, so they can
    reach other destination via these transits
\end{itemize}


\paragraph{Submission.}
In your lab report, please include the content you add to the 
BIRD configuration file, and provide proper explanation.
Please also include screenshots (such as traceroute) to demonstrate 
that your task is successful. 


% *******************************************
% SECTION
% *******************************************
\section{Task 2: Transit Autonomous System} 

If two ASes want to connect, they can peer with each other
at an Internet exchange point. The question is how two ASes in two different locations
get connected to each other. It is hard for them
to find a common location to peer. To solve this problem,
a special type of AS is needed.

This type of AS have BGP routers in many Internet
Exchange Points, where they peer with many other ASes. Once packets get into
its networks, they will be pulled from one IX to another IX (typically via
some internal routers), and eventually
hand it over to another AS. This type of AS provides the transit
service for other ASes. That is how the hosts in one AS can reach the hosts in
another AS, even though these ASes are not peers with each other.
This special of AS is called \textit{Transit AS}.
In this task, we will first understand how a transit AS works and
then we will configure a transit AS in our Internet emulator.  


% -------------------------------------------
% SUBSECTION
% -------------------------------------------
\subsection{Reading: Understanding IBGP} 

Just like the peering of BGP routers from different
autonomous systems, for the BGP routers in the same
autonomous systems to exchange information, they
also need to peer with each other and run the BGP
protocol to exchange information.
The BGP protocol conducted by the BGP routers
inside the same AS is called IBGP (Internal BGP).

When we establish a BGP session between two routers with the same ASN, it will be considered
as an IBGP session, and when the session is between two routers with different ASNs, the session
is considered as an EBGP session (External BGP).
Therefore, the way to define
an IBGP session is the same as defining an EBGP session.
However IBGP sessions have several different behaviors 
than the EBGP sessions.


\begin{itemize}
\item In IBGP sessions, when sending routes to peers,
routers will not prepend their own ASN in the \texttt{AS\_PATH},
and the \texttt{nexthop} attribute will not be altered either.

\item RFC 4271 prohibits the advertisement of the routes received from 
an IBGP peer to another IBGP peer; otherwise, there will be a loop.
This is because IBGP does not add their own information
to the AS path, BGP routers will not be able to know whether
their peers already know the AS path or not. If forwarding
is enabled, a BGP router will keep forwarding information
to their peers, creating loops. 
Because there is no forwarding, 
RFC 4271 states that all the BGP routers within a single AS must be 
fully meshed, i.e., all BGP routers peer with one another internally.
\end{itemize}


Let's use AS-3 to understand how the IBGP peering works inside 
this transit AS. 
This AS has four BGP routers, and we pick the one 
connected to the IX-103 (Miami) Internet exchange.
The following is the IBGP configuration on this router:

\begin{lstlisting}
protocol bgp ibgp1 {
    ipv4 {
        table t_bgp;
        import all;
        export all;
        igp table t_ospf;
    };
    local    10.0.0.7 as 3;
    neighbor 10.0.0.6 as 3;
}
protocol bgp ibgp2 {
    ipv4 {
        ... same is ibgp1 ...
    };
    local    10.0.0.7 as 3;
    neighbor 10.0.0.5 as 3;
}
protocol bgp ibgp3 {
    ipv4 {
        ... same is ibgp1 ...
    };
    local    10.0.0.7 as 3;
    neighbor 10.0.0.8 as 3;
}
\end{lstlisting}


\paragraph{Loopback interface.}
A BGP router has multiple IP address (one for each network interface), so which IP address
should be used in the IBGP peering? Any of the IP addresses will work, but 
if we use the IP address of a particular network interface, if that interface 
goes down or get disabled, the IP address become unreachable, and the 
peering using the IP address will fail. 

To solve this problem, it is suggested that a loopback interface is used 
in the IBGP peering. The loopback interface is virtual and always stays up. 
Therefore, the IGBP session can still remain intact even if some other 
interfaces fail. In the event of link failure, the interior routing 
protocol will automatically find an alternate path to the peer's loopback
IP address. 

In the configuration, each of the \texttt{10.0.0.x/32} addresses is 
the IP address of the loopback interface (called \texttt{dummy} in our setup). 
These addresses are announced via the OSPF routing protocol, so all the 
routers inside the same AS know how to reach these addresses. 



\paragraph{The \texttt{NEXT\_HOP} attribute.}
When a BGP router sends a route to an internal BGP peer that is multiple hops
away, the \texttt{NEXT\_HOP} route attribute makes no sense to the peer, 
because this next hop (assuming it is \texttt{X}) is the addresses of a 
boundary routers directly connected to the sender, not the receiver. 
The peer must figure out what the immediate next hop is in order to reach \texttt{X}, 
and the distance to \texttt{X} (for path selection purpose). 

To achieve that, after receiving a route from an internal BGP peer,
the BGP router will do a route lookup using a routing table.
This routing table is an IGP routing table containing the AS-internal routes. 
The \texttt{igp table} entry specifies this IGP routing table. 



% -------------------------------------------
% SUBSECTION
% -------------------------------------------
\subsection{Task 2.a: IBGP Exploration} 


For the task, we first need to find some traffics that 
go through AS-3. We will ping \texttt{10.164.0.71} from a host in AS-162. Using the 
map client program, we can see that the packets go through
the AS-3 transit AS. If this is not consistent with your observation,
do find some other traffics that go through AS-3. 

We will now disable the IBGP sessions either the map client or 
from the command line. See the following example.

\begin{lstlisting}
bird> show protocols
Name       Proto      Table      State  Since         Info
...
ibgp1      BGP        ---        up     20:19:03.800  Established
ibgp2      BGP        ---        up     20:19:11.921  Established
ibgp3      BGP        ---        up     20:20:50.238  Established

bird> disable ibgp3 
bird> show protocols ibgp3
Name       Proto      Table      State  Since         Info
ibgp3      BGP        ---        down   20:26:44.526
\end{lstlisting}
 
Before disabling IBGP, show the routing table 
on the BGP router (using \texttt{"ip route"}). Compare the 
results before and after disabling IBGP, and explain
your observations. 


% -------------------------------------------
% SUBSECTION
% -------------------------------------------
\subsection{Reading: Understanding IGP} 

Routers inside an autonomous system need to
communicate with each other, so they can tell each other
what networks they are connected to, so others can figure out
the best path to get to those networks. That is what
was missing in the previous task.


This type of routing protocol is called IGP (Interior Gateway Protocol).
There are several IGP protocols, including OSPF (a link state routing protocol) and RIP (a
distance-vector routing protocol). BIRD supports both of them. In this
task, we will only use the OSPF protocol.

Let us add the OSPF protocol to our BIRD configuration file. In the following,
we create a \texttt{"protocol ospf"} block and specify a table called \texttt{t\_ospf}.
That means all the OSPF routing information will be stored in
this table. Inside the \texttt{"protocol bgp ibgp"} block, we tell BIRD to
use the \texttt{t\_ospf} table to resolve the nexthop for
the route obtained from the internal BGP peers.

\begin{lstlisting}
table t_ospf;  # Define a new table 
ipv4 table t_ospf;
protocol ospf ospf1 {
    ipv4 {
        table t_ospf;
        import all;
        export all;
    };
    area 0 {
        interface "ix104" { stub; };
        interface "net_103_104" { hello 1; dead count 2; };
    };
}

# Create a pipe between the OSPF and master tables. 
protocol pipe {
    table t_ospf;
    peer table master4;
    import none;
    export all;
}
\end{lstlisting}

Detailed configuration of OSPF can be quite complicated, and it is beyond the
scope of this lab.
An area with ID 0 is called a backbone area. All other areas need
to be connected to the backbone area. In our configuration, we put all routes
in the backbone area to make things simple. This
particular BGP router (\texttt{AS2}'s router in \texttt{IX100})
is connected to two networks, one through \texttt{eth0} and the other
through \texttt{eth1}.

Information from both networks will be included in the OSPF protocol,
so others know how to reach these two networks. However,
while \texttt{eth0} is connected to the internal network,
\texttt{eth1} is the interface used by the router
to connect to an outside network, the IXP's network. This network is provided
by IXP for peering purposes. We do need to include this network in the OSPF
protocol, so the internal router knows how to reach this network. However, we
will not run OSPF protocol with anybody on this network, because they
do not belong to \texttt{AS2}; they are outsiders.
You do not want to leak the internal network information to the outside;
nor do you want the outsider to manipulate your internal routes using OSPF.
Therefore, you should not run the OSPF protocol with
the routers on this external network.
You only run OSPF with the internal routers.
That is why we use \texttt{stub}, meaning the information from this network
will be used in the OSPF protocol, but we will not run OSPF on
this network.

Explain the \texttt{hello 1; dead count 2;} part. 



% -------------------------------------------
% SUBSECTION
% -------------------------------------------
\subsection{Task 2.b: IGP Exploration} 

In this task, we will disable the OSPF routing protocol, and see 
how it affects the routing. There are several ways to disable
OSPF. One way is to do it inside \texttt{birdc}: 

\begin{lstlisting}
# bridc
birdc> show protocols
...
ospf1      OSPF       t_ospf     up     19:49:43.343  Running
...

birdc> disable ospf1
birdc> show protocols ospf1
ospf1      OSPF       t_ospf     down   19:57:37.187
\end{lstlisting}
 
Another way is to modify the \texttt{bird.conf} configuration file,
changing the \texttt{"import/export all"} to 
\texttt{"import/export none"}. This will keep OSPF running, but  
no routes obtained from the protocol will be used. 

\begin{lstlisting}
protocol ospf ospf1 {
    ipv4 {
        table t_ospf;
        import none;  (*@\pointleft{import no route from peer}@*) 
        export none;  (*@\pointleft{export no route to peer}@*) 
    };
    ...
}
\end{lstlisting}
 

\paragraph{Task.} We pick AS-3 transit autonomous system in this 
task. This AS has four BGP routers, and we pick the one 
connected to the IX-103 (Miami) Internet exchange.
For the task, we also need to find some traffics that 
go through AS-3. 
We will ping \texttt{10.164.0.71} from a host in AS-162. Using the 
map client program, we can see that the packets go through
the AS-3 transit AS. If this is not consistent with your observation,
do find some other traffics that go through AS-3. 


We will now disable the OSPF using one of the methods mentioned above.
Before disabling OSPF, show the routing table 
on the BGP router (using \texttt{"ip route"}). Compare the 
results before and after disabling OSPF. Based on the 
observation, explain why the IGP is essential for the transit 
autonomous systems. 



% *******************************************
% SECTION
% *******************************************
\section{Task 3: Path Selection} 

Use filter to conduct path-selection experiments, helping students
understand how the path selection works.
These are the top 2 rules used by BIRD: 
(1) prefer route with the highest Local Preference attribute,
the path with the highest local preference is preferred;
(2) prefer route with the shortest AS path.

Try the followings:
\begin{itemize}
  \item Change \texttt{bgp\_local\_pref} 
  \item Prepend AS: change the AS path: \texttt{bgp\_path.prepend(1000)} 
\end{itemize}
 

We pick AS-150, which peers with both AS-2 and AS-3. 
We will use AS2 as the backup upstream link.

\begin{lstlisting}
protocol bgp u_as2 {
    ipv4 {
        table t_bgp;
        import filter {
            bgp_large_community.add(PROVIDER_COMM);
            bgp_local_pref = 10;
	    bgp_path.prepend(2);
	    bgp_path.prepend(2);
            accept;
        };
        export where bgp_large_community ~ [LOCAL_COMM, CUSTOMER_COMM];
        next hop self;
    };
    local    10.100.0.150 as 150;
    neighbor 10.100.0.2   as 2;
\end{lstlisting}
 


% *******************************************
% SECTION
% *******************************************
\section{Task N: IP Anycast} 

Provide the AS (IP address T is an IP anycast address). 
Ask students to conduct the following experiment:

\begin{itemize}
  \item Inspect the BGP route table to identify the route.
  \item From A and B, ping T. They get to two different T.
  \item Break a BGP session, so they get to the same T.
  \item Use UDP instead of ping.
\end{itemize}
 


% *******************************************
% SECTION
% *******************************************
\section{Task N: BGP Attacks} 


% -------------------------------------------
% SUBSECTION
% -------------------------------------------
\subsection{BGP Session Reset Attack} 



Countermeasure: TTL security switch (RFC 5082).
Design an experiment to demonstrate its effect.

\begin{lstlisting}
protocol bgp c_as12 {
    ipv4 {
        table t_bgp;
        import filter {
	   ...
        };
        export all;
        next hop self;
    };
    local 10.104.0.3 as 3;
    neighbor 10.104.0.12 as 12;
    ttl security;
}
\end{lstlisting}
 

% -------------------------------------------
% SUBSECTION
% -------------------------------------------
\subsection{IP Prefix Hijacking} 



\begin{lstlisting}
protocol static hijacks {
    ipv4 {
        table t_bgp;
    };
    route 10.153.0.0/25 blackhole   { bgp_large_community.add(LOCAL_COMM); };
    route 10.153.0.128/25 blackhole { bgp_large_community.add(LOCAL_COMM); };
}
\end{lstlisting}
 


% *******************************************
% SECTION
% ******************************************* 
\section{Submission}

%%%%%%%%%%%%%%%%%%%%%%%%%%%%%%%%%%%%%%%%

You need to submit a detailed lab report, with screenshots,
to describe what you have done and what you have observed.
You also need to provide explanation
to the observations that are interesting or surprising.
Please also list the important code snippets followed by
explanation. Simply attaching code without any explanation will not
receive credits.

%%%%%%%%%%%%%%%%%%%%%%%%%%%%%%%%%%%%%%%%


\end{document}



