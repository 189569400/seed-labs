%%%%%%%%%%%%%%%%%%%%%%%%%%%%%%%%%%%%%%%%%%%%%%%%%%%%%%%%%%%%%%%%%%%%%%
%%  Copyright by Collin Howland, Faith Letzkus, and Julio Trujillo  %%
%%  at Washington University in St. Louis. 			    %%
%%  This work is licensed under the Creative Commons                %%
%%  Attribution-NonCommercial-ShareAlike 4.0 International License. %%
%%  To view a copy of this license, visit                           %%
%%  http://creativecommons.org/licenses/by-nc-sa/4.0/.              %%
%%%%%%%%%%%%%%%%%%%%%%%%%%%%%%%%%%%%%%%%%%%%%%%%%%%%%%%%%%%%%%%%%%%%%%

\documentclass[11pt]{article}

\usepackage[most]{tcolorbox}
\usepackage{times}
\usepackage{epsf}
\usepackage{epsfig}
\usepackage{amsmath, alltt, amssymb, xspace}
\usepackage{wrapfig}
\usepackage{fancyhdr}
\usepackage{url}
\usepackage{verbatim}
\usepackage{fancyvrb}
\usepackage{adjustbox}
\usepackage{listings}
\usepackage{color}
\usepackage{subfigure}
\usepackage{cite}
\usepackage{sidecap}
\usepackage{pifont}
\usepackage{mdframed}
\usepackage{textcomp}
\usepackage{enumitem}


% Horizontal alignment
\topmargin      -0.50in  % distance to headers
\oddsidemargin  0.0in
\evensidemargin 0.0in
\textwidth      6.5in
\textheight     8.9in 

\newcommand{\todo}[1]{
\vspace{0.1in}
\fbox{\parbox{6in}{TODO: #1}}
\vspace{0.1in}
}


\newcommand{\unix}{{\tt Unix}\xspace}
\newcommand{\linux}{{\tt Linux}\xspace}
\newcommand{\minix}{{\tt Minix}\xspace}
\newcommand{\ubuntu}{{\tt Ubuntu}\xspace}
\newcommand{\setuid}{{\tt Set-UID}\xspace}
\newcommand{\openssl} {\texttt{openssl}}


\pagestyle{fancy}
\lhead{\bfseries SEED Labs}
\chead{}
\rhead{\small \thepage}
\lfoot{}
\cfoot{}
\rfoot{}


\definecolor{dkgreen}{rgb}{0,0.6,0}
\definecolor{gray}{rgb}{0.5,0.5,0.5}
\definecolor{mauve}{rgb}{0.58,0,0.82}
\definecolor{lightgray}{gray}{0.90}


\lstset{%
  frame=none,
  language=,
  backgroundcolor=\color{lightgray},
  aboveskip=3mm,
  belowskip=3mm,
  showstringspaces=false,
%  columns=flexible,
  basicstyle={\small\ttfamily},
  numbers=none,
  numberstyle=\tiny\color{gray},
  keywordstyle=\color{blue},
  commentstyle=\color{dkgreen},
  stringstyle=\color{mauve},
  breaklines=true,
  breakatwhitespace=true,
  tabsize=3,
  columns=fullflexible,
  keepspaces=true,
  escapeinside={(*@}{@*)}
}

\newcommand{\newnote}[1]{
\vspace{0.1in}
\noindent
\fbox{\parbox{1.0\textwidth}{\textbf{Note:} #1}}
%\vspace{0.1in}
}


%% Submission
\newcommand{\seedsubmission}{You need to submit a detailed lab report, with screenshots,
to describe what you have done and what you have observed.
You also need to provide explanation
to the observations that are interesting or surprising.
Please also list the important code snippets followed by
explanation. Simply attaching code without any explanation will not
receive credits.}

%% Book
\newcommand{\seedbook}{\textit{Computer \& Internet Security: A Hands-on Approach}, 2nd
Edition, by Wenliang Du. See details at \url{https://www.handsonsecurity.net}.}

%% Videos
\newcommand{\seedisvideo}{\textit{Internet Security: A Hands-on Approach},
by Wenliang Du. See details at \url{https://www.handsonsecurity.net/video.html}.}

\newcommand{\seedcsvideo}{\textit{Computer Security: A Hands-on Approach},
by Wenliang Du. See details at \url{https://www.handsonsecurity.net/video.html}.}

%% Lab Environment
\newcommand{\seedenvironment}{This lab has been tested on our pre-built
Ubuntu 16.04 VM, which can be downloaded from the SEED website. }

\newcommand{\seedenvironmentA}{This lab has been tested on our pre-built
Ubuntu 16.04 VM, which can be downloaded from the SEED website. }

\newcommand{\seedenvironmentB}{This lab has been tested on our pre-built
Ubuntu 20.04 VM, which can be downloaded from the SEED website. }

\newcommand{\seedenvironmentAB}{This lab has been tested on our pre-built
Ubuntu 16.04 and 20.04 VMs, which can be downloaded from the SEED website. }

\newcommand{\nodependency}{Since we use containers to set up the lab environment, 
this lab does not depend too much on our SEED VM. You can do this lab
using other VMs or physical machines. }






\lhead{\bfseries SEED Labs -- Clickjacking Lab}

\begin{document}

\begin{center}
{\LARGE Clickjacking}
\end{center}

\vspace{0.1in}
\fbox{\parbox{6in}{\small Copyright \copyright\ 2021 \ \ by Collin
      Howland, Faith Letzkus, Julio Trujillo, and Steve Cole at  
      Washington University in St.\ Louis.  \\
      This work is licensed under a Creative Commons
      Attribution-NonCommercial-ShareAlike 4.0 International License. 
      If you remix, transform, or build upon the material, 
      this copyright notice must be left intact, or reproduced in a way
      that is reasonable to the medium in which the work is being 
      re-published.}}
\vspace{0.1in}

% *******************************************
% Overview
% ******************************************* 
\section{Overview}

\paragraph{} Clickjacking, also known as a ``UI redress attack,'' is an
attack that tricks a user into clicking on something they do not intend
to when visiting a webpage, thus ``hijacking'' the click. In this lab,
we will explore a common attack vector for clickjacking: the attacker
creates a malicious webpage that loads the content of a legitimate page
but overlays one or more of its buttons with invisible button(s) that
trigger malicious actions.  When a user attempts to click on the
legitimate page's buttons, the browser registers a click on the
invisible button instead, triggering the malicious action.

\paragraph{Example scenario} Suppose an attacker acquires the domain
``starbux.com'' and creates a website with that URL. The site first
loads the legitimate target website ``starbucks.com'' in an iframe
element spanning the entire webpage, so that the
malicious``starbux.com'' website looks identical to the legitimate
``starbucks.com'' website.  The attacker's site then places an invisible
button on top of the `Menu' button on the displayed ``starbucks'' page;
the button triggers a 1-click purchase of the attacker's product on
Amazon.  If the user is logged on to Amazon when they try to click the
legitimate button, the inadvertent click on the invisible button will
make the unintended purchase without the user's knowledge or consent.

\paragraph{Lab environment} The activities in this document have been
tested on our pre-built Ubuntu 20.04 VM, which can be downloaded from
the SEED website.  

%\paragraph{Acknowledgment} 
%\begin{quote}
%This lab was developed while the authors were students at Washington
%University in St.\ Louis.
%\end{quote}


% *******************************************
% Environment Setup
% *******************************************j
\section{Environment Setup}

%%%%% Installations %%%%%%

\subsection{Installations}
The following software is required for this lab:

\begin{enumerate}
    \item Python 3 (already installed in SEED Ubuntu 20.04 VM)
    \begin{itemize}
        \item You can use the command \texttt{python3 --version} to check
        whether it is installed: if so, the version number will be 
        printed, and if nothing is printed, it is not installed.  
        \item If it is not installed, install it using your preferred 
        method of installation. Here is the downloads page on the 
        official Python website:
        \href{https://www.python.org/downloads/release/python-391/}{https://www.python.org/downloads/release/python-391/}
    \end{itemize}

    \item Pip3
    \begin{itemize}
      \item You can check whether pip3 (a package manager for Python) is
      installed  with the command `pip3 --version' .
      \item If not, install it with the command `sudo apt install pip3'.
    \end{itemize}

    \item Flask 
    \begin{itemize}
      \item Flask is a Python package that we will use to run a simple
      web server.  Run the following command to install Flask: 
      \texttt{pip install -U Flask}
      \item To call Flask's scripts easily, we must add Flask to the
      \texttt{PATH} environment variable.  Add the following lines at 
      the bottom of the file {\tt /home/seed/.bashrc}:

      \begin{verbatim}
      PATH = PATH:/home/seed/.local/bin
      export PATH
      \end{verbatim}
      Then run the command {\tt source /home/seed/.bashrc } to update 
      the \texttt{PATH} variable in the current session.

      \item For more information on Flask, see 
      \href{https://pypi.org/project/Flask/}{https://pypi.org/project/Flask/}
    \end{itemize}
    
    \item Download and unzip Web\_Clickjacking.zip from the seed lab website.
\end{enumerate}


%%%%% How to Run a Flask Server %%%%%

\subsection{Run the Flask Server} To run our simple server, navigate to
the directory called \textit{benign\_server} in the provided materials.
The file \texttt{server.py} in that directory contains code to run a
simple server using Flask. To start the server, execute the following
commands from inside the \textit{benign\_server} directory:
\begin{itemize} 
  \item[\$] \texttt{export FLASK\_APP=server.py} 
  \item[\$] \texttt{flask run} 
\end{itemize}

\paragraph{Important note} Any time you make an update to the server,
you will need to restart it for the update to take effect. To restart,
first stop the server using the key combination \texttt{Ctrl+C}, then
type \texttt{flask run} start it again. If you exit out of your VM, you
may need to run the export command again (\texttt{\$ export
FLASKAPP=server.py}) before running the server.

\paragraph{Test} Navigate to the page:
\href{http://localhost:5000/benign}{http://localhost:5000/benign}.
You should see a page for Alice's Cupcakes, a fictional local bakery. If
the whole site does not fit in your window, use \texttt{Ctrl + (minus
key)} and \texttt{Ctrl + (plus key)} to zoom out and in respectively
until the site fits.  Note that the header and footer buttons on the
site are just placeholders and do not contain live links.

\subsection{Open Malicious HTML File}

To prepare to explore the clickjacking attack, open the file
\texttt{malicioius/malicious.html} in a separate tab. For example, in
Firefox, use this format in the address bar:
\\\texttt{file:///path/to/dir/malicious/malicious.html}  .  The site
currently displays a single button which, when clicked, takes the user
to a simulated malicious webpage, but in reality could perform a variety
of malicious actions.  

The goal of the attacker is to overlay this button onto a view of the
benign webpage on the malicious site, so that a victim user will
inadvertently click the malicious button when they think they are
clicking a button on the benign webpage.


% *******************************************
% Lab Tasks
% *******************************************
\section{Lab Tasks} 

%%%%%%%%%% Task 1 %%%%%%%%%% 

\subsection{Task 1: Copy that site!}

In the \textit{server} folder, you will find the files comprising the
(benign) website for Alice's Cupcakes: \texttt{benign.html} and
\texttt{server.py}. This site will be served locally on your VM using
Flask. In the \textit{malicious} folder, you will find the materials
needed for the attack, which uses a malicious website defined by the
files \texttt{malicious.html} and \texttt{malicious.css}. This site will
also be served locally. You will be making changes to all of these files
throughout the lab. 

Our first step as the attacker is to add code to \texttt{malicious.html}
so that it mimics the benign website defined in \texttt{benign.html} as
closely as possible.

\noindent A popular way to do this is with an HTML Inline Frame element
(``iframe''). An iframe enables embedding one HTML page within another.
The \texttt{src} attribute of the iframe specifies the site to be
embedded, and when the iframe code is executed on a page, the embedded
site is loaded into the iframe. 

\paragraph{Embed the benign site into the malicious site}
\begin{itemize}
    \item Start the benign server \texttt{benign.html} running
    locally using the Flask server (see Environment Setup section 
    above).
    \item Add an iframe HTML element in \texttt{malicious.html} that 
    pulls from \texttt{http://localhost:5000/benign}. 
    \item Modify the CSS in \texttt{malicious.css} using the height,
    width, and position attributes to make the iframe cover the whole 
    page and make the button overlay the iframe.
    
    \item Hints:
    \begin{itemize}
       \item Explicitly set the iframe to have no border.
       \item Investigate the `absolute' and `relative' settings of the 
       position attribute to determine which should be used.
    \end{itemize} 

    \item Test your changes by opening malicious.html (see Environment
    Setup section above).  

\end{itemize} 

% \textbf{Solution for Instructors:}
% <!-- iframe in malicious.html -->
% <iframe src="http://localhost:5000/benign" frameborder="0" ></iframe>

% /* iframe css in malicious.css */
% iframe {
%     height:100%;
%     width:100%;
%     position:absolute;
% }

\paragraph{Questions:}
\begin{enumerate}
    \item With the server running, what does \texttt{malicious.html} 
    look like?
        % Solution for Instructors:
        % It looks exactly like benign.html, except that the
        % button from malicious.html has been overlaid on the page. 
        % Clicking on the ``Explore Menu" button takes you to the 
        % benign.html page as it normally functions.
\end{enumerate}

%%%%%%%%%% Task 2 %%%%%%%%%% 

\subsection{Task 2: Let’s Get Clickjacking!}

\paragraph{basic clickjacking attack}
Add to the given CSS code for a ``button'' object in
\texttt{malicious.css} to make the malicious button in
\texttt{malicious.html} invisible. Position the button such that it
covers the ``Explore Menu'' button within the iframe added in the
previous Task. There are a variety of strategies for doing this; we
recommend looking into these CSS attributes \texttt{margin-left},
\texttt{margin-top}, \texttt{color}, and \texttt{background-color}. When
this task is complete, you will have a functioning clickjacking attack. 

% \textbf{Solution for Instructors:}
% /* malicious button css in malicious.css */
% button{
%     /* Given button code for size and shape. You do not need to edit this. */
%     position: absolute;
%     border: none;
%     color: white;
%     padding: 35px 35px;
%     text-align: center;
%     font-size: 40px;
%     border-radius: 15px;
%     /* end of given button code */


%     /* TODO: edit/add attributes below for the malicious button */
%     /* You will want to change the button's position on the page and make the button transparent */

%     /* solution */
%     margin-left: 50px;
%     margin-top: 350px;
%     color: hsla(0, 0%, 0%, 0);
%     background-color: hsla(0, 0%, 0%, 0);
%   }

%   /* alternate solution */
%     margin-left: 50px;
%     margin-top: 350px;
%     color: white; opacity: 0%;
%     background-color: blue; opacity: 0%;
%   }


\paragraph{Questions:}
\begin{enumerate}
    \item How does the malicious site’s appearance compare to that of
the benign site?
        % Solution for Instructors:
        % The malicious site looks identical to the benign site. This is
        % because the malicious site is displaying the benign site in a
        % frame covering its whole page.
    \item What happens when you try to click on the ``Explore Menu'' button?
        % Solution for Instructors:
        % It takes the victim user to you\_have\_been\_hacked.html. The
        % user should be misled to believe they are clicking on an 
        % innocent ``Explore Menu" button, when in reality they are 
        % actually clicking on the invisible malicious button.
    \item Describe a scenario where clickjacking of the type just
    demonstrated could occur and lead to undesirable consequences for 
    the user.
        % Solution for Instructors:
        % There are many plausible scenarios. A clickjacking could occur
        % on any website, such as that of a restaurant (as in the 
        % example used in this lab), online store, or local business. 
        % Instead of the malicious link taking the user to a harmless 
        % page with some placeholder text, it could instead download a 
        % real virus to the user's computer. Alternatively, if it were 
        % done against an online store where purchases can be made, the
        % attacker could insert and overlay multiple invisible page 
        % elements such as text fields and a button. Then, when a user
        % enters credit card information to pay, they inadvertently 
        % hand over all of this information to the attacker via the 
        % overlaid fields. These are just a few example cases.
\end{enumerate}


%%%%%%%%%% Task 3 %%%%%%%%%% 

\subsection{Task 3: Bust That Frame!}
``Frame busting'' is the practice of writing scripts to prevent a web
page from being displayed within a frame.  One way to do this is by
including script code in a website that prevents the site from being
framed -- that is, prevents another site from opening it in an iframe.
The technique we will implement today checks that the site to be
protected is the topmost site on any page where it is being displayed,
thus preventing buttons on an attacker's page from being overlaid on top
of it. 

\paragraph{Write the frame busting script} 

Open the file \texttt{benign\_server/templates/benign.html}, which
contains code for the Alice's Cupcakes homepage, which we would like to
protect from clickjacking. Here, you will fill in the Javascript method
called $makeThisFrameOnTop()$. Your code should compare
\texttt{window.top} and \texttt{window.self} to find out if the top
window and the current site's window are the same object (as they should
be). If not, use the
\href{https://developer.mozilla.org/en-US/docs/Web/API/Location}{Location
Object} to set the location of the top window to be the same as the
location of the current site's window. This should be a simple method
and take no more than a few lines of code. Test it out and confirm that
your script successfully stops the clickjacker. 

\paragraph{Reminder} Remember that any time you make changes to a file
on the server, you must restart the server (see ``Important note'' in
the Setup section). 


% \textbf{Solution for Instructors:} 

% // One possible solution: 
% function makeThisFrameOnTop() {
%   if(window.top != window.self) {
%       window.top.location = window.self.location
%   }
% }

% // Another possible solution: 
% function makeThisFrameOnTop() {
%   if(top != self) {
%       top.location = self.location
%   }
% }

\paragraph{Questions:}
\begin{enumerate}
    \item What happens when you try navigating to the malicious site?
        % Solution for Instructors:
        % If the frame-busting script is working, when you navigate to 
	% the malicious site, it should now automatically redirect you to
	% benign.html. This is because in benign.html, you checked to see
	% if benign.html was embedded in some site (not the topmost
	% frame), and if so, you broke out of that embedded frame. 
    \item What happens when you try clicking the button?
        % Solution for Instructors:
        % The browser navigates to benign.html (i.e.\ stays on the same
        % page) since we are only interacting with
        % benign.html.  The malicious button overlay does not
        % exist anymore.
\end{enumerate}



%%%%%%%%%% Task 4 %%%%%%%%%% 

\subsection{Task 4: Attacker Countermeasure (Bust the Buster)}

\paragraph{Disable the frame-busting script}
Now let's explore how simple it is for the attacker to create a
workaround for front-end clickjacking defenses like frame busting. There
are multiple ways of accomplishing this, such as with a double frame
attack, but one of the simplest in the current scenario is to add the
``sandbox'' attribute to the malicious iframe. Read more about the
sandbox attribute on
\underline{\href{https://developer.mozilla.org/en-US/docs/Web/HTML/Element/iframe}{this
page}} about iframes, then add the ``sandbox'' attribute to the iframe
in \texttt{malicious.html} and answer the following questions.  

% \textbf{Solution for Instructors:} 
% <!-- Updated iframe with sandbox attribute in malicious.html -->
% \begin{algorithmic}
%     \State <iframe sandbox src="http://localhost:5000/benign" frameborder="0"></iframe>
% \end{algorithmic}


\paragraph{Questions:}
\begin{enumerate}
    \item What does the sandbox attribute do? Why does this prevent the frame buster from working?
        % Solution for Instructors:
        % The sandbox attribute, by default, applies a number of
        % restrictions to the frame, including preventing the embedded
        % site from running scripts such as the one employed by the
        % frame buster we tested in the previous Task. Setting different
        % properties of the sandbox attribute allow finer-grained
        % control of iframe capabilities.
    \item What happens when you navigate to the malicious site after
updating the iframe to use the `sandbox' attribute?
	% Solution for Instructors:
	% When the students try to navigate to the malicious site now,
	% it will stay on the malicious page instead of immediately
	% navigating away (to the Alice's Cupcakes page) as the
	% frame-busting script previously made it do. 
    \item What happens when you try clicking the button?
        % Solution for Instructors:
        % Clicking the button should again take you to
	% you\_have\_been\_hacked.html ; i.e., the original clickjacking
	% attack is fully functional again.
\end{enumerate}


%%%%%%%%%% Task 5 %%%%%%%%%% 

\subsection{Task 5: The Ultimate Bust}
As we saw in Task 4, front-end defenses are not sufficient to prevent
clickjacking, because front-end defenses such as frame-busting can be
directly circumvented by other front-end settings on the attacker's
webpage. To prevent clickjacking attacks more comprehensively, we need
to set up back-end (i.e.\ server-side) defenses.

Server-side defenses can prevent front-end attacks such as the ones
presented in this lab so far, which rely on a malicious website being
able to fetch the benign website's content before the benign site has a
chance to execute any front-end defenses. 

Specifically, by providing a special HTTP header in response to a
request for the content of a legitimate website, the site can instruct
browsers to only display the content according to specific matching
rules designed to exclude clickjacking attack scenarios.

One header used for this purpose is called ``X-Frame-Options,'' and its
value specifies the conditions under which the requested content will be
displayed.  This header is not part of the official HTTP specification
but is processed by many common browsers.

Another more standardized header that can provide protection against
clickjacking is the ``Content-Security-Policy'' header, which can
specify a diverse set of directives as part of a site's Content Security
Policy (CSP) intended to stop many types of attacks.  The CSP header
directive relevant for stopping clickjacking is ``frame-ancestors'' ,
which specifies valid parents that may embed a page in a frame.

You can read more about the XFO attribute
\underline{\href{https://developer.mozilla.org/en-US/docs/Web/HTTP/Headers/X-Frame-Options}{here}},
and more about CSPs
\underline{\href{https://developer.mozilla.org/en-US/docs/Web/HTTP/CSP}{here}}.

\paragraph{Modify the response header} Navigate to \texttt{server.py}.
Add code to $benign()$ to modify the HTTP response header served with
the page in order to prevent the clickjacking attack.  Specifically, set
the X-Frame-Options (XFO) attribute to the value \texttt{"DENY"} and the
Content-Security-Policy (CSP) attribute to the value
\texttt{"frame-ancestors `none' "}.

\paragraph{Hint} Flask (along with many other common web-development
frameworks) provides a simple interface to modify a response's headers.
If \texttt{resp} is a response object to be returned by a function, you
can modify its header with the following command:
\\\texttt{resp.headers['attribute'] = value} , where \texttt{attribute}
is the header attribute you would like to modify and \texttt{value} is
the value to which it should be set. 

\paragraph{Reminder} Remember that any time you make an update to the
server, you will need to restart it (see ``Important note" in Setup).


% \textbf{Solution for Instructors:} 
% resp.headers['X-Frame-Options'] = ``DENY''
% resp.headers['Content-Security-Policy'] = ``frame-ancestors 'none' ''


\paragraph{Questions:}
\begin{enumerate}
    \item What is the X-Frame-Options HTTP header attribute, and why is
    it set to ``DENY'' to prevent the attack?
        % Solution for Instructors:
        % The X-Frame-Options HTTP response header can be used to
        % indicate whether or not a browser should be allowed to render
        % a page in a <frame> , <iframe> , <embed> or <object>. Setting
        % its value to "DENY" prevents the page from being rendered by
        % any other page when fetched in a frame.  See more here: 
        % https://developer.mozilla.org/en-US/docs/Web/HTTP/Headers/X-Frame-Options
    \item What is the Content-Security-Policy header attribute, and why
    is it set to ``frame-ancestors `none' '' to prevent the attack?
	% A Content Security Policy (CSP) is an added layer of security
	% for a page that helps to defend against certain types of
	% attacks, including Cross-Site Scripting (XSS) and
	% data-injection attacks. These attacks are used for everything
	% from data theft to site defacement to distribution of malware.
	% Setting ``frame-ancestors `none' '' specifies that no valid
	% parents of a page (including the origin of a <frame> trying to
	% fetch the page) may embed the page. See more on
        % CSPs here: https://developer.mozilla.org/en-US/docs/Web/HTTP/CSP
        % Note that this setting is redundant with the XFO
        % setting, but is the preferred option as CSPs are more widely
        % supportd than the XFO option.
    \item What happens when you navigate to the malicious site
    after modifying the response headers? What do you see?
        % Solution for Instructors:
        % The malicious site should not even display the benign
	% site's content. This is because the response header from the
	% server directs the browser that it cannot be displayed within
	% a frame. The attacker's page thus does not have the
	% opportunity to work around the content file's code (including
	% any attempts at frame busting) because the code is blocked by
	% the browser after parsing the HTTP header and thus before
	% displaying the page's content. 
	%
	% **NOTE: the browser may allow the user to view the embedded
	% frame in another tab or window, with a warning that explains
	% the XFO/CSP options and the risk of clickjacking attack.

\end{enumerate}

\paragraph{Learn more}
There are a number of other ways to perform clickjacking besides the one
explored in this lab, and many possible malicious consequences beyond
the ones suggested here. To learn more about clickjacking, visit
\href{https://owasp.org/www-community/attacks/Clickjacking}{https://owasp.org/www-community/attacks/Clickjacking}.

% *******************************************
% Submission
% ******************************************* 
\section{Submission}

\begin{quote}
\seedsubmission
\end{quote}


\end{document}
