%%%%%%%%%%%%%%%%%%%%%%%%%%%%%%%%%%%%%%%%%%%%%%%%%%%%%%%%%%%%%%%%%%%%%%
%%  Copyright by Collin Howland, Faith Letzkus, and Julio Trujillo  %%
%%  at Washington University in St. Louis. 			    %%
%%  This work is licensed under the Creative Commons                %%
%%  Attribution-NonCommercial-ShareAlike 5.0 International License. %%
%%  To view a copy of this license, visit                           %%
%%  http://creativecommons.org/licenses/by-nc-sa/4.0/.              %%
%%%%%%%%%%%%%%%%%%%%%%%%%%%%%%%%%%%%%%%%%%%%%%%%%%%%%%%%%%%%%%%%%%%%%%

\newcommand{\commonfolder}{../../common-files}
\newcommand{\webcommon}{../Web_Common}

\documentclass[11pt]{article}

\usepackage[most]{tcolorbox}
\usepackage{times}
\usepackage{epsf}
\usepackage{epsfig}
\usepackage{amsmath, alltt, amssymb, xspace}
\usepackage{wrapfig}
\usepackage{fancyhdr}
\usepackage{url}
\usepackage{verbatim}
\usepackage{fancyvrb}
\usepackage{adjustbox}
\usepackage{listings}
\usepackage{color}
\usepackage{subfigure}
\usepackage{cite}
\usepackage{sidecap}
\usepackage{pifont}
\usepackage{mdframed}
\usepackage{textcomp}
\usepackage{enumitem}


% Horizontal alignment
\topmargin      -0.50in  % distance to headers
\oddsidemargin  0.0in
\evensidemargin 0.0in
\textwidth      6.5in
\textheight     8.9in 

\newcommand{\todo}[1]{
\vspace{0.1in}
\fbox{\parbox{6in}{TODO: #1}}
\vspace{0.1in}
}


\newcommand{\unix}{{\tt Unix}\xspace}
\newcommand{\linux}{{\tt Linux}\xspace}
\newcommand{\minix}{{\tt Minix}\xspace}
\newcommand{\ubuntu}{{\tt Ubuntu}\xspace}
\newcommand{\setuid}{{\tt Set-UID}\xspace}
\newcommand{\openssl} {\texttt{openssl}}


\pagestyle{fancy}
\lhead{\bfseries SEED Labs}
\chead{}
\rhead{\small \thepage}
\lfoot{}
\cfoot{}
\rfoot{}


\definecolor{dkgreen}{rgb}{0,0.6,0}
\definecolor{gray}{rgb}{0.5,0.5,0.5}
\definecolor{mauve}{rgb}{0.58,0,0.82}
\definecolor{lightgray}{gray}{0.90}


\lstset{%
  frame=none,
  language=,
  backgroundcolor=\color{lightgray},
  aboveskip=3mm,
  belowskip=3mm,
  showstringspaces=false,
%  columns=flexible,
  basicstyle={\small\ttfamily},
  numbers=none,
  numberstyle=\tiny\color{gray},
  keywordstyle=\color{blue},
  commentstyle=\color{dkgreen},
  stringstyle=\color{mauve},
  breaklines=true,
  breakatwhitespace=true,
  tabsize=3,
  columns=fullflexible,
  keepspaces=true,
  escapeinside={(*@}{@*)}
}

\newcommand{\newnote}[1]{
\vspace{0.1in}
\noindent
\fbox{\parbox{1.0\textwidth}{\textbf{Note:} #1}}
%\vspace{0.1in}
}


%% Submission
\newcommand{\seedsubmission}{You need to submit a detailed lab report, with screenshots,
to describe what you have done and what you have observed.
You also need to provide explanation
to the observations that are interesting or surprising.
Please also list the important code snippets followed by
explanation. Simply attaching code without any explanation will not
receive credits.}

%% Book
\newcommand{\seedbook}{\textit{Computer \& Internet Security: A Hands-on Approach}, 2nd
Edition, by Wenliang Du. See details at \url{https://www.handsonsecurity.net}.}

%% Videos
\newcommand{\seedisvideo}{\textit{Internet Security: A Hands-on Approach},
by Wenliang Du. See details at \url{https://www.handsonsecurity.net/video.html}.}

\newcommand{\seedcsvideo}{\textit{Computer Security: A Hands-on Approach},
by Wenliang Du. See details at \url{https://www.handsonsecurity.net/video.html}.}

%% Lab Environment
\newcommand{\seedenvironment}{This lab has been tested on our pre-built
Ubuntu 16.04 VM, which can be downloaded from the SEED website. }

\newcommand{\seedenvironmentA}{This lab has been tested on our pre-built
Ubuntu 16.04 VM, which can be downloaded from the SEED website. }

\newcommand{\seedenvironmentB}{This lab has been tested on our pre-built
Ubuntu 20.04 VM, which can be downloaded from the SEED website. }

\newcommand{\seedenvironmentAB}{This lab has been tested on our pre-built
Ubuntu 16.04 and 20.04 VMs, which can be downloaded from the SEED website. }

\newcommand{\nodependency}{Since we use containers to set up the lab environment, 
this lab does not depend too much on our SEED VM. You can do this lab
using other VMs or physical machines. }






\lhead{\bfseries SEED Labs -- Clickjacking Lab}

\begin{document}

\begin{center}
{\LARGE Clickjacking}
\end{center}

\vspace{0.1in}
\fbox{\parbox{6in}{\small Copyright \copyright\ 2021 \ \ by Collin
      Howland, Faith Letzkus, Julio Trujillo, and Steve Cole at  
      Washington University in St.\ Louis.  \\
      This work is licensed under a Creative Commons
      Attribution-NonCommercial-ShareAlike 4.0 International License. 
      If you remix, transform, or build upon the material, 
      this copyright notice must be left intact, or reproduced in a way
      that is reasonable to the medium in which the work is being 
      re-published.}}
\vspace{0.1in}

% *******************************************
% Overview
% ******************************************* 
\section{Overview}

Clickjacking, also known as a ``UI redress attack,'' is an attack that
tricks a user into clicking on something they do not intend to when
visiting a webpage, thus ``hijacking'' the click. In this lab, we will
explore a common attack vector for clickjacking: the attacker creates a
webpage that loads the content of a legitimate page but overlays one or
more of its buttons with invisible button(s) that trigger malicious
actions.  When a user attempts to click on the legitimate page's
buttons, the browser registers a click on the invisible button instead,
triggering the malicious action.

\paragraph{Example scenario.} Suppose an attacker acquires the domain
\texttt{starbux.com} and creates a website with that URL. The site first
loads the legitimate target website \texttt{starbucks.com} in an iframe
element spanning the entire webpage, so that the
malicious \texttt{starbux.com} website looks identical to the legitimate
\texttt{starbucks.com} website.  The attacker's site then places an invisible
button on top of the `Menu' button on the displayed \texttt{starbucks} page;
the button triggers a 1-click purchase of the attacker's product on
Amazon.  If the user is logged on to Amazon when they try to click the
legitimate button, the inadvertent click on the invisible button will
make the unintended purchase without the user's knowledge or consent.

\paragraph{Topic coverage.} This lab covers the following topics:
\begin{itemize}[noitemsep]
 \item Clickjacking attack
 \item Countermeasures: frame busting and HTTP headers
 \item Iframes and sandboxes
 \item JavaScript
\end{itemize}

\paragraph{Lab environment.}
\seedenvironmentB
\nodependency



% *******************************************
% Lab Environment Setup
% *******************************************j
\section{Lab Environment Setup}

In this lab, we will use two websites.  The first is the vulnerable
homepage of the fictional business ``Alice's Cupcakes'', accessible at
\url{http://www.cjlab.com}.  The second is the attacker's malicious web site
that is used for hijacking clicks intended for the Alice's Cupcakes
page, accessible at \url{http://www.cjlab-attacker.com}. We
use containers to set up the web servers.
 
% ------------------------------------------- SUBSECTION
% -------------------------------------------
\subsection{Container Setup and Commands}

%%%%%%%%%%%%%%%%%%%%%%%%%%%%%%%%%%%%%%%%%%%%
Please download the
\texttt{Labsetup.zip} file to your VM from the lab's website,
unzip it, enter the \texttt{Labsetup} folder, and 
use the \texttt{docker-compose.yml} file to 
set up the lab environment. Detailed explanation
of the content in this file and all the involved 
\texttt{Dockerfile} can be found from the 
user manual, which is linked to the website of this lab.
If this is the first time you set up a SEED lab environment
using containers, it is very important that you read 
the user manual. 

In the following, we list some of the commonly
used commands related to Docker and Compose. 
Since we are going to use 
these commands very frequently, we have created aliases for them
in the \texttt{.bashrc} file (in our provided SEEDUbuntu 20.04 VM).


\begin{lstlisting}
$ docker-compose build  # Build the container image
$ docker-compose up     # Start the container
$ docker-compose down   # Shut down the container

// Aliases for the Compose commands above
$ dcbuild       # Alias for: docker-compose build
$ dcup          # Alias for: docker-compose up
$ dcdown        # Alias for: docker-compose down
\end{lstlisting}


All the containers will be running in the background. To run
commands on a container, we often need to get a shell on
that container. We first need to use the \texttt{"docker ps"}  
command to find out the ID of the container, and then
use \texttt{"docker exec"} to start a shell on that 
container. We have created aliases for them in
the \texttt{.bashrc} file.

\begin{lstlisting}
$ dockps        # Alias for: docker ps --format "{{.ID}}  {{.Names}}" 
$ docksh <id>   # Alias for: docker exec -it <id> /bin/bash

# The following example shows how to get a shell inside hostC
$ dockps
b1004832e275  hostA-10.9.0.5
0af4ea7a3e2e  hostB-10.9.0.6
9652715c8e0a  hostC-10.9.0.7

$ docksh 96
root@9652715c8e0a:/#  

# Note: If a docker command requires a container ID, you do not need to 
#       type the entire ID string. Typing the first few characters will 
#       be sufficient, as long as they are unique among all the containers. 
\end{lstlisting}


If you encounter problems when setting up the lab environment, 
please read the ``Common Problems'' section of the manual
for potential solutions.


%%%%%%%%%%%%%%%%%%%%%%%%%%%%%%%%%%%%%%%%%%%%


% -------------------------------------------
% SUBSECTION
% -------------------------------------------
\subsection{Websites}

We use two containers, one running the website for Alice's Cupcakes (the
``defender'', with IP address \texttt{10.9.0.5}) , and the other running
the website for the attacker (with IP address \texttt{10.9.0.105}).  The
IP addresses for these two containers must be consistent in the
\texttt{docker-compose.yml} file and the \texttt{/etc/hosts} file (see
below), and we recommend not changing them from their default values.

\paragraph{The Defender container.}
The website for Alice's Cupcakes is hosted on an Apache web server.  The
website setup is included in \texttt{apache\_defender.conf} inside the
defender image folder.  The configuration specifies the URL for the
website and the folder where the web application code is stored.  The
configuration also contains placeholders for HTTP response headers
returned by the server (commented out), which will be filled in during
the course of the lab as a defense countermeasure.

\begin{lstlisting}
<VirtualHost *:80>
     DocumentRoot /var/www/defender
     ServerName www.cjlab.com
#    Header set <Header-name> "<value>";
#    Header set Content-Security-Policy " \
#             <directive> '<value>'; \
#           "
</VirtualHost>
\end{lstlisting}

Because we need to modify the defender's web page inside this container,
for convenience as well as to allow the modified files to persist beyond
the containers, we have mounted a folder (\texttt{Labsetup/defender} on
the hosting VM) to the container's \texttt{/var/www/defender} folder,
which is the \texttt{DocumentRoot} folder in our Apache configuration.
Therefore any files we modify inside the \texttt{defender} folder on the
VM will be seen as modified by the defender's web server on the
container. 

\paragraph{The Attacker container.}
The attacker's website is also hosted on an Apache web server.  The
website setup is included in \texttt{apache\_attacker.conf} inside the
attacker image folder.  The Apache configuration for this website is
as follows:

\begin{lstlisting}
<VirtualHost *:80>
    ServerName www.cjlab-attacker.com
    DocumentRoot "/var/www/attacker"
    DirectoryIndex attacker.html
</VirtualHost>
\end{lstlisting}
 
Because we need to modify the attacker's web page inside this container,
for convenience as well as to allow the modified files to persist beyond
the containers, we have mounted a folder (\texttt{Labsetup/attacker} on
the hosting VM) to the container's \texttt{/var/www/attacker} folder,
which is the \texttt{DocumentRoot} folder in our Apache configuration.
Therefore any files we modify inside the \texttt{attacker} folder on the
VM will be seen as modified by the attacker's web server on the
container. 

\paragraph{Important note.} Any time you make updates to the websites,
you may need to clear the browser's cache and/or rebuild and restart the
containers for the change to be visible, depending on the scope of the
change.

\paragraph{DNS configuration.}
We access the defender website and the attacker website using their
respective URLs.  To enable these hostnames to be mapped to their
corresponding IP addresses, we need to add the following entries to the
\texttt{/etc/hosts} file on the VM.  You need to use the root privilege
to change this file (using \texttt{sudo}).

\begin{lstlisting}
10.9.0.5        www.cjlab.com
10.9.0.105      www.cjlab-attacker.com
\end{lstlisting}


\paragraph{Test: defender.} Build and start the containers, then
navigate to the page \url{http://www.cjlab.com} on the VM.  You should
see a page for Alice's Cupcakes, a fictional local bakery. If the whole
site does not fit in your window, use \texttt{Ctrl + (minus key)} and
\texttt{Ctrl + (plus key)} to zoom out and in respectively until the
site fits.  Note that the header and footer buttons on the site are just
placeholders and do not contain live links.

\paragraph{Test: attacker.}
Next, navigate to the page \url{http://www.cjlab-attacker.com}.  You
should see a single button which, when clicked, takes you to a page that
tells you you've been hacked. In a real attack, this button could 
perform a variety of malicious actions.  

The goal of the attacker is to overlay this button onto a view of the
defender's webpage displayed on the attacker's site, so that a victim
user will inadvertently click the malicious button when they think they
are clicking a button on the defender's webpage.


% *******************************************
% Lab Tasks
% *******************************************
\section{Lab Tasks} 

%%%%%%%%%% Task 1 %%%%%%%%%% 

\subsection{Task 1: Copy that site!}

In the \texttt{defender} folder, you will find the files comprising the
website for Alice's Cupcakes: \texttt{index.html} and
\texttt{defender.css}.  In the \texttt{attacker} folder, you will find
the files comprising the attacker's website: \texttt{attacker.html} and
\texttt{attacker.css}.  You will be making changes to all of these files
except \texttt{defender.css} throughout the lab. 

Our first step as the attacker is to add code to \texttt{attacker.html}
so that it mimics the Alice's Cupcakes website as closely as possible.
A common way to do this is with an HTML Inline Frame element
(``iframe''). An iframe enables embedding one HTML page within another.
The \texttt{src} attribute of the iframe specifies the site to be
embedded, and when the iframe code is executed on a page, the embedded
site is loaded into the iframe. 

\paragraph{Embed the defender's site into the attacker's site.}
\begin{itemize}
    \item Add an iframe HTML element in \texttt{attacker.html} that 
    pulls from \texttt{http://www.cjlab.com}. 
    \item Modify the CSS in \texttt{attacker.css} using the 
    \texttt{height}, \texttt{width}, and \texttt{position} attributes 
    to make the iframe cover the whole page and the button overlay the
    iframe.
    
    \item Hints:
    \begin{itemize}
       \item Explicitly set the iframe to have no border.
       \item Investigate the `absolute' and `relative' settings of the 
       \texttt{position} attribute to determine which should be used.
    \end{itemize} 

    \item Test your changes by navigating to the attacker's website.
          (Remember that you may need to clear the browser's cache and
          reload the page to see changes made after the initial load.)

\end{itemize} 

\paragraph{Question:}
\begin{enumerate}
    \item With the iframe inserted, what does the attacker's website
    look like?
\end{enumerate}

%%%%%%%%%% Task 2 %%%%%%%%%% 

\subsection{Task 2: Let's Get Clickjacking!}

\paragraph{Basic clickjacking attack.}
Add code to the CSS specification of a ``button'' object given in
\texttt{attacker.css} to make the malicious button in
\texttt{attacker.html} invisible. Position the button so that it
covers the ``Explore Menu'' button within the iframe added in the
previous Task. There are a variety of ways to accomplish this Task; we
recommend using the CSS attributes \texttt{margin-left}, 
\texttt{margin-top}, \texttt{color}, and \texttt{background-color}. When
this Task is complete, you will have a functioning clickjacking attack. 

\paragraph{Questions:}
\begin{enumerate}
\setcounter{enumi}{1}
    \item How does the appearance of the attacker's site compare to that of
	  the defender's site?

    \item What happens when you click on the ``Explore Menu'' button on
    the attacker's site?

    \item Describe an attack scenario in which the style of 
    clickjacking implemented for this Task leads to undesirable 
    consequences for a victim user.
\end{enumerate}


%%%%%%%%%% Task 3 %%%%%%%%%% 

\subsection{Task 3: Bust That Frame!}
``Frame busting'' is the practice of preventing a web page from being
displayed within a frame, thus defending against the type of attack
implemented in the previous Task.  One way to bust frames is to include
script code in the webpage source that prevents the site from being
framed -- that is, it prevents other sites from opening the webpage in
an iframe.  In this Task we will add script code to the defender's
webpage that ensures it is the topmost window on any page where it is
being displayed, thus preventing buttons on an attacker's page from
being overlaid on top of it. 

\paragraph{Write the frame-busting script.} 

Open the file \texttt{defender/index.html}, which contains code for the
Alice's Cupcakes homepage. We would like to protect the homepage from
clickjacking. Your task is to fill in the Javascript method called
$makeThisFrameOnTop()$. Your code should compare \texttt{window.top} and
\texttt{window.self} to find out if the top window and the current
site's window are the same object (as they should be). If not, use the
\href{https://developer.mozilla.org/en-US/docs/Web/API/Location}{Location
Object} to set the location of the top window to be the same as the
location of the current site's window. This should be a simple method
and take no more than a few lines of code. Test it out and confirm that
your script successfully stops the clickjacker. 

\paragraph{Reminder.} Remember that any time you make changes to one of
the websites, you may need to clear the browser's cache and reload the
page for the changes to take effect.

\paragraph{Questions:}
\begin{enumerate}
\setcounter{enumi}{4}
    \item What happens when you navigate to the attacker's site
          now?

    \item What happens when you click the button?
\end{enumerate}


%%%%%%%%%% Task 4 %%%%%%%%%% 

\subsection{Task 4: Attacker Countermeasure (Bust the Buster)}

\paragraph{Disable the frame-busting script.}
Now let's explore how an attacker can create a
workaround for front-end clickjacking defenses like frame busting. There
are multiple workarounds, 
%such as with a double frame attack, 
but one of the simplest in the current scenario is to add the
\texttt{sandbox} attribute to the malicious iframe. Read more about the
\texttt{sandbox} attribute on
\underline{\href{https://developer.mozilla.org/en-US/docs/Web/HTML/Element/iframe}{this
page}} about iframes, then add the \texttt{sandbox} attribute to the iframe
in \texttt{attacker.html} and answer the following questions.  


\paragraph{Questions:}
\begin{enumerate}
\setcounter{enumi}{6}
    \item What does the \texttt{sandbox} attribute do? 
          Why does this prevent the frame buster from working?

    \item What happens when you navigate to the attacker's site after
          updating the iframe to use the \texttt{sandbox} attribute?

    \item What happens when you click the button on the attacker's site?
\end{enumerate}


%%%%%%%%%% Task 5 %%%%%%%%%% 

\subsection{Task 5: The Ultimate Bust}
As we saw in the previous Task, front-end defenses such as frame busting
can be directly circumvented by other front-end settings on the
attacker's webpage and are therefore not sufficient to prevent
clickjacking. To prevent clickjacking attacks more comprehensively, we
need to set up back-end (i.e.\ server-side) defenses.  Modern websites
can cooperate with common browsers to provide such defenses.

The front-end attacks presented in previous Tasks all rely on the
ability of an attacker's webpage code (running in a victim user's
browser) to fetch a benign website's content before the benign webpage
code has a chance to execute any front-end defenses. 

To block this capability, special HTTP headers have
been created that specify to browsers the 
circumstances under which a website's content should or should not be loaded.
By providing one of these HTTP headers with its response to a
request for content, the website can instruct
browsers to only display the content according to specific matching
rules designed to exclude clickjacking attack scenarios.
These headers are not part of the official HTTP specification,
but are processed by many common browsers.

One such header is called ``X-Frame-Options'',
%and its
%value specifies the conditions under which the requested content will be
%displayed.  This header is not part of the official HTTP specification
%but is processed by many common browsers.
and a newer, more popular one is called ``Content-Security-Policy''. 
This header can
include a diverse set of key-value directives to implement a site's Content Security
Policy (CSP) with the intention of stopping many common attacks while
perserving the desired content-sharing behavior of the site.  The CSP header
directive relevant for preventing clickjacking is ``frame-ancestors'' ,
which specifies the valid parents that may embed a page in a frame.

You can read more about the XFO attribute
\underline{\href{https://developer.mozilla.org/en-US/docs/Web/HTTP/Headers/X-Frame-Options}{here}},
and more about CSPs
\underline{\href{https://developer.mozilla.org/en-US/docs/Web/HTTP/CSP}{here}}.

\paragraph{Modify the defender's response headers.} Open the Apache
configuration file for the defender's website
(\texttt{apache\_defender.conf} in the defender's image folder).
Uncomment the lines that specify the HTTP response headers served with
the page, and substitute appropriate text in order to prevent the
clickjacking attack.  Specifically, set the X-Frame-Options (XFO) header
to the value \texttt{"DENY"} and the Content-Security-Policy (CSP)
header to contain the directive \texttt{"frame-ancestors `none' "}.

\paragraph{Hint.} Because you are making a change to the server's
configuration, you will need to rebuild and restart the containers for
the change to take effect.


\paragraph{Questions:}
\begin{enumerate}
\setcounter{enumi}{9}
    \item What is the X-Frame-Options HTTP header attribute, and why is
    it set to ``DENY'' to prevent the attack?

    \item What is the Content-Security-Policy header attribute, and why
    is it set to ``frame-ancestors `none' '' to prevent the attack?

    \item What happens when you navigate to the attacker's site
    after modifying each response header (one at a time)? What do you
    see when you click the button?
\end{enumerate}

\paragraph{Learn more.}
There are other ways to perform clickjacking besides the one
explored in this lab, and many possible malicious consequences beyond
the ones suggested here. To learn more about clickjacking, visit the
Open Web Application Security Project (OWASP) page on Clickjacking 
\href{https://owasp.org/www-community/attacks/Clickjacking}{here}.

% *******************************************
% Submission
% ******************************************* 
\section{Submission}

\begin{quote}
\seedsubmission
\end{quote}


\end{document}
