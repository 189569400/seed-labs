%%%%%%%%%%%%%%%%%%%%%%%%%%%%%%%%%%%%%%%%%%%%%%%%%%%%%%%%%%%%%%%%%%%%%%
%%  Copyright by Wenliang Du.                                       %%
%%  This work is licensed under the Creative Commons                %%
%%  Attribution-NonCommercial-ShareAlike 4.0 International License. %%
%%  To view a copy of this license, visit                           %%
%%  http://creativecommons.org/licenses/by-nc-sa/4.0/.              %%
%%%%%%%%%%%%%%%%%%%%%%%%%%%%%%%%%%%%%%%%%%%%%%%%%%%%%%%%%%%%%%%%%%%%%%


% -------------------------------------------
% SUBSECTION
% ------------------------------------------- 
\subsection{Testing the DNS Setup}

From the User container, we will run a series of commands to ensure 
that our lab setup is correct. In your lab report, please document
your testing results. 


\paragraph{Get the IP address of \texttt{ns.attacker32.com}.}
When we run the following \texttt{dig} command, 
the local DNS server will forward the request to the Attacker nameserver 
due to the \texttt{forward} zone entry added to the local DNS server's
configuration file. Therefore, the answer should come from
the \texttt{attacker32.com.zone} file that we set up on the Attacker nameserver.
If this is not what you get, your setup has an issue. Please describe your
observation in your lab report. 

\begin{lstlisting}
$ dig ns.attacker32.com
\end{lstlisting}



\paragraph{Get the IP address of \texttt{www.example.com}.} 
Two nameservers are now hosting the \texttt{example.com} 
domain, one is the domain's official nameserver, and the other 
is the Attacker container. We will query these two nameservers and see what 
response we will get. 
Please run the following two commands (from the User machine), 
and describe your observation. 


\begin{lstlisting}
// Send the query to our local DNS server, which will send the query
// to example.com's official nameserver. 
$ dig www.example.com

// Send the query directly to ns.attacker32.com 
$ dig @ns.attacker32.com www.example.com
\end{lstlisting}
 


Obviously, nobody is going to ask \texttt{ns.attacker32.com} for 
the IP address of \texttt{www.example.com}; they will always ask
the \texttt{example.com} domain's official nameserver for 
answers. The objective of the DNS cache poisoning attack
is to get the victims to ask 
\texttt{ns.attacker32.com} for the IP address of 
\texttt{www.example.com}. Namely, if our attack is successful, 
if we just run the first \texttt{dig} command, the one
without the \texttt{@} option, we should get the 
fake result from the attacker, instead of getting 
the authentic one from the domain's legitimate nameserver.


