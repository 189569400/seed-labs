%%%%%%%%%%%%%%%%%%%%%%%%%%%%%%%%%%%%%%%%%%%%%%%%%%%%%%%%%%%%%%%%%%%%%%
%%  Copyright by Wenliang Du.                                       %%
%%  This work is licensed under the Creative Commons                %%
%%  Attribution-NonCommercial-ShareAlike 4.0 International License. %%
%%  To view a copy of this license, visit                           %%
%%  http://creativecommons.org/licenses/by-nc-sa/4.0/.              %%
%%%%%%%%%%%%%%%%%%%%%%%%%%%%%%%%%%%%%%%%%%%%%%%%%%%%%%%%%%%%%%%%%%%%%%

\documentclass[11pt]{article}

\usepackage[most]{tcolorbox}
\usepackage{times}
\usepackage{epsf}
\usepackage{epsfig}
\usepackage{amsmath, alltt, amssymb, xspace}
\usepackage{wrapfig}
\usepackage{fancyhdr}
\usepackage{url}
\usepackage{verbatim}
\usepackage{fancyvrb}
\usepackage{adjustbox}
\usepackage{listings}
\usepackage{color}
\usepackage{subfigure}
\usepackage{cite}
\usepackage{sidecap}
\usepackage{pifont}
\usepackage{mdframed}
\usepackage{textcomp}
\usepackage{enumitem}


% Horizontal alignment
\topmargin      -0.50in  % distance to headers
\oddsidemargin  0.0in
\evensidemargin 0.0in
\textwidth      6.5in
\textheight     8.9in 

\newcommand{\todo}[1]{
\vspace{0.1in}
\fbox{\parbox{6in}{TODO: #1}}
\vspace{0.1in}
}


\newcommand{\unix}{{\tt Unix}\xspace}
\newcommand{\linux}{{\tt Linux}\xspace}
\newcommand{\minix}{{\tt Minix}\xspace}
\newcommand{\ubuntu}{{\tt Ubuntu}\xspace}
\newcommand{\setuid}{{\tt Set-UID}\xspace}
\newcommand{\openssl} {\texttt{openssl}}


\pagestyle{fancy}
\lhead{\bfseries SEED Labs}
\chead{}
\rhead{\small \thepage}
\lfoot{}
\cfoot{}
\rfoot{}


\definecolor{dkgreen}{rgb}{0,0.6,0}
\definecolor{gray}{rgb}{0.5,0.5,0.5}
\definecolor{mauve}{rgb}{0.58,0,0.82}
\definecolor{lightgray}{gray}{0.90}


\lstset{%
  frame=none,
  language=,
  backgroundcolor=\color{lightgray},
  aboveskip=3mm,
  belowskip=3mm,
  showstringspaces=false,
%  columns=flexible,
  basicstyle={\small\ttfamily},
  numbers=none,
  numberstyle=\tiny\color{gray},
  keywordstyle=\color{blue},
  commentstyle=\color{dkgreen},
  stringstyle=\color{mauve},
  breaklines=true,
  breakatwhitespace=true,
  tabsize=3,
  columns=fullflexible,
  keepspaces=true,
  escapeinside={(*@}{@*)}
}

\newcommand{\newnote}[1]{
\vspace{0.1in}
\noindent
\fbox{\parbox{1.0\textwidth}{\textbf{Note:} #1}}
%\vspace{0.1in}
}


%% Submission
\newcommand{\seedsubmission}{You need to submit a detailed lab report, with screenshots,
to describe what you have done and what you have observed.
You also need to provide explanation
to the observations that are interesting or surprising.
Please also list the important code snippets followed by
explanation. Simply attaching code without any explanation will not
receive credits.}

%% Book
\newcommand{\seedbook}{\textit{Computer \& Internet Security: A Hands-on Approach}, 2nd
Edition, by Wenliang Du. See details at \url{https://www.handsonsecurity.net}.}

%% Videos
\newcommand{\seedisvideo}{\textit{Internet Security: A Hands-on Approach},
by Wenliang Du. See details at \url{https://www.handsonsecurity.net/video.html}.}

\newcommand{\seedcsvideo}{\textit{Computer Security: A Hands-on Approach},
by Wenliang Du. See details at \url{https://www.handsonsecurity.net/video.html}.}

%% Lab Environment
\newcommand{\seedenvironment}{This lab has been tested on our pre-built
Ubuntu 16.04 VM, which can be downloaded from the SEED website. }

\newcommand{\seedenvironmentA}{This lab has been tested on our pre-built
Ubuntu 16.04 VM, which can be downloaded from the SEED website. }

\newcommand{\seedenvironmentB}{This lab has been tested on our pre-built
Ubuntu 20.04 VM, which can be downloaded from the SEED website. }

\newcommand{\seedenvironmentAB}{This lab has been tested on our pre-built
Ubuntu 16.04 and 20.04 VMs, which can be downloaded from the SEED website. }

\newcommand{\nodependency}{Since we use containers to set up the lab environment, 
this lab does not depend too much on our SEED VM. You can do this lab
using other VMs or physical machines. }







\newcommand{\seedlabcopyright}[1]{
\vspace{0.1in}
\fbox{\parbox{6in}{\small Copyright \copyright\ {#1}\ \ by Wenliang Du.\\
      This work is licensed under a Creative Commons
      Attribution-NonCommercial-ShareAlike 4.0 International License.
      If you remix, transform, or build upon the material, 
      this copyright notice must be left intact, or reproduced in a way that is reasonable to
      the medium in which the work is being re-published.}}
\vspace{0.1in}
}





\lhead{\bfseries SEED Labs -- Secret-Key Encryption Lab}

\begin{document}


\begin{center}
{\LARGE Secret-Key Encryption Lab}
\end{center}

\seedlabcopyright{2018}

\newcounter{task}
\setcounter{task}{1}
\newcommand{\tasks} {\bf {\noindent (\arabic{task})} \addtocounter{task}{1} \,}



% *******************************************
% SECTION
% ******************************************* 
\section{Overview}

The learning objective of this lab is for students to get familiar with
the concepts in the secret-key encryption and some common attacks 
on encryption. From this lab, students will gain a first-hand experience
on encryption algorithms, encryption modes, paddings, and initial vector (IV). 
Moreover, students will be able to use tools and write programs to 
encrypt/decrypt messages. 

Many common mistakes have been made by developers in using 
the encryption algorithms and modes. These mistakes
weaken the strength of the encryption, and eventually
lead to vulnerabilities. This lab exposes students to 
some of these mistakes, and ask students to launch attacks
to exploit those vulnerabilities. This lab covers the following topics:

\begin{itemize}[noitemsep]
\item Secret-key encryption
\item Substitution cipher and frequency analysis
\item Encryption modes, IV, and paddings
\item Common mistakes in using encryption algorithms
\item Programming using the crypto library
\end{itemize}


\paragraph{Readings.}
Detailed coverage of the secret-key encryption can be found in the following:

\begin{itemize}
\item Chapter 21 of the SEED Book, \seedbook
\end{itemize}


\paragraph{Lab Environment.} \seedenvironment




% *******************************************
% SECTION
% ******************************************* 
\section{Task 1: Frequency Analysis}

It is well-known that monoalphabetic substitution cipher (also known as monoalphabetic cipher) 
is not secure, because it can be subjected to frequency analysis. In this lab, you are given 
a cipher-text that is encrypted using a monoalphabetic cipher; namely,
each letter in the original text is replaced by another letter, 
where the replacement does not vary (i.e., a letter is always replaced by the same letter
during the encryption). Your job is to find out the original text using 
frequency analysis. It is known that the original text is an English article. 


In the following, we describe how we encrypt the original article, and what 
simplification we have made. Instructors can use the same
method to encrypt an article of their choices, instead of asking 
students to use the ciphertext made by us.

\begin{itemize}

\item Step 1: let us do some simplification to the original article. 
We convert all upper cases to lower cases, and then
removed all the punctuations and numbers.  We do keep the spaces between words, so
you can still see the boundaries of the words in the ciphertext. 
In real encryption using monoalphabetic cipher,
spaces will be removed. We keep the spaces to simplify the task. We did this using the 
following command:

\begin{lstlisting}
$ tr [:upper:] [:lower:] < article.txt > lowercase.txt
$ tr -cd '[a-z][\n][:space:]' < lowercase.txt > plaintext.txt
\end{lstlisting}


\item Step 2: let us generate the encryption key, i.e., the substitution table.
We will permute the alphabet from \texttt{a}
to \texttt{z} using Python, and use the permuted alphabet as the key. See the following program.   

\begin{lstlisting}
$ python
>>> import random
>>> s = "abcdefghijklmnopqrstuvwxyz"
>>> list = random.sample(s, len(s))
>>> ''.join(list)
'sxtrwinqbedpvgkfmalhyuojzc'
\end{lstlisting}
 

\item Step 3: we use the \texttt{tr} command to do the encryption. 
We only encrypt letters, while leaving the space and return characters alone. 


\begin{lstlisting}
$ tr 'abcdefghijklmnopqrstuvwxyz' 'sxtrwinqbedpvgkfmalhyuojzc' \
        < plaintext.txt > ciphertext.txt
\end{lstlisting}

\end{itemize}


We have created a ciphertext using a different encryption key (not the one described above).
You can download it from the lab's website. Your job is to use the frequency analysis 
to figure out the encryption key and the original plaintext.  



\paragraph{Guidelines.} Using the frequency analysis, you can find 
out the plaintext for some of the characters quite easily. For those characters, you may want
to change them back to its plaintext, as you may be able to get more clues. 
It is better to use capital letters for plaintext, so for the same letter, we 
know which is plaintext and which is ciphertext. 
You can use the \texttt{tr} command to do this. For example, in the following,
we replace letters \texttt{a}, \texttt{e}, and \texttt{t} in
\texttt{in.txt} with letters \texttt{X}, \texttt{G}, \texttt{E}, respectively;
the results are saved in \texttt{out.txt}. 


\begin{lstlisting}
$ tr 'aet' 'XGE' < in.txt > out.txt
\end{lstlisting}
 

There are many online resources that you can use. We list four useful links in
the following: 


\begin{itemize}
\item \url{http://www.richkni.co.uk/php/crypta/freq.php} : This website can produce the
statistics fro a ciphertext, including the single-letter frequencies, bigram 
frequencies (2-letter sequence), and trigram frequencies (3-letter sequence), etc. 

\item \url{https://en.wikipedia.org/wiki/Frequency_analysis}: This Wikipedia page
provides frequencies for a typical English plaintext. 

\item \url{https://en.wikipedia.org/wiki/Bigram}: Bigram frequency.

\item  \url{https://en.wikipedia.org/wiki/Trigram}: Trigram frequency.
\end{itemize}
 
 


% *******************************************
% SECTION
% ******************************************* 
\section{Task 2: Encryption using Different Ciphers and Modes}

In this task, we will play with various encryption algorithms 
and modes. You can use the following {\tt openssl enc} 
command to encrypt/decrypt a file. To see the manuals, you can 
type {\tt man openssl} and {\tt man enc}.

\begin{lstlisting}
$ openssl enc -ciphertype -e  -in plain.txt -out cipher.bin \
              -K  00112233445566778889aabbccddeeff \
              -iv 0102030405060708 
\end{lstlisting}

Please replace the {\tt ciphertype} with a specific cipher type,
such as {\tt -aes-128-cbc}, {\tt -bf-cbc}, {\tt -aes-128-cfb},
etc. In this task, you should try at least 3 different ciphers. 
You can find the meaning of the 
command-line options and all the supported cipher types 
by typing {\tt "man enc"}.
We include some common options for the {\tt openssl enc} 
command in the following:

\begin{lstlisting}
  -in <file>     input file
  -out <file>    output file
  -e             encrypt
  -d             decrypt
  -K/-iv         key/iv in hex is the next argument
  -[pP]          print the iv/key (then exit if -P)
\end{lstlisting}




% *******************************************
% SECTION
% ******************************************* 
\section{Task 3: Encryption Mode -- ECB vs. CBC} 

The file {\tt pic\_original.bmp} can be downloaded from 
this lab's website, and it contains a simple picture. 
We would like to encrypt this picture, so people without the 
encryption keys cannot know what is in the picture. Please 
encrypt the file using the ECB (Electronic Code Book)  
and CBC (Cipher Block Chaining) modes, and then do the following: 

\begin{enumerate}
\item Let us treat the encrypted picture as 
a picture, and use a picture viewing software to display it. However,
For the {\tt .bmp} file, the first 54 bytes contain the header information 
about the picture, we have to set it correctly, so 
the encrypted file can be treated as a legitimate {\tt .bmp} file.
We will replace the header of the encrypted picture 
with that of the original picture. We can use 
the \texttt{bless} hex editor tool (already installed 
on our VM) to directly modify binary files. We can also use the 
following commands to get the header from \texttt{p1.bmp},
the data from \texttt{p2.bmp} (from offset 55 to the end of the file), 
and then combine the header and data
together into a new file.

\begin{lstlisting}
$ head -c 54 p1.bmp  > header
$ tail -c +55 p2.bmp > body
$ cat header body > new.bmp
\end{lstlisting}


\item Display the encrypted picture using a picture 
viewing program (we have installed an image viewer program called
\texttt{eog} on our VM). 
Can you derive any useful information
about the original picture from the encrypted picture?
Please explain your observations.

\end{enumerate}

Select a  picture of your choice, repeat the experiment above, 
and report your observations. 




% *******************************************
% SECTION
% ******************************************* 
\section{Task 4: Padding}

For block ciphers, when the size of a plaintext is not a multiple
of the block size, padding may be required. 
All the block ciphers normally use PKCS\#5 padding, which is known 
as standard block padding (see Chapter 21.4 of the SEED book for details). 
We will conduct the following experiments to 
understand how this type of padding works:

\begin{enumerate}
\item Use ECB, CBC, CFB, and OFB modes to encrypt a file (you can pick
any cipher).  Please report which modes have paddings and which ones 
do not. For those that do not need paddings, please explain why.


\item Let us create three files, which contain 5 bytes, 10 bytes, and 16 bytes, respectively. 
We can use the following \texttt{"echo -n"} command to create such files. The following example
creates a file \texttt{f1.txt} with length 5 (without the \texttt{-n} option, the length will
be 6, because a newline character will be added by \texttt{echo}):  

\begin{lstlisting}[backgroundcolor=]
$ echo -n "12345" > f1.txt
\end{lstlisting}

We then use \texttt{"openssl enc -aes-128-cbc -e"} to encrypt these three files using 
128-bit AES with CBC mode.  Please describe
the size of the encrypted files. 

We would like to see what is added to the padding during the encryption. To achieve
this goal, we will decrypt these files using \texttt{"openssl enc -aes-128-cbc -d"}.
Unfortunately, decryption by default will automatically remove the padding, making it 
impossible for us to see the padding. However, the command does have an option called
\texttt{"-nopad"}, which disables the padding, i.e., during the decryption, the command will not 
remove the padded data. Therefore, by looking at the decrypted 
data, we can see what data are used in the padding. 
Please use this technique to figure out what paddings are added to the three files. 

It should be noted that padding data may not be printable, so you need to
use a hex tool to display the content. The following example shows 
how to display a file in the hex format:

\begin{lstlisting}
$ hexdump -C p1.txt
00000000  31 32 33 34 35 36 37 38  39 49 4a 4b 4c 0a   |123456789IJKL.|
$ xxd p1.txt
00000000: 3132 3334 3536 3738 3949 4a4b 4c0a            123456789IJKL.
\end{lstlisting}
 
\end{enumerate}



% *******************************************
% SECTION
% ******************************************* 
\section{Task 5: Error Propagation -- Corrupted Cipher Text} 


To understand the error propagation property of various encryption modes, we would like to
do the following exercise:

\begin{enumerate}
\item Create a text file that is at least 1000 bytes long.  
\item Encrypt the file using the AES-128 cipher.
\item Unfortunately, a single bit of the 55th byte in the encrypted 
      file got corrupted. You can achieve this corruption using 
      the \texttt{bless} hex editor.
\item Decrypt the corrupted ciphertext file using the correct key and IV.
\end{enumerate}

Please answer the following question: 
How much information can you recover by decrypting the corrupted file, if the 
encryption mode is ECB, CBC, CFB, or OFB, respectively? Please answer this 
question before you conduct this task, and then find out whether your answer
is correct or wrong after you finish this task. 
Please provide justification.





% *******************************************
% SECTION
% ******************************************* 
\section{Task 6: Initial Vector (IV) and Common Mistakes}

Most of the encryption modes require an initial vector (IV). Properties of an IV depend 
on the cryptographic scheme used. If we are not careful in selecting IVs,
the data encrypted by us may not be secure at all, even though we are using 
a secure encryption algorithm and mode. The objective of this task is to
help students understand the problems if an IV is not selected properly. 
The detailed guidelines for this task is provided in 
Chapter 21.5 of the SEED book.


% -------------------------------------------
% SUBSECTION
% ------------------------------------------- 
\subsection{Task 6.1} 

A basic requirement for IV is \textit{uniqueness},
which means that no IV may be reused under the same key. To understand why,
please encrypt the same plaintext using (1) two different IVs, and (2)
the same IV. Please describe your observation, based on which, explain why
IV needs to be unique.   



% -------------------------------------------
% SUBSECTION
% ------------------------------------------- 
\subsection{Task 6.2. Common Mistake: Use the Same IV} 

One may argue that if the plaintext does not repeat, using 
the same IV is safe. Let us look at the Output Feedback (OFB) mode. 
Assume that the attacker gets hold of a plaintext (\texttt{P1})  
and a ciphertext (\texttt{C1}) , 
can he/she decrypt other encrypted messages if the IV is always the same? 
You are given the following information, please try to figure out
the actual content of \texttt{P2} based on \texttt{C2}, \texttt{P1},
and \texttt{C1}. 


\begin{lstlisting}
Plaintext  (P1): This is a known message!
Ciphertext (C1): a469b1c502c1cab966965e50425438e1bb1b5f9037a4c159

Plaintext  (P2): (unknown to you)
Ciphertext (C2): bf73bcd3509299d566c35b5d450337e1bb175f903fafc159
\end{lstlisting}
 
If we replace OFB in this experiment with 
CFB (Cipher Feedback), how much of \texttt{P2} can be revealed? You
only need to answer the question; there is no need to demonstrate that.

The attack used in this experiment is called the \textit{known-plaintext
attack}, which is an attack model for cryptanalysis where the
attacker has access to both the plaintext and its
encrypted version (ciphertext). If this can lead to
the revealing of further secret information, the encryption scheme is 
not considered as secure. 

%For OFB, if plaintext and ciphertext is know, we can get the random sequence (one-time
%pad is not one-time).

%For CFB, known-plaintext, the 1st block will be compromised



% -------------------------------------------
% SUBSECTION
% ------------------------------------------- 
\subsection{Task 6.3. Common Mistake: Use a Predictable IV} 

From the previous tasks, we now know that IVs cannot 
repeat. Another important requirement on IV is that 
IVs need to be unpredictable for many schemes, i.e., IVs need to
be randomly generated. In this task, we will see what is going to happen if
IVs are predictable. 

Assume that Bob just sent out an encrypted message, and Eve knows that its 
content is either \texttt{Yes} or \texttt{No}; Eve can see the ciphertext and the IV used 
to encrypt the message, but since the encryption algorithm AES is quite
strong, Eve has no idea what the actual content is. However, since Bob uses
predictable IVs, Eve knows exactly what IV Bob is going to use next.
The following summarizes what Bob and Eve know:

\begin{lstlisting}
Encryption method: 128-bit AES with CBC mode.

Key (in hex):    00112233445566778899aabbccddeeff (known only to Bob)
Ciphertext (C1): bef65565572ccee2a9f9553154ed9498 (known to both)
IV used on P1 (known to both) 
     (in ascii): 1234567890123456 
     (in hex)  : 31323334353637383930313233343536 
Next IV (known to both)
     (in ascii): 1234567890123457 
     (in hex)  : 31323334353637383930313233343537 
\end{lstlisting}


A good cipher should not only tolerate the known-plaintext attack described
previously, it should also tolerate the \textit{chosen-plaintext attack}, 
which is an attack model for cryptanalysis where the attacker can obtain the
ciphertext for an arbitrary plaintext. Since AES is a strong cipher that 
can tolerate the chosen-plaintext attack, Bob does not mind encrypting any 
plaintext given by Eve; he does use a different IV for each plaintext, 
but unfortunately, the IVs he generates are not random, and they can
always be predictable.


Your job is to construct a message \texttt{P2} and ask Bob to encrypt it and give you the
ciphertext. Your objective is to use this opportunity to 
figure out whether the actual content of \texttt{P1}  is \texttt{Yes} or \texttt{No}.  



%\item \textbf{Task 6.4. Encrypting IV or not.} 
%Should IV be encrypted? See CBC mode. In this case, the actual IV becomes a constant. Do
%an experiment (two plaintext with different IVs will generate the same ciphertext)



% -------------------------------------------
% SUBSECTION
% ------------------------------------------- 
\subsection{Additional Readings}

There are more advanced cryptanalysis on IV that is beyond the scope of this lab. Students 
can read the article posted in this URL: \url{https://defuse.ca/cbcmodeiv.htm}. 
Because the requirements on IV really depend on cryptographic schemes, it is hard to
remember what properties should be maintained when we select an IV.
However, we will be safe if we always use a new IV for each encryption, and the 
new IV needs to be generated using a good pseudo random number 
generator, so it is unpredictable by adversaries. 
See another SEED lab (Random Number Generation Lab) for 
details on how to generate cryptographically strong pseudo random numbers. 



% *******************************************
% SECTION
% ******************************************* 
\section{Task 7: Programming using the Crypto Library}

This task is mainly designed for students in Computer Science/Engineering
or related fields, where programming is required. Students should
check with their professors to see whether this task is required 
for their courses or not.


In this task, you are given a plaintext and a ciphertext, and 
your job is to find the key that is used for the encryption. 
You do know the following facts:

\begin{itemize} 
\item The {\tt aes-128-cbc} cipher is used for the encryption. 
\item The key used to encrypt this plaintext is an English
word shorter than 16 characters; the word can be found from a typical 
English dictionary.  Since the word has less than 16 characters (i.e. 128
bits), pound signs (\texttt{\#}: hexadecimal value is \texttt{0x23}) 
are appended to the end of the word to form a key of 128 bits.
\end{itemize} 
  

Your goal is to write a program to 
find out the encryption key. You can download a English word list 
from the Internet.  We have also linked one on the web page
of this lab. The plaintext, ciphertext, and IV are listed
in the following:


\begin{lstlisting}
Plaintext (total 21 characters): This is a top secret.
Ciphertext (in hex format): 764aa26b55a4da654df6b19e4bce00f4  
                            ed05e09346fb0e762583cb7da2ac93a2 
IV (in hex format):         aabbccddeeff00998877665544332211
\end{lstlisting}



You need to pay attention to the following issues: 

\begin{itemize} 
\item If you choose to store the plaintext message in 
a file, and feed the file to your program, you need to check
whether the file length is 21. If you type the message in a text editor,
you need to be aware that some editors may add a special
character to the end of the file. The easiest way to store the message
in a file is to use the following 
command (the \texttt{-n} flag
tells \texttt{echo} not to  add a trailing newline):


\begin{lstlisting}[backgroundcolor=]
$ echo -n "This is a top secret." > file
\end{lstlisting}
 
\item In this task, you are supposed to write your own program 
to invoke the crypto library. No credit will be given 
if you simply use the {\tt openssl} commands to do this task. 
Sample code can be found from the following URL:
\begin{lstlisting}[backgroundcolor=]
 https://www.openssl.org/docs/man1.0.2/crypto/EVP_EncryptInit.html
\end{lstlisting}


\item When you compile your code using  \texttt{gcc},
do not forget to include the \texttt{-lcrypto} flag, because your code 
needs the \texttt{crypto} library. See the following example: 

\begin{lstlisting}[backgroundcolor=]
$ gcc -o myenc myenc.c -lcrypto 
\end{lstlisting}

\end{itemize} 

\paragraph{Note to instructors.} 
We encourage instructors to generate their own plaintext and ciphertext using 
a different key; this way students will not be able to get the answer from
another place or from previous courses. Instructors can use the
following commands to achieve this goal (please replace the word \texttt{example}
with another secret word, and add the correct number of \# signs to make
the length of the string to be 16):

\begin{lstlisting}
$ echo -n "This is a top secret." > plaintext.txt
$ echo -n "example#########" > key
$ xxd -p key
6578616d706c65232323232323232323
$ openssl enc -aes-128-cbc -e -in plaintext.txt -out ciphertext.bin \
      -K  6578616d706c65232323232323232323  \
      -iv 010203040506070809000a0b0c0d0e0f  \
$ xxd -p ciphertext.bin
e5accdb667e8e569b1b34f423508c15422631198454e104ceb658f5918800c22
\end{lstlisting}
 


% *******************************************
% SECTION
% *******************************************
\section{Submission}

\seedsubmission



\end{document}
%%%%%%%%%%%%%%%%%%%%%%%%%%%%%%%%%%%%%%%%%%%%%%%%%%%%%%%%%%%%%%%%%%%%%
%%%%%%%%%%%%%%%%%%%%%%%%%%%%%%%%%%%%%%%%%%%%%%%%%%%%%%%%%%%%%%%%%%%%%



% This part is no longer used 
\subsection{Task 6: Encrypting UDP Communication}

\mynote{
{\bf For Instructor:} The goal of Task 6 and Task 7 is to get students to use encryption
programmatically. It is not very necessary to do both tasks. Instructors
can choose either one of them (or both). If the encryption topic is covered
in the context of a Network Security class, Task 6 is a better choice, 
because it uses encryption to secure a communication channel. 
}

\noindent
So far, we have learned how to use the existing tools provided by
\openssl to encrypt and decrypt messages. We would like to build our own
tools. Therefore,  the objective of this task
is to learn how to use \openssl's  crypto library to encrypt/decrypt 
messages in programs. 


OpenSSL provides an API called EVP, which 
is a high-level interface to cryptographic functions. 
Although OpenSSL also has direct interfaces for each individual
encryption algorithm, the EVP library provides 
a common interface for various encryption algorithms. 
To ask EVP to use a specific algorithm, we simply need to 
pass our choice to the EVP interface.
A sample code is given in 
\url{http://www.openssl.org/docs/crypto/EVP_EncryptInit.html}.
Please get familiar with this program. 


In this task, we will build a client/server program, which use the UDP protocol and
the communication is encrypted. The client and server side of the program should 
satisfy the following requirements:

\begin{itemize}
  \item The data sent between the client and server should be encrypted using 
    an encryption algorithm and mode of your choice (e.g. \texttt{aes-128-cbc}).

  \item The client program runs in a loop, and keeps asking the user to type a message. 
    Once the user hits the return, the message will be sent to the server using UDP. 
    Upon receiving the UDP packet, the server simply prints out the decrypted message. 

  \item The client and server need a common secret key to encrypt the communication.
    Obviously, they cannot send the key to each other in plaintext. In practice, they 
    need to run a key-exchange protocol to get the key. We will cover that protocol
    after we have learned the public key encryption. In this task, we assume that the client
    and server has already obtained a secret key between them, so we will just    
    type in the key as an argument when we run the client and server programs. 
    {\bf Note:} you cannot hardcode the key in your program, because it is a very bad habit.

  \item The Initial Vector (IV) needs to be randomly generated by the client, 
    and should be sent to the server in plaintext. You can put the IV at the beginning of the
    UDP data. In your report, please answer why the client cannot reuse the IV throughout its
    execution (hint: try to fix your IV in your client program, then type the same message, and
    use Wireshark to see what you can infer about the encrypted data). 
\end{itemize}


