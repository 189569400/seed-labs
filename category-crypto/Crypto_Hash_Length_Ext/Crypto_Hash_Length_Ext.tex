%%%%%%%%%%%%%%%%%%%%%%%%%%%%%%%%%%%%%%%%%%%%%%%%%%%%%%%%%%%%%%%%%%%%%%
%%  Copyright by Wenliang Du.                                       %%
%%  This work is licensed under the Creative Commons                %%
%%  Attribution-NonCommercial-ShareAlike 4.0 International License. %%
%%  To view a copy of this license, visit                           %%
%%  http://creativecommons.org/licenses/by-nc-sa/4.0/.              %%
%%%%%%%%%%%%%%%%%%%%%%%%%%%%%%%%%%%%%%%%%%%%%%%%%%%%%%%%%%%%%%%%%%%%%%

\newcommand{\commonfolder}{../../common-files}

\documentclass[11pt]{article}

\usepackage[most]{tcolorbox}
\usepackage{times}
\usepackage{epsf}
\usepackage{epsfig}
\usepackage{amsmath, alltt, amssymb, xspace}
\usepackage{wrapfig}
\usepackage{fancyhdr}
\usepackage{url}
\usepackage{verbatim}
\usepackage{fancyvrb}
\usepackage{adjustbox}
\usepackage{listings}
\usepackage{color}
\usepackage{subfigure}
\usepackage{cite}
\usepackage{sidecap}
\usepackage{pifont}
\usepackage{mdframed}
\usepackage{textcomp}
\usepackage{enumitem}


% Horizontal alignment
\topmargin      -0.50in  % distance to headers
\oddsidemargin  0.0in
\evensidemargin 0.0in
\textwidth      6.5in
\textheight     8.9in 

\newcommand{\todo}[1]{
\vspace{0.1in}
\fbox{\parbox{6in}{TODO: #1}}
\vspace{0.1in}
}


\newcommand{\unix}{{\tt Unix}\xspace}
\newcommand{\linux}{{\tt Linux}\xspace}
\newcommand{\minix}{{\tt Minix}\xspace}
\newcommand{\ubuntu}{{\tt Ubuntu}\xspace}
\newcommand{\setuid}{{\tt Set-UID}\xspace}
\newcommand{\openssl} {\texttt{openssl}}


\pagestyle{fancy}
\lhead{\bfseries SEED Labs}
\chead{}
\rhead{\small \thepage}
\lfoot{}
\cfoot{}
\rfoot{}


\definecolor{dkgreen}{rgb}{0,0.6,0}
\definecolor{gray}{rgb}{0.5,0.5,0.5}
\definecolor{mauve}{rgb}{0.58,0,0.82}
\definecolor{lightgray}{gray}{0.90}


\lstset{%
  frame=none,
  language=,
  backgroundcolor=\color{lightgray},
  aboveskip=3mm,
  belowskip=3mm,
  showstringspaces=false,
%  columns=flexible,
  basicstyle={\small\ttfamily},
  numbers=none,
  numberstyle=\tiny\color{gray},
  keywordstyle=\color{blue},
  commentstyle=\color{dkgreen},
  stringstyle=\color{mauve},
  breaklines=true,
  breakatwhitespace=true,
  tabsize=3,
  columns=fullflexible,
  keepspaces=true,
  escapeinside={(*@}{@*)}
}

\newcommand{\newnote}[1]{
\vspace{0.1in}
\noindent
\fbox{\parbox{1.0\textwidth}{\textbf{Note:} #1}}
%\vspace{0.1in}
}


%% Submission
\newcommand{\seedsubmission}{You need to submit a detailed lab report, with screenshots,
to describe what you have done and what you have observed.
You also need to provide explanation
to the observations that are interesting or surprising.
Please also list the important code snippets followed by
explanation. Simply attaching code without any explanation will not
receive credits.}

%% Book
\newcommand{\seedbook}{\textit{Computer \& Internet Security: A Hands-on Approach}, 2nd
Edition, by Wenliang Du. See details at \url{https://www.handsonsecurity.net}.}

%% Videos
\newcommand{\seedisvideo}{\textit{Internet Security: A Hands-on Approach},
by Wenliang Du. See details at \url{https://www.handsonsecurity.net/video.html}.}

\newcommand{\seedcsvideo}{\textit{Computer Security: A Hands-on Approach},
by Wenliang Du. See details at \url{https://www.handsonsecurity.net/video.html}.}

%% Lab Environment
\newcommand{\seedenvironment}{This lab has been tested on our pre-built
Ubuntu 16.04 VM, which can be downloaded from the SEED website. }

\newcommand{\seedenvironmentA}{This lab has been tested on our pre-built
Ubuntu 16.04 VM, which can be downloaded from the SEED website. }

\newcommand{\seedenvironmentB}{This lab has been tested on our pre-built
Ubuntu 20.04 VM, which can be downloaded from the SEED website. }

\newcommand{\seedenvironmentAB}{This lab has been tested on our pre-built
Ubuntu 16.04 and 20.04 VMs, which can be downloaded from the SEED website. }

\newcommand{\nodependency}{Since we use containers to set up the lab environment, 
this lab does not depend too much on our SEED VM. You can do this lab
using other VMs or physical machines. }







\newcommand{\seedlabcopyright}[1]{
\vspace{0.1in}
\fbox{\parbox{6in}{\small Copyright \copyright\ {#1}\ \ by Wenliang Du.\\
      This work is licensed under a Creative Commons
      Attribution-NonCommercial-ShareAlike 4.0 International License.
      If you remix, transform, or build upon the material, 
      this copyright notice must be left intact, or reproduced in a way that is reasonable to
      the medium in which the work is being re-published.}}
\vspace{0.1in}
}






\lhead{\bfseries SEED Labs -- Hash Length Extension Attack Lab}


\begin{document}


\begin{center}
{\LARGE Hash Length Extension Attack Lab}
\end{center}

\seedlabcopyright{2019}


\section{Introduction}

When a client and a server communicate over the internet, they are subject to MITM attacks.
An attacker can intercept the request from the client. The attacker may choose to modify 
the data and send the modified request to the server. In such a scenario, the server needs 
to verify the integrity of the request received. The standard way to verify the integrity 
of the request is to attach a tag called MAC to the request. There are many ways to 
calculate MAC, and some of the methods are not secure.  


MAC is calculated from a secret key and a message. A naive way to 
calculate MAC is to concatenate the key with the message and
calculate the one way hash of the resulting string. This method seems to be fine, but 
it is subject to an attack called length extension attack, which  
allows attackers to modify the message
while still being able to generate a valid MAC based on the modified message, without knowing the secret key.

The objective of this lab is to help students understand how 
the length extension attack works. Students will launch the 
attack against a server program; they will forge a valid command 
and get the server to execute the command. 



\paragraph{Readings.} Detailed coverage of the one way hash function can be 
fond in the following: 

\begin{itemize}
\item Chapter 22 of the SEED Book, \seedbook
\end{itemize}


\paragraph{Lab environment.} 
\seedenvironmentB
\nodependency




% *******************************************
% SECTION
% *******************************************
\section{Lab Environment}

We have set up a web server for this lab. 
A client can send a list of commands to this server. 
Each request must attach a MAC computed based on a secret
key and the list of commands. The server will 
only execute the commands in the request if 
the MAC is verified successfully. 
We will use the host VM as the client and use 
a container for the web server.


\paragraph{Container Setup and Commands.}
%%%%%%%%%%%%%%%%%%%%%%%%%%%%%%%%%%%%%%%%%%%%
Please download the
\texttt{Labsetup.zip} file to your VM from the lab's website,
unzip it, enter the \texttt{Labsetup} folder, and 
use the \texttt{docker-compose.yml} file to 
set up the lab environment. Detailed explanation
of the content in this file and all the involved 
\texttt{Dockerfile} can be found from the 
user manual, which is linked to the website of this lab.
If this is the first time you set up a SEED lab environment
using containers, it is very important that you read 
the user manual. 

In the following, we list some of the commonly
used commands related to Docker and Compose. 
Since we are going to use 
these commands very frequently, we have created aliases for them
in the \texttt{.bashrc} file (in our provided SEEDUbuntu 20.04 VM).


\begin{lstlisting}
$ docker-compose build  # Build the container image
$ docker-compose up     # Start the container
$ docker-compose down   # Shut down the container

// Aliases for the Compose commands above
$ dcbuild       # Alias for: docker-compose build
$ dcup          # Alias for: docker-compose up
$ dcdown        # Alias for: docker-compose down
\end{lstlisting}


All the containers will be running in the background. To run
commands on a container, we often need to get a shell on
that container. We first need to use the \texttt{"docker ps"}  
command to find out the ID of the container, and then
use \texttt{"docker exec"} to start a shell on that 
container. We have created aliases for them in
the \texttt{.bashrc} file.

\begin{lstlisting}
$ dockps        # Alias for: docker ps --format "{{.ID}}  {{.Names}}" 
$ docksh <id>   # Alias for: docker exec -it <id> /bin/bash

# The following example shows how to get a shell inside hostC
$ dockps
b1004832e275  hostA-10.9.0.5
0af4ea7a3e2e  hostB-10.9.0.6
9652715c8e0a  hostC-10.9.0.7

$ docksh 96
root@9652715c8e0a:/#  

# Note: If a docker command requires a container ID, you do not need to 
#       type the entire ID string. Typing the first few characters will 
#       be sufficient, as long as they are unique among all the containers. 
\end{lstlisting}


If you encounter problems when setting up the lab environment, 
please read the ``Common Problems'' section of the manual
for potential solutions.


%%%%%%%%%%%%%%%%%%%%%%%%%%%%%%%%%%%%%%%%%%%%



\paragraph{About the web server.}
We use the domain \texttt{www.seedlab-hashlen.com} to host the server program. 
In our VM, we map this hostname to 
the web server container (\texttt{10.9.0.80}). This can be
achieved by adding the following entry to the 
\texttt{/etc/hosts} file.

\begin{lstlisting}
10.9.0.80  www.seedlab-hashlen.com
\end{lstlisting}


The server code is in the \path{Labsetup/image_flask/app} folder.
It has two directories. 
The \texttt{www} directory contains the server code, and the 
\texttt{LabHome} directory contains a secret file and 
the key used for computing the MAC. 




\paragraph{Sending requests.} The server program accepts the following commands:

\begin{itemize}
\item The \texttt{lstcmd} command: the server will list all the files in the
\texttt{LabHome} folder.
\item The \texttt{download} command: the server will return the contents of the 
specified file from the \texttt{LabHome} directory.
\end{itemize}

A typical request sent by the client to the server is shown below. 
The server requires a \texttt{uid} argument to be passed. It uses 
\texttt{uid} to get the MAC key from 
\texttt{LabHome/key.txt}. The command 
in the example below is \texttt{lstcmd}, and its value is set to \texttt{1}. It requests
the server to list all the files. The last argument is the MAC computed based on the 
secret key (shared by the client and the server) and the command arguments.
Before executing the command, the server will verify the MAC to ensure
the command's integrity. 

\begin{lstlisting}
http://www.seedlab-hashlen.com/?myname=JohnDoe&uid=1001&lstcmd=1
&mac=dc8788905dbcbceffcdd5578887717c12691b3cf1dac6b2f2bcfabc14a6a7f11
\end{lstlisting}


Students should replace the value \texttt{JohnDoe} in the 
\texttt{myname} field with their actual names (no space 
is allowed). This parameter is to make sure that different students'
results are different, so students cannot copy from one
another. The server does not use this argument, but it checks 
whether the argument is present or not. Requests will be rejected
if this field is not included. 
Instructors
can use this argument to check whether students have done the work by
themselves. No point will be given if students do not use
their real names in this task.


The following shows another example. The request includes 
two commands: list all the files and download the 
file \texttt{secret.txt}. Similarly, a valid MAC needs to be attached, 
or the server will not execute these commands.  

\begin{lstlisting}
http://www.seedlab-hashlen.com/?myname=JohnDoe&uid=1001&lstcmd=1
&download=secret.txt
&mac=dc8788905dbcbceffcdd5578887717c12691b3cf1dac6b2f2bcfabc14a6a7f11
\end{lstlisting}



% *******************************************
% SECTION
% ******************************************* 
\section{Tasks}


% -------------------------------------------
% SUBSECTION
% ------------------------------------------- 
\subsection{Task 1: Send Request to List Files}

In this task, we will send a benign request to the server so we
can see how the server responds to the request. The request we want to send
is as follows:

\begin{lstlisting}
http://www.seedlab-hashlen.com/?myname=<name>&uid=<need-to-fill>
&lstcmd=1&mac=<need-to-calculate>
\end{lstlisting}

To send such a request, other than using our real names, we need to fill in the 
two missing arguments. Students need
to pick a uid number from the \texttt{key.txt} in the \texttt{LabHome}
directory. This file contains a list of colon-separated uid and key values. Students
can use any uid and its associated key value. For example, students can use uid
\texttt{1001} and its key \texttt{123456}.

The second missing argument is the MAC, which can be calculated by
concatenating the key with the contents of the requests \texttt{R} 
(the argument part only), with a colon added in between. 
See the following example:
 
\begin{lstlisting}
Key:R = 123456:myname=JohnDoe&uid=1001&lstcmd=1
\end{lstlisting}
 
The MAC will be calculated using the following command:

\begin{lstlisting}
$ echo -n "123456:myname=JohnDoe&uid=1001&lstcmd=1" | sha256sum
7d5f750f8b3203bd963d75217c980d139df5d0e50d19d6dfdb8a7de1f8520ce3  -
\end{lstlisting}

We can then construct the complete request and send it to the server program using
the browser:

\begin{lstlisting}
http://www.seedlab-hashlen.com/?myname=JohnDoe&uid=1001&lstcmd=1
&mac=7d5f750f8b3203bd963d75217c980d139df5d0e50d19d6dfdb8a7de1f8520ce3 
\end{lstlisting}

\paragraph{Task.} Please send a download command to the server, and 
show that you can get the results back. 


% -------------------------------------------
% SUBSECTION
% ------------------------------------------- 
\subsection{Task 2: Create Padding}

To conduct the hash length extension attack, we 
need to understand how padding is calculated for
one-way hash. The block size of SHA-256 is 64 bytes, so 
a message \texttt{M} will be padded to the multiple of 
64 bytes during the hash calculation. 
According to RFC 6234, paddings for SHA256 consist of one byte
of \textbackslash x80, followed by a many 0's, followed by a 
64-bit (8 bytes) length field (the length is the number 
of \textbf{bits} in the \texttt{M}). 

Assume that the original message is \texttt{M = "This is a test message"}. 
The length of \texttt{M} is \texttt{22} bytes, so the 
padding is \texttt{64 - 22 = 42} bytes, including \texttt{8} bytes of the length field.   
The length of \texttt{M} in term of bits
is \texttt{22 * 8 = 176 = 0xB0}.
SHA256 will be performed in the following padded 
message: 

\begin{lstlisting}
"This is a test message"
"\x80"
"\x00\x00\x00\x00\x00\x00\x00\x00\x00\x00"
"\x00\x00\x00\x00\x00\x00\x00\x00\x00\x00"
"\x00\x00\x00\x00\x00\x00\x00\x00\x00\x00"
"\x00\x00\x00"
"\x00\x00\x00\x00\x00\x00\x00\xB0"
\end{lstlisting}


It should be noted that 
the length field uses the Big-Endian byte order, i.e., 
if the length of the message is \texttt{0x012345}, 
the length field in the padding should be:
\begin{lstlisting}
"\x00\x00\x00\x00\x00\x01\x23\x45"
\end{lstlisting}


\paragraph{Task.} Students need to construct the padding for the following
message (the actual value of the \texttt{<key>} and \texttt{<uid>} should
be obtained from the \texttt{LabHome/key.txt} file.  

\begin{lstlisting}
<key>:myname=<name>&uid=<uid>&lstcmd=1
\end{lstlisting}



% -------------------------------------------
% SUBSECTION
% ------------------------------------------- 
\subsection{Task 3: Compute MAC using Secret Key}

In this task, we will add an extra message \texttt{N = "Extra message"} 
to the padded original message \texttt{M = "This is a test message"}, 
and compute its hash value. The program is listed below. 

\begin{lstlisting}
/* calculate_mac.c */
#include <stdio.h>
#include <openssl/sha.h>

int main(int argc, const char *argv[])
{
   SHA256_CTX c;
   unsigned char buffer[SHA256_DIGEST_LENGTH];
   int i;

   SHA256_Init(&c);
   SHA256_Update(&c,
      "This is a test message"
      "\x80"
      "\x00\x00\x00\x00\x00\x00\x00\x00\x00\x00"
      "\x00\x00\x00\x00\x00\x00\x00\x00\x00\x00"
      "\x00\x00\x00\x00\x00\x00\x00\x00\x00\x00"
      "\x00\x00\x00"
      "\x00\x00\x00\x00\x00\x00\x00\xB0"
      "Extra message",
      64+13);
   SHA256_Final(buffer, &c);

   for(i = 0; i < 32; i++) {
     printf("%02x", buffer[i]);
   }			
   printf("\n");
   return 0;	 
}
\end{lstlisting}

Students can compile and run the above program as follows:

\begin{lstlisting}
$ gcc calculate_mac.c -o calculate_mac -lcrypto
$ ./calculate_mac 
\end{lstlisting}


\paragraph{Task.}
Students should change the code in the listing above and compute 
the MAC for the following request (assume that we know
the secret MAC key):

\begin{lstlisting}
http://www.seedlab-hashlen.com/?myname=<name>&uid=<uid>
&lstcmd=1<padding>&download=secret.txt
&mac=<hash-value>
\end{lstlisting}

Just like the previous task, the value of \texttt{<name>} should be your actual name. 
The value of the \texttt{<uid>} and the MAC key should be 
obtained from the \texttt{LabHome/key.txt} file.
Please send this request to the server, and see whether you can
successfully download the \texttt{secret.txt} file.  

It should be noted that in the URL, 
all the hexadecimal numbers in the padding 
need to be encoded by changing
\texttt{\textbackslash x} to \texttt{\%}. For example, 
\texttt{\textbackslash x80} in the padding should be 
replaced with \texttt{\%80} in the URL above. 
On the server side, encoded data in the URL 
will be changed back to the hexadecimal numbers. 


% -------------------------------------------
% SUBSECTION
% ------------------------------------------- 
\subsection{Task 4: The Length Extension Attack}


In the previous task, we show how a legitimate user calculates the MAC (with
the knowledge of the MAC key). In this task, we will do it as an attacker,
i.e., we do not know the MAC key. However, we do know the MAC of a valid 
request \texttt{R}. Our job is to forge a new request based on \texttt{R},
while still being able to compute the valid MAC. 

Given the original message \texttt{M="This is a test message"} and its 
MAC value, we will show how to add a message 
\texttt{"Extra message"} to the end of the padded \texttt{M}, and 
then compute its MAC, without knowing the secret MAC key. 

\begin{lstlisting}
$ echo -n "This is a test message" | sha256sum
6f3438001129a90c5b1637928bf38bf26e39e57c6e9511005682048bedbef906
\end{lstlisting}

The program below can be used to compute the MAC for the 
new message: 

\begin{lstlisting}[escapechar=|]
/* length_ext.c */
#include <stdio.h>
#include <arpa/inet.h>
#include <openssl/sha.h>

int main(int argc, const char *argv[])
{
  int i;
  unsigned char buffer[SHA256_DIGEST_LENGTH];
  SHA256_CTX c;
  
  SHA256_Init(&c);
  for(i=0; i<64; i++)
     SHA256_Update(&c, "*", 1);
  
  // MAC of the original message M (padded)
  c.h[0] = htole32(0x6f343800);
  c.h[1] = htole32(0x1129a90c);
  c.h[2] = htole32(0x5b163792);
  c.h[3] = htole32(0x8bf38bf2);
  c.h[4] = htole32(0x6e39e57c);	
  c.h[5] = htole32(0x6e951100);
  c.h[6] = htole32(0x5682048b);	
  c.h[7] = htole32(0xedbef906);
  
  // Append additional message
  SHA256_Update(&c, "Extra message", 13); 
  SHA256_Final(buffer, &c);

  for(i = 0; i < 32; i++) {
     printf("%02x", buffer[i]);
  }	 
  printf("\n");
  return 0;
}
\end{lstlisting}

Students can compile the program as follows:

\begin{lstlisting}
$ gcc length_ext.c -o length_ext -lcrypto
\end{lstlisting}

\paragraph{Task.} Students should first generate a valid MAC for the
following request (where \texttt{<uid>} and the MAC key should be 
obtained from the \texttt{LabHome/key.txt} file): 

\begin{lstlisting}
http://www.seedlab-hashlen.com/?myname=<name>&uid=<uid>
&lstcmd=1&mac=<mac>
\end{lstlisting}

Based on the \texttt{<mac>} value calculated above, please construct a new request that
includes the \texttt{download} command. You are not allowed to use the secret key this time.
The URL looks like below.

\begin{lstlisting}
http://www.seedlab-hashlen.com/?myname=<name>&uid=<uid>
&lstcmd=1<padding>&download=secret.txt&mac=<new-mac>
\end{lstlisting}
 
Please send the constructed request to the server, and show that you can
successfully get the content of the \texttt{secret.txt} file.  



% -------------------------------------------
% SUBSECTION
% ------------------------------------------- 
\subsection{Task 5: Attack Mitigation using HMAC}

In the tasks so far, we have observed the damage caused when a developer
computes a MAC in an insecure way by concatenating the key and the message.
In this task, we will fix the mistake made by the developer. The standard
way to calculate MACs is to use HMAC. Students
should modify the server program's \texttt{verify\textunderscore mac()}
function and use Python's \texttt{hmac} module to calculate the MAC. The
function resides in \texttt{lab.py}. Given a key and message (both of type
string), the HMAC can be computed as shown below:

\begin{lstlisting}
mac = hmac.new(bytearray(key.encode('utf-8')), msg=message.encode('utf-8', 
      'surrogateescape'), digestmod=hashlib.sha256).hexdigest()
\end{lstlisting}

Students should repeat Task 1 to send a request to list files while using HMAC for 
the MAC calculation. Assuming that the chosen key is 123456, the HMAC can be computed 
in the following:

\begin{lstlisting}
$ python3
Python 3.5.2
[GCC 5.4.0 20160609] on linux
Type "help", "copyright", "credits" or "license" for more information
>>> import hmac
>>> import hashlib
>>> key='123456'
>>> message='lstcmd=1'
>>> hmac.new(bytearray(key.encode('utf-8')), msg=message.encode('utf-8', 
	'surrogateescape'), digestmod=hashlib.sha256).hexdigest()
\end{lstlisting}

Students should describe why a malicious request using length extension and
extra commands will fail MAC verification when the client and server use
HMAC.


% *******************************************
% SECTION
% ******************************************* 
\section{Submission} 

%%%%%%%%%%%%%%%%%%%%%%%%%%%%%%%%%%%%%%%%

You need to submit a detailed lab report, with screenshots,
to describe what you have done and what you have observed.
You also need to provide explanation
to the observations that are interesting or surprising.
Please also list the important code snippets followed by
explanation. Simply attaching code without any explanation will not
receive credits.

%%%%%%%%%%%%%%%%%%%%%%%%%%%%%%%%%%%%%%%%


\end{document}











