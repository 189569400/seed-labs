%%%%%%%%%%%%%%%%%%%%%%%%%%%%%%%%%%%%%%%%%%%%%%%%%%%%%%%%%%%%%%%%%%%%%%
%%  Copyright by Wenliang Du.                                       %%
%%  This work is licensed under the Creative Commons                %%
%%  Attribution-NonCommercial-ShareAlike 4.0 International License. %%
%%  To view a copy of this license, visit                           %%
%%  http://creativecommons.org/licenses/by-nc-sa/4.0/.              %%
%%%%%%%%%%%%%%%%%%%%%%%%%%%%%%%%%%%%%%%%%%%%%%%%%%%%%%%%%%%%%%%%%%%%%%

\documentclass[11pt]{article}

\usepackage[most]{tcolorbox}
\usepackage{times}
\usepackage{epsf}
\usepackage{epsfig}
\usepackage{amsmath, alltt, amssymb, xspace}
\usepackage{wrapfig}
\usepackage{fancyhdr}
\usepackage{url}
\usepackage{verbatim}
\usepackage{fancyvrb}
\usepackage{adjustbox}
\usepackage{listings}
\usepackage{color}
\usepackage{subfigure}
\usepackage{cite}
\usepackage{sidecap}
\usepackage{pifont}
\usepackage{mdframed}
\usepackage{textcomp}
\usepackage{enumitem}


% Horizontal alignment
\topmargin      -0.50in  % distance to headers
\oddsidemargin  0.0in
\evensidemargin 0.0in
\textwidth      6.5in
\textheight     8.9in 

\newcommand{\todo}[1]{
\vspace{0.1in}
\fbox{\parbox{6in}{TODO: #1}}
\vspace{0.1in}
}


\newcommand{\unix}{{\tt Unix}\xspace}
\newcommand{\linux}{{\tt Linux}\xspace}
\newcommand{\minix}{{\tt Minix}\xspace}
\newcommand{\ubuntu}{{\tt Ubuntu}\xspace}
\newcommand{\setuid}{{\tt Set-UID}\xspace}
\newcommand{\openssl} {\texttt{openssl}}


\pagestyle{fancy}
\lhead{\bfseries SEED Labs}
\chead{}
\rhead{\small \thepage}
\lfoot{}
\cfoot{}
\rfoot{}


\definecolor{dkgreen}{rgb}{0,0.6,0}
\definecolor{gray}{rgb}{0.5,0.5,0.5}
\definecolor{mauve}{rgb}{0.58,0,0.82}
\definecolor{lightgray}{gray}{0.90}


\lstset{%
  frame=none,
  language=,
  backgroundcolor=\color{lightgray},
  aboveskip=3mm,
  belowskip=3mm,
  showstringspaces=false,
%  columns=flexible,
  basicstyle={\small\ttfamily},
  numbers=none,
  numberstyle=\tiny\color{gray},
  keywordstyle=\color{blue},
  commentstyle=\color{dkgreen},
  stringstyle=\color{mauve},
  breaklines=true,
  breakatwhitespace=true,
  tabsize=3,
  columns=fullflexible,
  keepspaces=true,
  escapeinside={(*@}{@*)}
}

\newcommand{\newnote}[1]{
\vspace{0.1in}
\noindent
\fbox{\parbox{1.0\textwidth}{\textbf{Note:} #1}}
%\vspace{0.1in}
}


%% Submission
\newcommand{\seedsubmission}{You need to submit a detailed lab report, with screenshots,
to describe what you have done and what you have observed.
You also need to provide explanation
to the observations that are interesting or surprising.
Please also list the important code snippets followed by
explanation. Simply attaching code without any explanation will not
receive credits.}

%% Book
\newcommand{\seedbook}{\textit{Computer \& Internet Security: A Hands-on Approach}, 2nd
Edition, by Wenliang Du. See details at \url{https://www.handsonsecurity.net}.}

%% Videos
\newcommand{\seedisvideo}{\textit{Internet Security: A Hands-on Approach},
by Wenliang Du. See details at \url{https://www.handsonsecurity.net/video.html}.}

\newcommand{\seedcsvideo}{\textit{Computer Security: A Hands-on Approach},
by Wenliang Du. See details at \url{https://www.handsonsecurity.net/video.html}.}

%% Lab Environment
\newcommand{\seedenvironment}{This lab has been tested on our pre-built
Ubuntu 16.04 VM, which can be downloaded from the SEED website. }

\newcommand{\seedenvironmentA}{This lab has been tested on our pre-built
Ubuntu 16.04 VM, which can be downloaded from the SEED website. }

\newcommand{\seedenvironmentB}{This lab has been tested on our pre-built
Ubuntu 20.04 VM, which can be downloaded from the SEED website. }

\newcommand{\seedenvironmentAB}{This lab has been tested on our pre-built
Ubuntu 16.04 and 20.04 VMs, which can be downloaded from the SEED website. }

\newcommand{\nodependency}{Since we use containers to set up the lab environment, 
this lab does not depend too much on our SEED VM. You can do this lab
using other VMs or physical machines. }







\newcommand{\seedlabcopyright}[1]{
\vspace{0.1in}
\fbox{\parbox{6in}{\small Copyright \copyright\ {#1}\ \ by Wenliang Du.\\
      This work is licensed under a Creative Commons
      Attribution-NonCommercial-ShareAlike 4.0 International License.
      If you remix, transform, or build upon the material, 
      this copyright notice must be left intact, or reproduced in a way that is reasonable to
      the medium in which the work is being re-published.}}
\vspace{0.1in}
}






\newcommand{\formatFigs}{./Figs}
\newcommand{\dockerfile}{\texttt{Dockerfile}\xspace}


\lhead{\bfseries SEED Labs -- Format String Vulnerability Lab (64-bit)}


\begin{document}


\begin{center}
{\LARGE Format String Vulnerability Lab (64-bit)}
\end{center}


\seedlabcopyright{2020}



% *******************************************
% SECTION
% ******************************************* 
\section{Overview}


The \texttt{printf()} function in C is used to print out a string according to a format.  Its
first argument is called \textit{format string}, which defines how the string should be
formatted. Format strings use placeholders marked by the \texttt{\%} character for the
\texttt{printf()} function to fill in data during the printing.  The use of format strings is
not only limited to the \texttt{printf()} function; many other functions, such as
\texttt{sprintf()}, \texttt{fprintf()}, and \texttt{scanf()}, also use format strings. Some
programs allow users to provide the entire or part of the contents in a format string. If such
contents are not sanitized, malicious users can use this opportunity to get the program to run
arbitrary code. A problem like this is called \textit{format string vulnerability}.


The objective of this lab is for students to gain the first-hand
experience on format string vulnerabilities by putting what they have learned 
about the vulnerability from class into actions. 
Students will be given a program with a format string
vulnerability; their task is to exploit
the vulnerability to achieve the following damage: (1) crash the 
program, (2) read the internal memory of the program, (3) modify
the internal memory of the program, and most severely, 
(4) inject and execute malicious code using the victim program's privilege. 
This lab covers the following topics:

\begin{itemize}[noitemsep]
\item Format string vulnerability
\item Code injection
\item Shellcode 
\item Reverse shell 
\end{itemize}


\noindent
\fbox{\parbox{\textwidth}{
\noindent
\textbf{Customization by instructor.} Instructors should customize
this lab by choosing a value for the \texttt{DUMMY\_SIZE} constant,
which is used during the compilation of the vulnerable program.
Different values can make the solutions
different. Please pick a value
between \texttt{0} and \texttt{300} for this lab.

\vspace{0.05in}
\begin{center}
\textbf{\large The \texttt{DUMMY\_SIZE} value for this lab is: \underline{\ \ \ \ \ \ \ \ \ \ \ }}
\end{center}
}}


\paragraph{Readings and videos.}
Detailed coverage of the format string attack can be found in the following:

\begin{itemize}
\item Chapter 6 of the SEED Book, \seedbook
\item Section 9 of the SEED Lecture at Udemy, \seedcsvideo
\item The lab also involves reverse shell, which is covered in Chapter 9 of the SEED book.
\end{itemize}



\paragraph{Lab environment.} \seedenvironmentB




\newpage
% *******************************************
% SECTION
% ******************************************* 
\section{Lab Setup}



% -------------------------------------------
% SUBSECTION
% ------------------------------------------- 
\subsection{The Vulnerable Program}

You can get the server program from the lab's website. 
This program has a format string vulnerability.
The 64-bit server program is slight different from the 32-bit version. 
When it runs, it listens to UDP port
\texttt{9090}. Whenever a UDP packet comes to this port, the program
gets the data and invokes \texttt{myprintf()} to print out the data. 
The server is a root daemon, i.e., it runs with the root privilege. 
Inside the \texttt{myprintf()} function, there is a format string
vulnerability. We will exploit this vulnerability to gain the root
privilege.  

\begin{lstlisting}[label=format:code:server, caption={The vulnerable server program
             \texttt{server.c} (can be downloaded from the lab's website)}]
#include <stdio.h>
#include <stdlib.h>
#include <unistd.h>
#include <string.h>
#include <sys/socket.h>
#include <netinet/ip.h>

#define PORT 9090

/* Changing this size will change the layout of the stack.
 * We have added 2 dummy arrays: in main() and myprintf().
 * Instructors can change this value each year, so students 
 * won't be able to use the solutions from the past.   
 * Suggested value: between 0 and 300  */
#ifndef DUMMY_SIZE
#define DUMMY_SIZE 100
#endif

char *secret = "A secret message\n";
unsigned long  target = 0x1122334455667788;

void myprintf(char *msg)
{
    unsigned long int *framep;

    // Copy the rbp value into framep, and print it out
    asm("movq %%rbp, %0" : "=r" (framep));
    printf("Frame Pointer (rbp) inside myprintf(): 0x%.16lx\n", 
            (unsigned long) framep);

    /* Change the size of the dummy array to randomize the parameters 
       for this lab. Need to use the array at least once */
    char dummy[DUMMY_SIZE];  memset(dummy, 0, DUMMY_SIZE);

    // This line has a format-string vulnerability
    printf("The target variable (value, before printf): 0x%.16lx\n", target);
    printf(msg);
    printf("The target variable (value, after printf):  0x%.16lx\n", target);
}

void main()
{
    struct sockaddr_in server;
    struct sockaddr_in client;
    int clientLen;
    char buf[1500];

    /* Change the size of the dummy array to randomize the parameters
       for this lab. Need to use the array at least once */
    char dummy[DUMMY_SIZE];  memset(dummy, 0, DUMMY_SIZE);

    /* Print out some internal data to simplify lab tasks */
    printf("Input buffer (address):        0x%.16lx\n", (unsigned long) buf);
    printf("The secret variable (address): 0x%.16lx\n", 
            (unsigned long) secret);
    printf("The target variable (address): 0x%.16lx\n", 
            (unsigned long) &target);
    myprintf("");

    int sock = socket(AF_INET, SOCK_DGRAM, IPPROTO_UDP);
    memset((char *) &server, 0, sizeof(server));
    server.sin_family = AF_INET;
    server.sin_addr.s_addr = htonl(INADDR_ANY);
    server.sin_port = htons(PORT);

    if (bind(sock, (struct sockaddr *) &server, sizeof(server)) < 0)
        perror("ERROR on binding");

    while (1) {
        bzero(buf, 1500);
        recvfrom(sock, buf, 1500-1, 0,
                 (struct sockaddr *) &client, &clientLen);
        myprintf(buf);
    }
    close(sock);
}
\end{lstlisting}


\paragraph{Compilation.} Compile the above program. For this task,
we will compile it to 64-bit binary code. 
During the compilation, you will see a
warning message. This warning is generated by a countermeasure implemented by
the \texttt{gcc} compiler against format string vulnerabilities. We can
ignore this warning for now. 


\begin{lstlisting}
// Note: N should be replaced by the value set by the instructor
$ gcc -DDUMMY_SIZE=N -z execstack -o server server.c
server.c: In function 'myprintf':
server.c:13:5: warning: format not a string literal and no format arguments 
               [-Wformat-security]
     printf(msg);
     ^
\end{lstlisting}

It should be noted that the program needs to be compiled using 
the \texttt{"-z execstack"} option, which allows the stack to be 
executable. Non-executable stack is a countermeasure against stack-based 
code injection attacks, but 
it can be defeated using the return-to-libc technique. To simplify 
this lab, we simply disable this defeat-able countermeasure. 



\paragraph{For instructors.} 
To prevent students from using the solutions from the past (or from those 
posted on the Internet), instructors can change the 
value for \texttt{DUMMY\_SIZE} by requiring students to compile the 
server code using a different \texttt{DUMMY\_SIZE} value. 
Without the \texttt{-DDUMMY\_SIZE}  
option, \texttt{DUMMY\_SIZE} is set to the default value 100 (defined
in the program). 
When this value changes, the layout of the stack 
will change, and the solution will be different. 
Students should ask their instructors for
the value of \texttt{N}.



% -------------------------------------------
% SUBSECTION
% ------------------------------------------- 
\subsection{Setting Up Containers for the Server} 

We will use two computers for this lab, one for the attacker,
and the other for running the vulnerable server. Instead of 
using two virtual machines, we will containers. 
We run the vulnerable server inside a Docker container, and 
launch the attack from the VM. The files needed for 
container creation is included in the zip file
that can be downloaded from the lab's website. 


\paragraph{\texttt{Dockerfile}}. 
Docker can build images automatically by reading the instructions from a 
file called \dockerfile. 
The content of \dockerfile is shown in the following:

\begin{lstlisting}
FROM ubuntu:20.04
  
RUN  apt-get update  \
     && apt-get -y install \
        iputils-ping \
        net-tools \
        iproute2  \
        nano \
     && apt-get clean


COPY server /
EXPOSE 9090

CMD [ "/bin/bash"]
\end{lstlisting}
 

Our container is built on top of the official \texttt{Ubuntu 20.04} Docker image,
and it installs several additional software packages. 
We copy the server program to the root folder inside the container (make sure 
compile the server code first). We expose the port \texttt{9090}, because
it is used by the server program. The final \texttt{CMD} entry
indicates that when the container starts, the \texttt{/bin/bash} program 
will be executed automatically. 


\paragraph{Building container image.}
\dockerfile is placed inside a folder called \texttt{server\_container}. Enter 
this folder, and run the following command to build the container image (the 
name specified in the \texttt{-t} option is the name of the container image, and you can
use any name you want): 

\begin{lstlisting}
$ docker build -t server . 
\end{lstlisting}



\paragraph{Running the container.}
Once the container image is built, we can use the following command to run an
instance based on the image. With the \texttt{--rm} option, the container instance will be 
automatically removed when it exits (the image will not be removed). 
The \texttt{-it} option is required because we need to get an interactive session
with the container. After the container starts, you will get a shell.
Execute the \texttt{ip addr} command to get the IP address of the container. 
We need it in the attack. 

\begin{lstlisting}
$ docker run --rm -it server 
root@1a7468ce7eaa:/# ip addr
1: lo: <LOOPBACK,UP,LOWER_UP> mtu 65536 qdisc noqueue state ...
    inet 127.0.0.1/8 scope host lo
34: eth0@if35: <BROADCAST,MULTICAST,UP,LOWER_UP> mtu 1500 ...
    inet (*@\textbf{172.17.0.2/16}@*) brd 172.17.255.255 scope global eth0
\end{lstlisting}


\paragraph{Several userful commands.}
Students can refer to the Docker's manual to learn more Docker's commands. 
We list some of the commands that might be useful in this lab. 

\begin{lstlisting}
# List all the running containers. 
$ docker ps 

# Stop a container 
$ docker stop <id>

# Run a bash command on a running container 
$ docker exec -it <id> /bin/bash
\end{lstlisting}
 


% -------------------------------------------
% SUBSECTION
% ------------------------------------------- 
\subsection{Running the Vulnerable Server} 

On the server, we run our server program using the root privilege. We
assume that this program is a privileged root daemon. The server listens to
port \texttt{9090}. When it runs,  it prints out some internal data 
that are needed for the attacks. In the real world, servers do not 
print out this kind of data. We do that to simplify the lab tasks. 
In real attacks, attackers need to find way to figure out these data
or guess them. 

\begin{lstlisting}
// On the server container
# /server_64
Input buffer (address):        0x00007fffffffd9c0
The secret variable (address): 0x0000555555556008
The target variable (address): 0x0000555555558010
Frame Pointer (rbp) inside myprintf():      0x00007fffffffd910
The target variable (value, before printf): 0x1122334455667788
The target variable (value, after printf):  0x1122334455667788
\end{lstlisting}


On the client VM, we can send data to the server using 
the \texttt{nc} command, where the flag \texttt{"-u"} means UDP (the server
program is a UDP server). The IP address in the following example should be 
replaced by the actual IP address of the server container. 
Yon can send any data to the server. The server program is supposed
to print out whatever is sent by you. 


\begin{lstlisting}
// Send out a "hello" message to the server
$ echo hello | nc -u 172.17.0.2 9090 

// Send the content of badfile to the server
$ nc -u 172.17.0.2 9090 < badfile
\end{lstlisting}


 




% *******************************************
% SECTION
% ******************************************* 
\section{Lab Tasks}


The lab tasks are the same as the 32-bit version, so we will not 
repeat them. Students should read the 32-bit version of the 
lab description for detailed tasks. Here, we will
only emphasize one of the challenges caused by
the x64 architecture. 


\paragraph{Challenge.} The main challenge caused by x64 is the zeros in the address.
Although the x64 architecture
supports 64-bit address space, only the address from
\texttt{0x00} through \texttt{0x00007FFFFFFFFFFF} is allowed. That means for
every address (8 bytes), the highest two bytes are always zeros.
This causes a problem.

In the attack, we need to place addresses inside the format string. For 
32-bit programs, we can put the addresses anywhere, because there 
are no zeros inside the address. We can no longer do this
for the 64-bit programs. If you put an address in the middle of 
your format string, when \texttt{printf()} parses the 
format string, it will stop the parsing when it sees a zero. Basically,
anything after the first zero in a format string will not 
be considered as part of the format string. 

The problem caused by zeros is different from that
in the buffer overflow attack, in which,
zeros will terminate the memory copy if \texttt{strpcy()} is used. 
Here, we do not have memory copy in the program, 
so we can have zeros in our input, but where to put them
is critical. 
There are many ways to solve this problem, and 
we leave this to students. In the lab report, students
should explain how they have solved this problem. 


\paragraph{Note.} If students have already done 
the 32-bit version of this lab, they can skip all the 
warm-up tasks, and only conduct the code injection attack (Task 6
and Task 7). 



% *******************************************
% SECTION
% ******************************************* 
\section{Submission}

\seedsubmission



\end{document}
