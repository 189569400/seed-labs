%%%%%%%%%%%%%%%%%%%%%%%%%%%%%%%%%%%%%%%%%%%%%%%%%%%%%%%%%%%%%%%%%%%%%%
%%  Copyright by Wenliang Du.                                       %%
%%  This work is licensed under the Creative Commons                %%
%%  Attribution-NonCommercial-ShareAlike 4.0 International License. %%
%%  To view a copy of this license, visit                           %%
%%  http://creativecommons.org/licenses/by-nc-sa/4.0/.              %%
%%%%%%%%%%%%%%%%%%%%%%%%%%%%%%%%%%%%%%%%%%%%%%%%%%%%%%%%%%%%%%%%%%%%%%

\documentclass[11pt]{article}

\usepackage[most]{tcolorbox}
\usepackage{times}
\usepackage{epsf}
\usepackage{epsfig}
\usepackage{amsmath, alltt, amssymb, xspace}
\usepackage{wrapfig}
\usepackage{fancyhdr}
\usepackage{url}
\usepackage{verbatim}
\usepackage{fancyvrb}
\usepackage{adjustbox}
\usepackage{listings}
\usepackage{color}
\usepackage{subfigure}
\usepackage{cite}
\usepackage{sidecap}
\usepackage{pifont}
\usepackage{mdframed}
\usepackage{textcomp}
\usepackage{enumitem}


% Horizontal alignment
\topmargin      -0.50in  % distance to headers
\oddsidemargin  0.0in
\evensidemargin 0.0in
\textwidth      6.5in
\textheight     8.9in 

\newcommand{\todo}[1]{
\vspace{0.1in}
\fbox{\parbox{6in}{TODO: #1}}
\vspace{0.1in}
}


\newcommand{\unix}{{\tt Unix}\xspace}
\newcommand{\linux}{{\tt Linux}\xspace}
\newcommand{\minix}{{\tt Minix}\xspace}
\newcommand{\ubuntu}{{\tt Ubuntu}\xspace}
\newcommand{\setuid}{{\tt Set-UID}\xspace}
\newcommand{\openssl} {\texttt{openssl}}


\pagestyle{fancy}
\lhead{\bfseries SEED Labs}
\chead{}
\rhead{\small \thepage}
\lfoot{}
\cfoot{}
\rfoot{}


\definecolor{dkgreen}{rgb}{0,0.6,0}
\definecolor{gray}{rgb}{0.5,0.5,0.5}
\definecolor{mauve}{rgb}{0.58,0,0.82}
\definecolor{lightgray}{gray}{0.90}


\lstset{%
  frame=none,
  language=,
  backgroundcolor=\color{lightgray},
  aboveskip=3mm,
  belowskip=3mm,
  showstringspaces=false,
%  columns=flexible,
  basicstyle={\small\ttfamily},
  numbers=none,
  numberstyle=\tiny\color{gray},
  keywordstyle=\color{blue},
  commentstyle=\color{dkgreen},
  stringstyle=\color{mauve},
  breaklines=true,
  breakatwhitespace=true,
  tabsize=3,
  columns=fullflexible,
  keepspaces=true,
  escapeinside={(*@}{@*)}
}

\newcommand{\newnote}[1]{
\vspace{0.1in}
\noindent
\fbox{\parbox{1.0\textwidth}{\textbf{Note:} #1}}
%\vspace{0.1in}
}


%% Submission
\newcommand{\seedsubmission}{You need to submit a detailed lab report, with screenshots,
to describe what you have done and what you have observed.
You also need to provide explanation
to the observations that are interesting or surprising.
Please also list the important code snippets followed by
explanation. Simply attaching code without any explanation will not
receive credits.}

%% Book
\newcommand{\seedbook}{\textit{Computer \& Internet Security: A Hands-on Approach}, 2nd
Edition, by Wenliang Du. See details at \url{https://www.handsonsecurity.net}.}

%% Videos
\newcommand{\seedisvideo}{\textit{Internet Security: A Hands-on Approach},
by Wenliang Du. See details at \url{https://www.handsonsecurity.net/video.html}.}

\newcommand{\seedcsvideo}{\textit{Computer Security: A Hands-on Approach},
by Wenliang Du. See details at \url{https://www.handsonsecurity.net/video.html}.}

%% Lab Environment
\newcommand{\seedenvironment}{This lab has been tested on our pre-built
Ubuntu 16.04 VM, which can be downloaded from the SEED website. }

\newcommand{\seedenvironmentA}{This lab has been tested on our pre-built
Ubuntu 16.04 VM, which can be downloaded from the SEED website. }

\newcommand{\seedenvironmentB}{This lab has been tested on our pre-built
Ubuntu 20.04 VM, which can be downloaded from the SEED website. }

\newcommand{\seedenvironmentAB}{This lab has been tested on our pre-built
Ubuntu 16.04 and 20.04 VMs, which can be downloaded from the SEED website. }

\newcommand{\nodependency}{Since we use containers to set up the lab environment, 
this lab does not depend too much on our SEED VM. You can do this lab
using other VMs or physical machines. }







\newcommand{\seedlabcopyright}[1]{
\vspace{0.1in}
\fbox{\parbox{6in}{\small Copyright \copyright\ {#1}\ \ by Wenliang Du.\\
      This work is licensed under a Creative Commons
      Attribution-NonCommercial-ShareAlike 4.0 International License.
      If you remix, transform, or build upon the material, 
      this copyright notice must be left intact, or reproduced in a way that is reasonable to
      the medium in which the work is being re-published.}}
\vspace{0.1in}
}






\lhead{\bfseries SEED Labs -- Race Condition Vulnerability Lab}

\def \code#1 {\fbox{\scriptsize{\texttt{#1}}}}

\begin{document}

\begin{center}
{\LARGE Race Condition Vulnerability Lab}

\vspace{0.05in}
Updated on July 26, 2020
\end{center}


\seedlabcopyright{2006 - 2020}



% *******************************************
% SECTION
% ******************************************* 
\section{Overview}

The learning objective of this lab is for students to gain the first-hand
experience on the race-condition vulnerability by putting what they have learned
about the vulnerability from class into actions.
A race condition occurs when multiple processes access and manipulate the same
data concurrently, and the outcome of the execution depends on the particular
order in which the access takes place. If a privileged program has a race-condition 
vulnerability, attackers can run a parallel process to ``race'' against
the privileged program, with an intention to change the behaviors 
of the program.

In this lab, students will be given a program with a race-condition
vulnerability; their task is to develop a scheme to exploit 
the vulnerability and gain the root privilege.  In addition to the
attacks, students will be guided to walk through several protection
schemes that can be used to counter the race-condition attacks.
Students need to evaluate
whether the schemes work or not and explain why. 
This lab covers the following topics:

\begin{itemize}[noitemsep]
\item Race condition vulnerability
\item Sticky symlink protection
\item Principle of least privilege
\end{itemize}




\paragraph{Readings and videos.}
Detailed coverage of the race condition attack can be found in the following:

\begin{itemize}
\item Chapter 7 of the SEED Book, \seedbook
\item Section 6 of the SEED Lecture at Udemy, \seedcsvideo
\end{itemize}


\paragraph{Related topics.}
There are three more labs related to race condition. One 
is the Dirty COW attack lab, which exploits a race condition vulnerability
inside the OS kernel (Chapter 8 of the SEED book covers this attack).
The other two are Meltdown and Spectre attack labs (Chapters 13 and 14 
of the SEED book). They exploit 
race conditions inside CPU. These four labs provide 
a comprehensive coverage of the race condition problem at 
different levels of a computer system, from application, kernel, 
to hardware. 


\paragraph{Lab environment.} \seedenvironment


% *******************************************
% SECTION
% ******************************************* 
\section{Lab Tasks}


% -------------------------------------------
% SUBSECTION
% ------------------------------------------- 
\subsection{Initial Setup}

Ubuntu 10.10 and later come with a built-in protection against race condition 
attacks. This scheme works by restricting who can follow a symlink. 
According to the documentation, ``symlinks in world-writable sticky 
directories (e.g. {\tt /tmp}) cannot be followed if the follower and directory 
owner do not match the symlink owner.''
In this lab, we need to disable this protection. You can achieve
that using the following commands:

\begin{lstlisting}
  // On Ubuntu 12.04, use the following:
  $ sudo sysctl -w kernel.yama.protected_sticky_symlinks=0

  // On Ubuntu 16.04, use the following:
  $ sudo sysctl -w fs.protected_symlinks=0
\end{lstlisting}



% -------------------------------------------
% SUBSECTION
% ------------------------------------------- 
\subsection{A Vulnerable Program}


The following program is a seemingly harmless program. It contains a race-condition
vulnerability. 

\begin{lstlisting}
/*  vulp.c  */

#include <stdio.h>
#include<unistd.h>

int main()
{
   char * fn = "/tmp/XYZ";
   char buffer[60];
   FILE *fp;

   /* get user input */
   scanf("%50s", buffer );

   if(!access(fn, W_OK)){                 (*@\ding{192}@*)
      fp = fopen(fn, "a+");               (*@\ding{193}@*)
      fwrite("\n", sizeof(char), 1, fp);
      fwrite(buffer, sizeof(char), strlen(buffer), fp);
      fclose(fp);
   }
   else printf("No permission \n");
}
\end{lstlisting}

The program above is a root-owned \setuid program;
it appends a string of user input to
the end of a temporary file {\tt /tmp/XYZ}. Since the code runs
with the root privilege, i.e., its effective use ID is zero, it 
can overwrite any file. To prevent itself from accidentally
overwriting other people's file, the program first 
checks whether the real user ID has the access permission to the file
{\tt /tmp/XYZ}; that is the purpose of the {\tt access()} call in Line
\ding{192}.  If the real user ID indeed has 
the right, the program opens the file in Line \ding{193} and append the user
input to the file. 

At first glance the program does not seem to have any problem.
However, there is a race condition vulnerability in this program: due to the 
time window between the check ({\tt access()}) and 
the use ({\tt fopen()}), there is a possibility that the file used by
{\tt access()} is different from the file used by {\tt fopen()}, even
though they have the same file name {\tt /tmp/XYZ}.  If a malicious 
attacker can somehow make {\tt /tmp/XYZ} a symbolic link pointing to
a protected file, such as \texttt{/etc/passwd}, inside the time window, 
the attacker can cause the user input
to be appended to \texttt{/etc/passwd} and as a result gain the root
privilege. The vulnerable runs with the root privilege, so
it can overwrite any file.


\paragraph{Set up the \setuid program.}
We first compile the above code, and turn its binary into a \setuid program that is owned by the
root. The following commands achieve this goal:

\begin{lstlisting}
$ gcc vulp.c -o vulp
$ sudo chown root vulp
$ sudo chmod 4755 vulp
\end{lstlisting}



% -------------------------------------------
% SUBSECTION
% ------------------------------------------- 
\subsection{Task 1: Choosing Our Target}

We would like to exploit the race condition vulnerability in
the vulnerable program.
We choose to target the password file \texttt{/etc/passwd}, which is not writable by
normal users. By exploiting the vulnerability, we would like to
add a record to the password file, with a goal of
creating a new user account that has the root privilege.
Inside the password file, each user has an entry, which consists of seven fields
separated by colons~(:). The entry for the root user is listed below.
For the root user, the third field~(the user ID field) has a value zero. Namely, when the root
user logs in, its process's user ID is set to zero, giving the process the root privilege.
Basically, the power of the root account does not come from its name, but instead from
the user ID field. If we want to create an account with the root privilege,
we just need to put a zero in this field.

\begin{lstlisting}[backgroundcolor=,]
   root:x:0:0:root:/root:/bin/bash
\end{lstlisting}

Each entry also contains a password field, which is the second field. In the example above, the
field is set to \texttt{"x"}, indicating that the password is stored in another file called
\texttt{/etc/shadow}~(the shadow file). If we follow this example, we have to use the race condition
vulnerability to modify both password and shadow files, which is not
very hard to do. However, there is a simpler solution. Instead of putting \texttt{"x"} in the
password file, we can simply put the password there, so the operating system will not look for
the password from the shadow file.

The password field does not hold the actual password; it holds the one-way hash value of the
password. To get such a value for a given password, we can add a new user in our own system using
the \texttt{adduser} command, and then get the one-way hash value of our password from
the shadow file.  Or we can simply copy the value from the \texttt{seed} user's entry,
because we know its password is \texttt{dees}. Interestingly, there is a magic value
used in \ubuntu live CD for a password-less account, and the magic value is
\texttt{U6aMy0wojraho} (the 6th character is zero, not letter \texttt{O}). If we put this value
in the password field of a user entry, we only need to hit the return key when prompted for
a password.


\paragraph{Task.} To verify whether the magic password works or not, we
manually (as a superuser) add the following entry to the end of the \texttt{/etc/passwd} file. 
Please report whether you can log into the \texttt{test} account without
typing a password, and check whether you have the root privilege.

\begin{lstlisting}[backgroundcolor=,]
    test:U6aMy0wojraho:0:0:test:/root:/bin/bash
\end{lstlisting}

After this task, please remove this entry from the password file. In the
next task, we need to achieve this goal as a normal user. Clearly, we are
not allowed to do that directly to the password file, but we can exploit a
race condition in a privileged program to achieve the same goal.


\paragraph{Warning.}
In the past, some students accidentally emptied the {\tt /etc/passwd} file 
during the attack (this could be caused by some race condition problems
inside the OS kernel). If you lose
the password file, you will not be able to log in again. To avoid this 
trouble, please make a copy of the original password file or take a
snapshot of the VM. This way, you can easily recover from the mishap. 



% -------------------------------------------
% SUBSECTION
% ------------------------------------------- 
\subsection{Task 2.A: Launching the Race Condition Attack}

The goal of this task is to exploit the race condition vulnerability in 
the vulnerable \setuid program listed earlier. The ultimate goal is to gain
the root privilege. 

The most critical step (i.e., making
\texttt{/tmp/XYZ} point to the password file) of our race condition attack must occur within
the window between check and use; namely between the \texttt{access()} 
and the \texttt{fopen()} calls in the vulnerable program.
Since we cannot modify the vulnerable program,
the only thing that we can do is to run our attacking program in parallel
to ``race'' against the target program, 
hoping to win the race condition, i.e., changing the link 
within that critical window.
Unfortunately, we cannot achieve the perfect timing. Therefore, 
the success of attack is probabilistic.
The probability of successful attack might be quite low if the window
are small. You need to think about how to increase the 
probability. For example,  you can run the vulnerable program for many
times; you only need to achieve success once among all these trials.



\paragraph{Creating symbolic links.}
You can call C function {\tt symlink()} to create symbolic links in your program.
Since \linux does not allow one to create a link if the link already exists,
we need to delete the old link first.
The following C code snippet shows how to remove a link and then make
{\tt /tmp/XYZ} point to {\tt /etc/passwd}:

\begin{lstlisting}
  unlink("/tmp/XYZ");
  symlink("/etc/passwd","/tmp/XYZ");
\end{lstlisting}

You can also use \linux command {\tt "ln -sf"} to create symbolic links.
Here the {\tt "f"} option means that if the link exists, remove the
old one first. The implementation of the
{\tt "ln"} command actually uses {\tt unlink()} and {\tt symlink()}.


\paragraph{Running the attack program.}
After you have implemented the attack program, you should run it 
in the background, and then run the vulnerable program 
in parallel. If the attack fails, the vulnerable program
will crash, so you need to run the vulnerable program repeatedly until the 
attack is successful. 



\paragraph{Running the vulnerable program and monitoring results.}
Since you need to run the vulnerable program for many
times, you need to write a program to automate the attack process. 
To avoid manually typing an input to the vulnerable program \texttt{vulp}, you can use 
input redirection. Namely, you save your input in a file, and ask
\texttt{vulp} to get the input from this file using \texttt{"vulp <
inputFile"}.

It may take a while before our attack can successfully modify the
password file, so we need a way to automatically detect whether the attack is
successful or not. There are many ways to do that; an easy way 
is to monitor the timestamp of the file.
The following shell script runs the \texttt{"ls -l"} command, which
outputs several piece of information about a
file, including the last modified time. By comparing the outputs of the
command with the ones produced previously, we can tell
whether the file has been modified or not. 


The following shell script runs the vulnerable program (\texttt{vulp}) in a loop,
using \texttt{passwd\_input} as the input. If the attack is successful, i.e.,
the \texttt{passwd} is modified, the shell script will stop. 
You do need to be a little bit patient. 
Normally, you should be able to succeed within 5 minutes. 



\begin{lstlisting}
#!/bin/bash

CHECK_FILE="ls -l /etc/passwd"
old=$($CHECK_FILE)
new=$($CHECK_FILE)
while [ "$old" == "$new" ]    (*@\reflectbox{\ding{217}} \textbf{Check if /etc/passwd is modified}@*)
do
   ./vulp < passwd_input      (*@\reflectbox{\ding{217}} \textbf{Run the vulnerable program}@*)
   new=$($CHECK_FILE)
done
echo "STOP... The passwd file has been changed"
\end{lstlisting}



\paragraph{A Note.}
If after 10 minutes, your attack is still not 
successful, you can stop the attack, and check the ownership
of the \texttt{/tmp/XYZ} file. If the owner of this file
becomes root, manually delete this file, and try your 
attack again, until your attack becomes successful. 
Please document this observation in your lab report. 
In Task 2.B, we will explain the reason and provide
an improved attack method. 



% -------------------------------------------
% SUBSECTION
% ------------------------------------------- 
\subsection{Task 2.B: An Improved Attack Method}

In Task 2.A, if you have done everything correctly, but still could not succeed
in the attack, check the ownership of \texttt{/tmp/XYZ}. You will find out    
that \texttt{/tmp/XYZ}'s owner has become root (normally, it should be \texttt{seed}). 
If this happens, your attack will never succeed, because your attack
program, running with the \texttt{seed} privilege, can no longer remove or
\texttt{unlink()} it.  
This is because the \texttt{/tmp}
folder has a ``sticky'' bit on, meaning that only the owner of the file can
delete the file, even though the folder is world-writable. 


In Task 2.A, we let you use the root's privilege to delete \texttt{/tmp/XYZ}, 
and then try your attack again. The undesirable condition happens randomly,
so by repeating the attack (with the ``help'' from root), you will eventually
succeed in Task 2.A. Obviously, getting help from root is not a real attack. 
We would like to get rid of that, and do it without root's help.


The main reason for that undesirable situation is that 
our attack program has a problem, a race condition problem, the exact problem that
we are trying to exploit in the victim program. How ironic!  
In the past, when we saw that problem, we simply advised students to 
delete the file and try the attack again. Thanks to one of my students, 
who was determined to figure out what the problem was. Because of his effort,
we finally understand why and have an improved solution. 


The main reason for the situation to happen is that the attack program is 
context switched out right after it removes {\tt /tmp/XYZ} (i.e., \texttt{unlink()}),
but before it links the name to another file (i.e., \texttt{symlink()}. 
Remember, the action to remove the existing 
symbolic link and create a new one is not atomic (it involves two 
separate system calls), so if the context switch occurs in the middle
(i.e., right after the removal of {\tt /tmp/XYZ}),
and the target \setuid program gets a chance to run its 
{\tt fopen(fn, "a+")} statement, it will create a new file with root being the owner.
After that, your attack program can no longer make changes to {\tt /tmp/XYZ}.


Basically, using the \texttt{unlink()} and \texttt{symlink()} approach, 
we have a race condition in our attack program. Therefore, while we 
are trying to exploit the race condition in the target program,
the target program may accidentally ``exploit'' the race condition
in our attack program, defeating our attack.  


To solve this problem, we need to make 
\texttt{unlink()} and \texttt{symlink()} atomic. Fortunately, there 
is a system call that allows us to achieve that. More accurately,
it allows us to atomically swap two symbolic links. 
The following program first makes two symbolic links \texttt{/tmp/XYZ}
and \texttt{/tmp/ABC}, and then using the
\texttt{SYS\_renameat2} system call to atomically switch them. 
This allows us to change what \texttt{/tmp/XYZ} points to
without introducing any race condition. 
It should be noted that there is no function
wrapper for the \texttt{SYS\_renameat2} system call in the \texttt{libc} library
on Ubuntu 16.04, so we have to invoke the system call using 
\texttt{syscall()}, instead of invoking it like a normal function call. 


\begin{lstlisting}
#include <unistd.h>
#include <sys/syscall.h>
#include <linux/fs.h>

int main()
{
   unsigned int flags = RENAME_EXCHANGE;

   unlink("/tmp/XYZ"); symlink("/dev/null",   "/tmp/XYZ");
   unlink("/tmp/ABC"); symlink("/etc/passwd", "/tmp/ABC");

   syscall(SYS_renameat2, 0, "/tmp/XYZ", 0, "/tmp/ABC", flags);

   return 0;
}
\end{lstlisting}


\paragraph{Tasks.} Please revise your attack program using 
this new strategy, and try your attack again. If everything 
is done correctly, your attack should be able to succeed. 




\subsection{Task 3: Countermeasure: Applying the Principle of Least Privilege}

The fundamental problem of the vulnerable program in this lab is 
the violation of the {\em Principle of Least Privilege}. 
The programmer does understand that the user who runs the program 
might be too powerful, so he/she introduced {\tt access()} to limit the user's 
power. However, this is not the proper approach. A better
approach is to apply the {\em Principle of Least Privilege}; 
namely, if users do not need certain privilege, the privilege
needs to be disabled.

We can use {\tt seteuid{}} system call to temporarily disable
the root privilege, and later enable it if necessary. Please use 
this approach to fix the vulnerability in the program, and then
repeat your attack. Will you be able to succeed? Please report your
observations and provide explanation.



% -------------------------------------------
% SUBSECTION
% ------------------------------------------- 
\subsection{Task 4: Countermeasure: Using \ubuntu's Built-in Scheme}

Ubuntu 10.10 and later come with a built-in protection scheme against race condition
attacks. In this task, you need to turn the protection back on using the
following commands:

\begin{lstlisting}
  // On Ubuntu 12.04, use the following command:
  $ sudo sysctl -w kernel.yama.protected_sticky_symlinks=1

  // On Ubuntu 16.04, use the following command:
  $ sudo sysctl -w fs.protected_symlinks=1
\end{lstlisting}

Conduct your attack after the protection is turned on.  
Please describe your observations. Please also explain
the followings: (1) How does this protection scheme work?
(2) What are the limitations of this scheme?






% *******************************************
% SECTION
% *******************************************
\section{Submission}

\seedsubmission


\end{document}
